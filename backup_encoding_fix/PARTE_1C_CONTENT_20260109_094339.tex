%% Continuação do Módulo 1: A Entrevista Estratégica

\section{Tema 1124 STJ: A Instrução Completa do PA}

\begin{novidade}
Esta tese muda COMPLETAMENTE a estratégia de instrução do processo administrativo antes do ajuizamento.

Se você não juntar todos os documentos no PA, a DIB pode ser fixada na CITAÇO, não na DER — e seu cliente perde MESES ou ANOS de atrasados!
\end{novidade}

\subsection{O Que Diz o Tema 1124}

O \textbf{Tema 1124 do STJ} (julgado em outubro/2025) fixou regras sobre \textbf{interesse de agir previdenciário} e \textbf{termo inicial do benefício (DIB)}:

\textbf{TESE FIXADA:}

O segurado deve apresentar \textbf{requerimento administrativo apto} com \textbf{documentação minimamente suficiente} para configurar interesse de agir judicial.

\subsection{Impacto Prático: DIB na DER vs DIB na Citação}

\begin{table}[H]
\centering
\caption{Impacto da DIB: DER vs Citação (Tema 1124 STJ)}
\begin{tabular}{|p{5cm}|c|p{5cm}|}
\hline
\textbf{Situação} & \textbf{DIB} & \textbf{Impacto Financeiro} \\
\hline
Provas já estavam no PA & \textbf{DER} & Atrasados desde o requerimento \\
\hline
Provas NÃO levadas ao INSS por desídia do segurado & \textbf{Citação} & Perde atrasados entre DER e citação \\
\hline
Prova surgiu após ajuizamento (era impossível antes) & Citação ou posterior & Depende do caso \\
\hline
INSS descumpriu dever de intimar para complementação & \textbf{DER} & Interesse de agir configurado \\
\hline
\end{tabular}
\end{table}

\begin{estrategiaCJP}
\textbf{REGRA DE OURO:} Junte TODA a documentação possível no processo administrativo ANTES de eventual judicialização. Isso garante DIB na DER, maximizando atrasados.

\textbf{DOCUMENTOS ESSENCIAIS A JUNTAR NO PA:}
\begin{itemize}
    \item[\cmark] PPP/LTCAT (tempo especial)
    \item[\cmark] CTPS completa com todos os vínculos
    \item[\cmark] Declaração de atividade rural + prova material
    \item[\cmark] Comprovantes de contribuição CI/facultativo
    \item[\cmark] Laudos médicos (se incapacidade)
    \item[\cmark] CTC de tempo público
    \item[\cmark] Certificado de Reservista
\end{itemize}

\textbf{EXCEÇO:} Se o documento era IMPOSSÍVEL de obter na via administrativa (ex: empresa falida sem PPP), a DIB pode ser fixada posteriormente.
\end{estrategiaCJP}

\subsection{Exemplo Prático: O Custo de Não Instruir o PA}

\textbf{Caso SEM instrução completa:}
\begin{itemize}
    \item DER: 01/01/2025
    \item Citação: 01/07/2025 (6 meses depois)
    \item RMI: R\$ 3.500/mês
    \item DIB fixada na citação (provas não estavam no PA)
    \item \textbf{Atrasados perdidos:} 6 meses $\times$ R\$ 3.500 = \textbf{R\$ 21.000}
\end{itemize}

\textbf{Caso COM instrução completa:}
\begin{itemize}
    \item DER: 01/01/2025
    \item Citação: 01/07/2025
    \item RMI: R\$ 3.500/mês
    \item DIB fixada na DER (provas já estavam no PA)
    \item \textbf{Atrasados garantidos:} R\$ 21.000 \textbf{preservados}
\end{itemize}

\subsection{Checklist de Instrução do PA (Tema 1124)}

\begin{acaoImediata}
\textbf{ANTES DE PROTOCOLAR O REQUERIMENTO ADMINISTRATIVO:}
\begin{itemize}
    \item[$\square$] PPP de TODOS os empregadores com exposição a agentes nocivos
    \item[$\square$] CTPS com todas as anotações legíveis
    \item[$\square$] Declaração rural com início de prova material
    \item[$\square$] Comprovantes de contribuição CI (GPS, DAS, etc.)
    \item[$\square$] Laudos médicos atualizados (se incapacidade)
    \item[$\square$] CTC de tempo público averbado
    \item[$\square$] Certificado de Reservista (se aplicável)
    \item[$\square$] Contracheques para comprovar salários (se divergência)
    \item[$\square$] Extrato FGTS (para vínculos PEXT)
\end{itemize}

\textbf{SE ALGUM DOCUMENTO NÃO EXISTE:}
\begin{itemize}
    \item[$\square$] Documentar a impossibilidade de obtenção
    \item[$\square$] Protocolar pedido formal à empresa/órgão
    \item[$\square$] Guardar evidência de que tentou obter
\end{itemize}

\textbf{APÓS O INDEFERIMENTO:}
\begin{itemize}
    \item[$\square$] Verificar se INSS intimou para complementação
    \item[$\square$] Se intimou e segurado não complementou $\rightarrow$ DIB na citação
    \item[$\square$] Se NÃO intimou $\rightarrow$ interesse de agir configurado $\rightarrow$ DIB na DER
\end{itemize}
\end{acaoImediata}

\section{Matriz de Priorização: Onde Focar Primeiro}

Nem toda armadilha tem o mesmo impacto. Use esta matriz para \textbf{priorizar} qual armadilha investigar primeiro quando o tempo é curto.

\subsection{Matriz de Impacto vs Dificuldade de Prova}

\begin{conceitoChave}
\textbf{ALTO IMPACTO + FÁCIL DE PROVAR (Prioridade MÁXIMA)}
\begin{itemize}
    \item[\cmark] \textbf{1.} Tempo Militar/CTC (Armadilha 4)
    \item[\cmark] \textbf{2.} Vínculos PEXT com CTPS (Armadilha 1)
    \item[\cmark] \textbf{3.} Atividades Concomitantes (Armadilha 2)
\end{itemize}

\textbf{ALTO IMPACTO + MÉDIO DIFÍCIL (Prioridade ALTA)}
\begin{itemize}
    \item \textbf{4.} Atividade Especial com PPP (Armadilha 5)
    \item \textbf{5.} Pecúlios e Abonos (Armadilha 8)
\end{itemize}

\textbf{MÉDIO IMPACTO + FÁCIL DE PROVAR (Prioridade MÉDIA)}
\begin{itemize}
    \item \textbf{6.} Afastamentos Indevidos (Armadilha 3)
    \item \textbf{7.} Salário-Maternidade no PBC (Armadilha 6)
\end{itemize}

\textbf{ALTO IMPACTO + DIFÍCIL DE PROVAR (Prioridade CONDICIONAL)}
\begin{itemize}
    \item[\xmark] \textbf{8.} Tempo Rural (Armadilha 7) — Só priorize se houver fortes indícios
\end{itemize}
\end{conceitoChave}

\subsection{Regra de Ouro da Priorização}

\textbf{Comece sempre pelas armadilhas que têm:}
\begin{enumerate}
    \item \textbf{Alto impacto} (adicionam +2 anos ou mais de tempo)
    \item \textbf{Prova documental disponível} (certificado, CTPS, CTC, etc.)
\end{enumerate}

\textbf{Deixe por último as armadilhas que dependem de:}
\begin{itemize}
    \item Testemunhas (tempo rural)
    \item Empresas que faliram (PPP perdido)
    \item Documentos que o cliente não tem e não conseguirá
\end{itemize}

\section{Fluxograma: Da Entrevista ao Diagnóstico Completo}

O processo completo do Pilar 1 segue este fluxo decisório:

\begin{estrategiaCJP}
\textbf{FLUXOGRAMA DO PILAR 1: ENTREVISTA ESTRATÉGICA}

\textbf{INÍCIO:} Cliente na Consulta Inicial

$\downarrow$

\textbf{ETAPA 0: VERIFICAÇO INICIAL (2026)}
\begin{itemize}
    \item Cliente cadastrou procuração eletrônica? (Seção 1.2)
    \item Se SIM $\rightarrow$ Acesso direto ao CNIS
    \item Se NÃO $\rightarrow$ Orientar cadastro
\end{itemize}

$\downarrow$

\textbf{ETAPA 1: PREPARAÇO DA ENTREVISTA}
\begin{itemize}
    \item Acessar CNIS (via procuração ou documento)
    \item Solicitar CTPS e documentos
    \item Agendar 1-2 horas de reunião
\end{itemize}

$\downarrow$

\textbf{ETAPA 2: CONDUZIR A ENTREVISTA}
\begin{itemize}
    \item Pedir narrativa cronológica
    \item Usar 8 perguntas-gatilho
    \item Anotar TUDO que o cliente diz
    \item Verificar CNIS em tempo real
\end{itemize}

$\downarrow$

\textbf{ETAPA 3: MAPEAR ARMADILHAS DETECTADAS}
\begin{itemize}
    \item[$\square$] Armadilha 1 (PEXT)?
    \item[$\square$] Armadilha 2 (Concomitantes)?
    \item[$\square$] Armadilha 3 (Afastamentos)?
    \item[$\square$] Armadilha 4 (Militar/CTC)?
    \item[$\square$] Armadilha 5 (Especial)?
    \item[$\square$] Armadilha 6 (Sal-Maternidade)?
    \item[$\square$] Armadilha 7 (Rural)?
    \item[$\square$] Armadilha 8 (Pecúlios)?
\end{itemize}

$\downarrow$

\textbf{ETAPA 4: SOLICITAR DOCUMENTAÇO}
\begin{itemize}
    \item Usar checklist da Seção 1.5
    \item Priorizar pela Matriz 1.7
    \item Dar prazo: 7-15 dias
    \item Alertar sobre Tema 1124 (instrução PA)
\end{itemize}
\end{estrategiaCJP}

\begin{conceitoChave}
\textbf{DECISÃO: Cliente trouxe documentos?}

\textbf{SIM:}\\
$\rightarrow$ Ir para o PILAR 2 (Módulos 2-3)\\
$\rightarrow$ Auditar CNIS (Módulo 2)\\
$\rightarrow$ Confirmar Armadilhas (Módulo 3)\\
$\rightarrow$ Listar o que precisa corrigir

\textbf{NÃO:}\\
$\rightarrow$ Reagendar (dar mais prazo)\\
$\rightarrow$ Ou prosseguir com o que tem

\textbf{RESULTADO FINAL:}\\
PILAR 3: Acertos (Módulo 4)
\end{conceitoChave}

\section{Casos Práticos: 3 Entrevistas Que Mudaram Tudo}

\subsection{CASO 1: O Soldador Esquecido}

\textbf{Perfil:}
\begin{itemize}
    \item Nome: Roberto, 59 anos
    \item Profissão: Mecânico industrial
    \item CNIS: 33 anos de contribuição
    \item Expectativa inicial: Aposentar em 2027 (faltavam 3 anos)
\end{itemize}

\textbf{Entrevista (Gatilho 3 - Atividade Especial):}

\textit{Advogado:} ``Roberto, me conta sua história profissional desde o começo.''

\textit{Roberto:} ``Comecei como aprendiz de mecânico aos 16 anos. Trabalhei em várias empresas. De 1990 a 2002, eu era soldador na [Metalúrgica XYZ]. Depois virei supervisor e não soldava mais.''

\textit{Advogado:} ``Soldador? Você usava equipamento de proteção? Tinha exposição a ruído, calor, fumaça?''

\textit{Roberto:} ``Sim, o dia todo. Usava máscara e protetor auricular, mas mesmo assim era muito ruído. A empresa entregava EPI, mas não resolvia o problema.''

\textbf{Descoberta:}\\
12 anos de atividade especial (1990-2002) não reconhecidos.

\textbf{Documentação Obtida:}\\
PPP da Metalúrgica XYZ (empresa ainda existe), comprovando exposição a:
\begin{itemize}
    \item Ruído acima de 85dB
    \item Fumos metálicos
    \item Radiação não ionizante (solda)
\end{itemize}

\textbf{Impacto:}
\begin{itemize}
    \item 12 anos especiais $\times$ 1,4 = 16,8 anos comuns
    \item \textbf{+4,8 anos de bônus de tempo}
    \item Nova data de aposentadoria: \textbf{Imediato (2026)}
\end{itemize}

\textbf{Ganho financeiro:}
\begin{itemize}
    \item Aposentar 2 anos antes = 24 meses $\times$ R\$ 4.200
    \item \textbf{Ganho imediato: R\$ 100.800}
\end{itemize}

\subsection{CASO 2: A Professora Rural}

\textbf{Perfil:}
\begin{itemize}
    \item Nome: Silvia, 54 anos
    \item Profissão: Professora municipal
    \item CNIS: 27 anos de contribuição
    \item Expectativa inicial: Aposentar em 2028 (faltavam 3 anos para 30)
\end{itemize}

\textbf{Entrevista (Gatilho 1 - Rural + Gatilho 2 - CTC):}

\textit{Advogado:} ``Silvia, onde você nasceu? Seus pais trabalhavam com o quê?''

\textit{Silvia:} ``Nasci em Registro-SP, zona rural. Meu pai era lavrador, tinha um sítio de banana. Morei lá até os 18 anos, quando fui fazer magistério. Ajudava meus pais na colheita o tempo todo.''

\textit{Advogado:} ``Você foi funcionária pública concursada?''

\textit{Silvia:} ``Sim, trabalhei na Prefeitura de 1992 a 1998. Depois pedi exoneração e fui para a rede estadual.''

\textbf{Descoberta:}
\begin{itemize}
    \item 6 anos de tempo rural (12-18 anos, 1983-1989)
    \item 6 anos de tempo público municipal (1992-1998) não averbados
\end{itemize}

\textbf{Documentação Obtida:}
\begin{itemize}
    \item Certidão de Casamento dos pais (profissão: ``Lavrador'')
    \item Histórico Escolar de escola rural
    \item CTC da Prefeitura (6 anos)
\end{itemize}

\textbf{Impacto:}
\begin{itemize}
    \item 27 anos (CNIS) + 6 anos (rural) + 6 anos (CTC) = \textbf{39 anos de tempo}
    \item Nova data de aposentadoria: \textbf{Hoje (2026)}
\end{itemize}

\textbf{Ganho financeiro:}
\begin{itemize}
    \item Aposentar 3 anos antes = 36 meses $\times$ R\$ 3.800
    \item \textbf{Ganho: R\$ 136.800}
\end{itemize}

\subsection{CASO 3: O Vendedor das Comissões Perdidas}

\textbf{Perfil:}
\begin{itemize}
    \item Nome: André, 62 anos, aposentado desde 2020
    \item Profissão: Vendedor
    \item RMI atual: R\$ 2.600,00
    \item CNIS: Salário registrado R\$ 2.200/mês
\end{itemize}

\textbf{Entrevista (Gatilho 7 - Pecúlios e Abonos):}

\textit{Advogado:} ``André, você recebia só salário fixo ou tinha comissão, gratificação?''

\textit{André:} ``Eu ganhava salário fixo de R\$ 2.200 + comissão sobre vendas. A comissão variava, mas todo mês eu recebia. Na média dava uns R\$ 2.500/mês de comissão.''

\textbf{Descoberta:}\\
O INSS considerou apenas o salário fixo (R\$ 2.200) no cálculo do PBC, ignorando as comissões habituais.

\textbf{Documentação Obtida:}\\
60 contracheques (2015-2020) mostrando:
\begin{itemize}
    \item Salário fixo: R\$ 2.200/mês
    \item Comissão média: R\$ 2.480/mês
    \item \textbf{Salário total médio:} R\$ 4.680/mês
\end{itemize}

\textbf{Impacto:}
\begin{itemize}
    \item PBC recalculado com salário total
    \item Nova RMI: R\$ 4.350 (vs R\$ 2.600 atual)
    \item \textbf{Aumento:} +R\$ 1.750/mês
\end{itemize}

\textbf{Ganho financeiro:}
\begin{itemize}
    \item Diferenças retroativas: 60 meses $\times$ R\$ 1.750 = \textbf{R\$ 105.000}
    \item Ganho vitalício (20 anos): R\$ 1.750 $\times$ 12 $\times$ 20 = \textbf{R\$ 420.000}
\end{itemize}

\section{Checklist Master do Pilar 1}

\begin{acaoImediata}
\textbf{CHECKLIST MASTER - PILAR 1: ENTREVISTA ESTRATÉGICA (2026)}

\textbf{ETAPA 0: VERIFICAÇO PROCURAÇO ELETRÔNICA}
\begin{itemize}
    \item[$\square$] Cliente tem conta gov.br nível prata/ouro?
    \item[$\square$] Procuração eletrônica cadastrada?
    \item[$\square$] Se NÃO $\rightarrow$ Orientar cadastro (enviar roteiro)
    \item[$\square$] Se SIM $\rightarrow$ Verificar acesso ao CNIS em tempo real
\end{itemize}

\textbf{ETAPA 1: PREPARAÇO}
\begin{itemize}
    \item[$\square$] CNIS atualizado acessado (via procuração ou documento)
    \item[$\square$] CTPS e documentos básicos solicitados
    \item[$\square$] Tempo reservado: 1-2 horas para entrevista
    \item[$\square$] Roteiro de perguntas-gatilho disponível
\end{itemize}

\textbf{ETAPA 2: CONDUÇO DA ENTREVISTA}
\begin{itemize}
    \item[$\square$] Cliente contou história cronológica (dos 12 anos até hoje)
    \item[$\square$] Pergunta-Gatilho 1 (Rural) aplicada
    \item[$\square$] Pergunta-Gatilho 2 (Militar/CTC) aplicada
    \item[$\square$] Pergunta-Gatilho 3 (Atividade Especial) aplicada
    \item[$\square$] Pergunta-Gatilho 4 (PEXT) aplicada
    \item[$\square$] Pergunta-Gatilho 5 (Concomitantes) aplicada
    \item[$\square$] Pergunta-Gatilho 6 (Afastamentos) aplicada
    \item[$\square$] Pergunta-Gatilho 7 (Pecúlios) aplicada
    \item[$\square$] Pergunta-Gatilho 8 (Salário-Maternidade) aplicada
\end{itemize}

\textbf{ETAPA 3: MAPEAMENTO DE ARMADILHAS}
\begin{itemize}
    \item[$\square$] Lista de armadilhas detectadas criada
    \item[$\square$] Cada armadilha tem justificativa (fala do cliente anotada)
    \item[$\square$] Priorização feita (usar Matriz de Priorização)
\end{itemize}

\textbf{ETAPA 4: DOCUMENTAÇO SOLICITADA}
\begin{itemize}
    \item[$\square$] Checklist de documentos entregue ao cliente
    \item[$\square$] Prazo definido: 7-15 dias
    \item[$\square$] Cliente entendeu a importância de cada documento
    \item[$\square$] Alertado sobre Tema 1124 (instruir PA completo)
\end{itemize}

\textbf{ETAPA 5: PRÓXIMOS PASSOS COMBINADOS}
\begin{itemize}
    \item[$\square$] Reagendamento marcado
    \item[$\square$] Cliente sabe que vamos para Pilar 2 (Diagnóstico)
    \item[$\square$] Expectativa de tempo total explicada (60-90 dias)
\end{itemize}

\textbf{ETAPA 6: REGISTRO INTERNO}
\begin{itemize}
    \item[$\square$] Anotações da entrevista digitadas/arquivadas
    \item[$\square$] Hipóteses de armadilhas registradas
    \item[$\square$] Lista de documentos aguardados registrada
    \item[$\square$] Status da procuração eletrônica anotado
\end{itemize}
\end{acaoImediata}

\section{Conclusão do Módulo 1}

Você acabou de dominar o \textbf{Pilar 1 do Sistema CJP: A Entrevista Estratégica}.

\textbf{O que você aprendeu:}
\begin{itemize}
    \item[\cmark] Por que o CNIS ``mente por omissão'' e não pode ser sua única fonte
    \item[\cmark] Como usar a \textbf{Procuração Eletrônica} para acesso direto ao Meu INSS
    \item[\cmark] Como conduzir uma entrevista investigativa (não um formulário)
    \item[\cmark] As 8 perguntas-gatilho que detectam as 8 Armadilhas Ocultas
    \item[\cmark] Qual documentação solicitar para cada tipo de segurado
    \item[\cmark] O impacto do \textbf{Tema 1124 STJ} na instrução do PA
    \item[\cmark] Como priorizar armadilhas por impacto vs dificuldade
    \item[\cmark] 3 casos reais onde a entrevista gerou ganhos de R\$ 100.000+
\end{itemize}

\textbf{O que muda na sua prática:}

\textbf{ANTES do Módulo 1:}\\
Recebia o CNIS $\rightarrow$ calculava $\rightarrow$ apresentava $\rightarrow$ honorários medianos

\textbf{DEPOIS do Módulo 1:}\\
Acessa CNIS via procuração $\rightarrow$ entrevista estrategicamente $\rightarrow$ descobre 3-5 erros $\rightarrow$ corrige $\rightarrow$ calcula com CNIS completo $\rightarrow$ apresenta ganho vitalício de R\$ 200.000+ $\rightarrow$ honorários premium

\begin{conceitoChave}
\textbf{A diferença?} Uma entrevista de 1 hora bem feita vale \textbf{R\$ 50.000 - R\$ 200.000} em ganho para o cliente.

E isso justifica seus honorários de \textbf{R\$ 1.500 - R\$ 3.000} por caso.
\end{conceitoChave}

\section{Próximo Passo}

Agora que você dominou a Entrevista (Pilar 1), é hora de ir para o \textbf{Pilar 2: O Diagnóstico Impecável}.

\textbf{No Módulo 2}, você aprenderá a \textbf{auditar o CNIS linha por linha}, decodificando os indicadores prioritários (nomenclatura atualizada pela Portaria DIRBEN/INSS n.\textsuperscript{o} 1.316/2025).

\textbf{No Módulo 3}, você aprenderá a \textbf{confirmar as 8 Armadilhas Ocultas} detectadas na entrevista (incluindo impacto do Tema 1090 STJ sobre EPI).

\section{Referências Legislativas (Módulo 1)}

BRASIL. \textbf{Lei n.\textsuperscript{o} 8.213}, de 24 de julho de 1991. Dispõe sobre os Planos de Benefícios da Previdência Social.

BRASIL. INSS. \textbf{Portaria Conjunta DIT/DIRBEN/INSS n.\textsuperscript{o} 10}, de 4 de novembro de 2025. Procuração eletrônica Meu INSS.

BRASIL. Superior Tribunal de Justiça. \textbf{Tema 1124}. Interesse de agir e termo inicial de benefícios. Outubro/2025.

BRASIL. Superior Tribunal de Justiça. \textbf{Tema 1090}. EPI e tempo especial. Abril/2025.

BRASIL. Superior Tribunal de Justiça. \textbf{Tema 1070}. Atividades concomitantes. Dezembro/2019.

BRASIL. \textbf{Lei n.\textsuperscript{o} 15.222}, de 28 de setembro de 2025. Prorrogação do salário-maternidade.

%% Continua na Parte 2A (Módulo 2)

%% ============================================================================
%% INFOGRÁFICO DO MÓDULO 1
%% ============================================================================
\clearpage
\backtotoc

\section*{\faImage\ Infográfico de Consolidação}

\begin{figure}[H]
    \centering
    \begin{tcolorbox}[colback=white, colframe=cjpAzulEscuro, title={\textbf{\faBookOpen\ Infográfico: Módulo 1 --- Entrevista Estratégica}}, fonttitle=\bfseries\color{white}, sharp corners=downhill, boxrule=2pt]
        \centering
        \includegraphics[width=0.95\textwidth, keepaspectratio]{modulo1}
    \end{tcolorbox}
    \caption{Resumo Visual do Módulo 1: Entrevista Estratégica}
    \label{fig:modulo1}
\end{figure}
