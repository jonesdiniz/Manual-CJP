% ============================================================================
% MÓDULO 4 - PARTE C: MODELOS RAC E FLUXOGRAMA
% ============================================================================

\section{Modelos Prontos (Templates RAC)}

A seguir, 3 modelos de RAC prontos para você adaptar aos seus casos. Copie, preencha os campos entre colchetes e protocole.

\subsection{Modelo 1: RAC Padrão (Múltiplos Acertos)}

\begin{acaoImediata}
\textbf{QUANDO USAR:} Cliente com múltiplos erros no CNIS (vínculos, tempo especial, contribuições).
\end{acaoImediata}

\begin{verbatim}
AO INSTITUTO NACIONAL DO SEGURO SOCIAL – INSS
AGÊNCIA: [Nome da Agência ou PROTOCOLO ONLINE]

REQUERIMENTO DE ACERTO DE CNIS
(Art. 19 do Decreto 3.048/99 | Portaria DIRBEN/INSS n\textsuperscript{o} 1.297/2025)

REQUERENTE: [Nome Completo do Segurado]
CPF: [XXX.XXX.XXX-XX]
NIT/PIS: [XXX.XXXXX.XX-X]
ENDEREÇO: [Rua, Número, Bairro, Cidade/UF, CEP]
TELEFONE: [(XX) XXXXX-XXXX]
E-MAIL: [email@example.com]

---

1. DO OBJETO

O(A) requerente vem, respeitosamente, solicitar a RETIFICAÇÃO DO CNIS 
(Cadastro Nacional de Informações Sociais) para corrigir os seguintes 
dados incorretos que impactam diretamente o cálculo de futuros 
benefícios previdenciários.

---

2. DOS FATOS E FUNDAMENTAÇÃO LEGAL

Conforme dispõe o Art. 29-A da Lei 8.213/91, o CNIS é a base de dados 
oficial para cálculo de benefícios. No entanto, o extrato atual do(a) 
requerente apresenta OMISSÕES E INCORREÇÕES que precisam ser sanadas.

Ressalta-se que, nos termos do Tema 1124 do STJ (outubro/2025), este 
requerimento está devidamente instruído com toda a documentação 
comprobatória disponível.

---

3. DOS ACERTOS SOLICITADOS

ACERTO 1: Inclusão de Vínculo Empregatício (PEXT)

- Empregador: [Razão Social da Empresa] (CNPJ [XX.XXX.XXX/XXXX-XX])
- Período: [DD/MM/AAAA] a [DD/MM/AAAA]
- Cargo: [Conforme CTPS]
- Situação Atual: Vínculo não consta no CNIS (indicador PEXT ausente)
- Documentos Comprobatórios:
  - Doc. 1: CTPS (páginas [X] a [Y])
  - Doc. 2: Extrato Analítico do FGTS (CEF)
  - Doc. 3: Termo de Rescisão de Contrato de Trabalho (TRCT)

ACERTO 2: Reconhecimento de Tempo Especial

- Empregador: [Razão Social da Empresa] (CNPJ [YY.YYY.YYY/YYYY-YY])
- Período: [DD/MM/AAAA] a [DD/MM/AAAA]
- Função: [Ex: Eletricista]
- Agente Nocivo: [Ex: Eletricidade acima de 115V]
- Enquadramento: Código [X.X.X] do Anexo 4 do Decreto 3.048/99
- Conversão Solicitada: [X] anos especiais $\times$ 1,40 = [Y] anos comuns
- Documentos Comprobatórios:
  - Doc. 4: PPP (Perfil Profissiográfico Previdenciário)
  - Doc. 5: LTCAT (Laudo Técnico das Condições Ambientais)

ACERTO 3: Complementação de Contribuições Abaixo do Mínimo

- Competências Afetadas: [MM/AAAA] a [MM/AAAA] (total de [X] meses)
- Situação Atual: Contribuições abaixo do salário mínimo pós EC 103/2019
- Valor Complementado: R$ [YYYY]
- Documentos Comprobatórios:
  - Doc. 6: GPS Complementares (Código 1929)
  - Doc. 7: Comprovantes de Pagamento

---

4. DO DIREITO

Base Legal:
- Lei 8.213/91, Art. 29-A (CNIS como base de cálculo)
- Decreto 3.048/99, Art. 19 (Correção de vínculos e salários)
- Portaria DIRBEN/INSS n\textsuperscript{o} 1.297/2025 (Procedimento RAC)
- Portaria DIRBEN/INSS n\textsuperscript{o} 1.316/2025 (Indicadores CNIS)

---

5. DOS PEDIDOS

Diante do exposto, REQUER:

a) A inclusão do vínculo empregatício com [Empresa X];
b) O reconhecimento do tempo especial com conversão;
c) A atualização das contribuições complementadas;
d) A emissão de novo extrato do CNIS corrigido.

---

6. RELAÇÃO DE DOCUMENTOS ANEXOS

- Doc. 1: CTPS (páginas [X] a [Y])
- Doc. 2: Extrato Analítico do FGTS
- Doc. 3: Termo de Rescisão de Contrato de Trabalho
- Doc. 4: PPP (Perfil Profissiográfico Previdenciário)
- Doc. 5: LTCAT (Laudo Técnico)
- Doc. 6: GPS Complementares ([X] unidades)
- Doc. 7: Comprovantes de Pagamento das GPS

---

[Cidade/UF], [DD] de [Mês] de [AAAA].

_________________________________________
[Nome Completo do Segurado]
CPF: [XXX.XXX.XXX-XX]

ou

_________________________________________
[Nome do Advogado(a)]
OAB/[UF] [XXXXX]
(Procuração anexa)
\end{verbatim}

\subsection{Modelo 2: RAC Tempo Especial (Focado)}

\begin{acaoImediata}
\textbf{QUANDO USAR:} Cliente com tempo especial não reconhecido (PPP disponível).
\end{acaoImediata}

\begin{verbatim}
AO INSTITUTO NACIONAL DO SEGURO SOCIAL – INSS

REQUERIMENTO DE ACERTO DE CNIS
RECONHECIMENTO DE TEMPO ESPECIAL

REQUERENTE: [Nome Completo]
CPF: [XXX.XXX.XXX-XX] | NIT: [XXX.XXXXX.XX-X]

---

1. DO OBJETO

Requer o RECONHECIMENTO DA NATUREZA ESPECIAL do tempo de serviço 
prestado na função de [Cargo/Função] junto à empresa [Razão Social], 
no período de [DD/MM/AAAA] a [DD/MM/AAAA], com a devida conversão 
de tempo especial para tempo comum.

---

2. DOS FATOS

O(A) requerente exerceu a função de [Cargo] exposto(a) de forma 
habitual e permanente ao agente nocivo [Nome do Agente - Ex: Ruído 
acima de 85 dB], conforme comprovado pelo PPP anexo.

Atualmente, o CNIS reconhece o período como TEMPO COMUM, o que 
reduz indevidamente o tempo total de contribuição.

---

3. DA FUNDAMENTAÇÃO LEGAL

3.1 Enquadramento do Agente Nocivo:
- Código [X.X.X] do Anexo 4 do Decreto 3.048/99

3.2 Base Legal:
- Lei 8.213/91, Art. 57 (Aposentadoria especial)
- Decreto 3.048/99, Art. 70 (Conversão de tempo especial)
- IN PRES/INSS 128/2022, Arts. 305-310 (PPP)

3.3 Jurisprudência:
- Tema 555 do STJ: Conversão permitida mesmo pós-EC 103/2019
- Tema 1090 do STJ: Ônus da prova sobre EPI favorece segurado

---

4. DO PEDIDO

REQUER a conversão do tempo especial de [X] anos para [Y] anos de 
tempo comum, aplicando-se o fator de conversão de 1,40 (homem) ou 
1,20 (mulher), conforme Art. 70 do Decreto 3.048/99.

---

5. DOCUMENTOS ANEXOS

- Doc. 1: PPP (Perfil Profissiográfico Previdenciário)
- Doc. 2: LTCAT (Laudo Técnico das Condições Ambientais)
- Doc. 3: CTPS (páginas do contrato de trabalho)

---

[Cidade/UF], [DD] de [Mês] de [AAAA].

_________________________________________
[Nome do Segurado ou Advogado(a)]
[CPF ou OAB]
\end{verbatim}

\subsection{Modelo 3: RAC Vínculo Rural}

\begin{acaoImediata}
\textbf{QUANDO USAR:} Cliente com tempo rural não reconhecido (economia familiar).
\end{acaoImediata}

\begin{verbatim}
AO INSTITUTO NACIONAL DO SEGURO SOCIAL – INSS

REQUERIMENTO DE ACERTO DE CNIS
RECONHECIMENTO DE TEMPO RURAL

REQUERENTE: [Nome Completo]
CPF: [XXX.XXX.XXX-XX] | NIT: [XXX.XXXXX.XX-X]

---

1. DO OBJETO

Requer o RECONHECIMENTO DE TEMPO DE SERVIÇO RURAL em regime de 
economia familiar, exercido no período de [DD/MM/AAAA] a [DD/MM/AAAA], 
na localidade de [Nome da Fazenda/Sítio], Município de [Nome], 
Estado de [UF].

---

2. DOS FATOS

O(A) requerente nasceu e cresceu em família de lavradores na zona 
rural do município de [Nome], tendo trabalhado desde a infância nas 
atividades agrícolas da propriedade familiar.

A família era conhecida na região como TRABALHADORES RURAIS, conforme 
comprovam os documentos anexos, incluindo Certidão de Casamento dos 
genitores (profissão: LAVRADOR), Histórico Escolar de escola rural 
e Declaração do Sindicato dos Trabalhadores Rurais.

O trabalho rural foi exercido em REGIME DE ECONOMIA FAMILIAR 
(Art. 11, VII, Lei 8.213/91), caracterizado pela:
- Atividade agrícola/pecuária indispensável à subsistência
- Participação ativa de todos os membros da família
- Ausência de empregados permanentes

---

3. DA FUNDAMENTAÇÃO LEGAL

3.1 Base Legal:
- Lei 8.213/91, Arts. 11 (VII) e 106
- Decreto 3.048/99, Arts. 9\textsuperscript{o} e 62
- Súmula 149 do STJ: "Início de prova material + prova testemunhal"

3.2 Do Início de Prova Material:
Os documentos anexos (Certidão de Casamento com profissão LAVRADOR, 
Histórico Escolar de escola rural, etc.) constituem início de prova 
material suficiente.

---

4. DAS PROVAS

4.1 Provas Materiais (Início de Prova):
- Doc. 1: Certidão de Casamento dos pais (profissão: LAVRADOR)
- Doc. 2: Certidão de Nascimento do(a) requerente (naturalidade rural)
- Doc. 3: Histórico Escolar da Escola Municipal [Nome] (zona rural)
- Doc. 4: Declaração do Sindicato dos Trabalhadores Rurais

4.2 Provas Testemunhais:
| Nome                    | CPF              | Relação  | Contato          |
|-------------------------|------------------|----------|------------------|
| [Nome Testemunha 1]     | [XXX.XXX.XXX-XX] | [Vizinho]| [(XX) XXXXX-XXXX]|
| [Nome Testemunha 2]     | [YYY.YYY.YYY-YY] | [Primo]  | [(XX) XXXXX-XXXX]|

---

5. DO PEDIDO

Diante do exposto, REQUER:

a) O reconhecimento do tempo rural no período de [DD/MM/AAAA] a 
   [DD/MM/AAAA];
b) A inclusão do período rural no CNIS;
c) Caso necessário, a designação de AUDIÊNCIA DE JUSTIFICAÇÃO 
   ADMINISTRATIVA para oitiva das testemunhas;
d) A emissão de novo extrato do CNIS corrigido.

---

6. RELAÇÃO DE DOCUMENTOS ANEXOS

- Doc. 1: Certidão de Casamento dos pais
- Doc. 2: Certidão de Nascimento do(a) requerente
- Doc. 3: Histórico Escolar (Escola Municipal [Nome])
- Doc. 4: Declaração do Sindicato dos Trabalhadores Rurais
- Doc. 5: Declarações escritas das testemunhas

---

[Cidade/UF], [DD] de [Mês] de [AAAA].

_________________________________________
[Nome do Segurado ou Advogado(a)]
[CPF ou OAB]
\end{verbatim}

\section{Fluxograma Completo: Do Diagnóstico ao CNIS Perfeito}

O processo completo do Pilar 4 segue este fluxo decisório:

\begin{estrategiaCJP}
\textbf{FLUXO DO PILAR 4: ACERTOS DE VÍNCULOS}

\textbf{ENTRADA:} Módulos 2 e 3 (Diagnóstico) $\rightarrow$ Erros identificados

$\downarrow$

\textbf{ETAPA 1: PRIORIZAR ACERTOS}\\
Ordenar por impacto financeiro (R\$ de ganho vitalício)

$\downarrow$

\textbf{ETAPA 2: COLETAR PROVAS}\\
Hierarquia: Nível 1 (oficiais) $>$ Nível 2 (patronais) $>$ Nível 3 (testemunhais)

$\downarrow$

\textbf{ETAPA 3: ORGANIZAR DOSSIÊ}\\
Pastas numeradas + PDFs legíveis + Índice de documentos

$\downarrow$

\textbf{ETAPA 4: ELABORAR RAC}\\
Usar templates desta seção (Modelos 1, 2 ou 3)

$\downarrow$

\textbf{ETAPA 5: PROTOCOLAR}\\
Via MEU INSS (online) ou Agência (presencial)

$\downarrow$

\textbf{ETAPA 6: ACOMPANHAR}\\
Prazo: 45-60 dias
\end{estrategiaCJP}

\begin{conceitoChave}
\textbf{DECISÕES POSSÍVEIS APÓS ANÁLISE:}

\textbf{EXIGÊNCIA?}\\
$\rightarrow$ SIM: Responder em 30 dias com documentos adicionais

\textbf{DEFERIDO?}\\
$\rightarrow$ SIM: Baixar novo CNIS atualizado $\rightarrow$ \textbf{CNIS PERFEITO!}

\textbf{INDEFERIDO?}\\
$\rightarrow$ SIM: Recurso Ordinário em 30 dias

\textbf{RECURSO DEFERIDO?}\\
$\rightarrow$ SIM: CNIS Perfeito $\rightarrow$ Pronto para Módulo 5 (Cálculos)\\
$\rightarrow$ NÃO: Considerar via judicial
\end{conceitoChave}

\begin{armadilha}
\textbf{ATENÇÃO: PRAZOS CRÍTICOS}

\begin{itemize}
    \item \textbf{Exigência:} Responder em até 30 dias (sob pena de arquivamento)
    \item \textbf{Recurso:} Interpor em até 30 dias do indeferimento
    \item \textbf{Revisão Judicial:} Até 10 anos após DIB (decadência)
\end{itemize}

Não perca prazos! Configure alertas no seu sistema de gestão.
\end{armadilha}

\begin{acaoImediata}
\textbf{RESULTADO FINAL DO PILAR 4:}

Ao concluir este módulo, você terá:

\begin{itemize}
    \item[\cmark] CNIS 100\% corrigido e atualizado
    \item[\cmark] Todos os vínculos reconhecidos
    \item[\cmark] Tempo especial convertido
    \item[\cmark] Contribuições complementadas
    \item[\cmark] Dossiê probatório organizado
\end{itemize}

\textbf{Próximo passo:} Módulo 5 (Cálculos) --- Calcular cenários com o CNIS perfeito.
\end{acaoImediata}

%% ============================================================================
%% INFOGRÁFICO DO MÓDULO 4
%% ============================================================================
\clearpage
\backtotoc

\section*{\faImage\ Infográfico de Consolidação}

\begin{figure}[H]
    \centering
    \begin{tcolorbox}[colback=white, colframe=cjpAzulEscuro, title={\textbf{\faBookOpen\ Infográfico: Módulo 4 --- Acertos e Documentação}}, fonttitle=\bfseries\color{white}, sharp corners=downhill, boxrule=2pt]
        \centering
        \includegraphics[width=0.95\textwidth, keepaspectratio]{modulo4}
    \end{tcolorbox}
    \caption{Resumo Visual do Módulo 4: Acertos e Documentação}
    \label{fig:modulo4}
\end{figure}
