%% Continuação do Módulo 2: Auditoria do CNIS

\section{Indicadores 5-7: Pendências Críticas (Continuação)}

\subsection{PEMP-CAD - Empresa Não Cadastrada}

\begin{acaoImediata}
Sem cadastro da empresa no CNPJ/CNIS, o vínculo NÃO é reconhecido automaticamente.

\textbf{Solução:} Prova documental ROBUSTA + verificação na Receita Federal.
\end{acaoImediata}

\textbf{O QUE É:}

PEMP-CAD significa ``Pendência de Empresa - Cadastro''. Aparece quando a empresa empregadora \textbf{não está cadastrada} no sistema CNPJ ou no CNIS do INSS.

\textbf{POR QUE OCORRE:}
\begin{enumerate}
    \item Empresa fechou antes de cumprir obrigações
    \item CNPJ baixado na Receita Federal
    \item Empresa ``fantasma''
    \item Erro de digitação no CNPJ
    \item Empresa mudou de razão social
\end{enumerate}

\textbf{IMPACTO NO BENEFÍCIO:}
\begin{itemize}
    \item[\xmark] Vínculo ignorado $\rightarrow$ Tempo não conta
    \item[\xmark] Salários desconsiderados
    \item[\xmark] Presunção de irregularidade
\end{itemize}

\textbf{ESTRATÉGIA DE RESOLUÇO:}

\textbf{PASSO 1:} Verificação Cadastral
\begin{enumerate}
    \item Acesse o site da Receita Federal
    \item Consulte o CNPJ informado na CTPS
    \item Veja a situação cadastral:
    \begin{itemize}
        \item[\cmark] ATIVA: Solicitar acerto de cadastro
        \item BAIXADA: Provar vínculo com documentos
        \item[\xmark] INEXISTENTE: Verificar CTPS
    \end{itemize}
\end{enumerate}

\begin{armadilha}
Se a empresa é ``fantasma'' (nunca existiu de fato), o caso fica MUITO difícil.

Precisará de:
\begin{itemize}
    \item Prova testemunhal FORTE (3+ pessoas)
    \item Documentos consistentes
    \item Explicação plausível
\end{itemize}
\end{armadilha}

\subsection{PVIN-REC-PROC-TRAB - Processo Trabalhista Pendente}

\begin{teseRevisional}
Processo trabalhista VENCIDO pode gerar:
\begin{itemize}
    \item Anos de tempo de contribuição
    \item Aumento de salários no PBC
    \item Base para revisão de benefício
\end{itemize}

Este indicador é uma OPORTUNIDADE de ouro, não um problema.
\end{teseRevisional}

\textbf{O QUE É:}

PVIN-REC-PROC-TRAB indica que há \textbf{processo trabalhista com sentença transitada em julgado} reconhecendo vínculo e/ou remunerações, mas o INSS \textbf{ainda não averbou}.

\textbf{OPORTUNIDADE DE OURO:}
\begin{itemize}
    \item[\cmark] Você JÁ tem a sentença $\rightarrow$ Não precisa provar novamente
    \item[\cmark] Valores reconhecidos judicialmente $\rightarrow$ Presunção absoluta
    \item[\cmark] Pode adicionar anos ao tempo
    \item[\cmark] Pode aumentar PBC
    \item[\cmark] Base para revisão
\end{itemize}

\textbf{Exemplo Prático:}

\textbf{Cliente:} Pedro, aposentado desde 2020 com RMI R\$ 3.100,00/mês\\
\textbf{Processo trabalhista (2022) reconheceu:}
\begin{itemize}
    \item 3 anos de vínculo não registrado (2005-2008)
    \item Salários de R\$ 3.500,00/mês (vs. R\$ 1.200,00 no CNIS)
\end{itemize}

\textbf{IMPACTO DA AVERBAÇO:}
\begin{itemize}
    \item Tempo adicional: +3 anos
    \item Nova RMI: R\$ 3.710,00/mês (+R\$ 610,00)
    \item Retroativo: R\$ 29.280,00
    \item Ganho vitalício: R\$ 146.400,00
\end{itemize}

\begin{estrategiaCJP}
\textbf{Conexão com Tema 1124 STJ:}

Se o vínculo foi reconhecido em processo trabalhista:
\begin{itemize}
    \item Anexe sentença transitada em julgado ao RAC
    \item Anexe planilha homologada
    \item Anexe certidão de trânsito
    \item Documentação completa = DIB na DER
\end{itemize}
\end{estrategiaCJP}

\textbf{Jurisprudência:}
\begin{itemize}
    \item STJ REsp 1.352.721 (Tema 558): Não é necessário recolhimento para averbar tempo trabalhista
    \item TRF4: Sentença trabalhista = prova plena
\end{itemize}

\subsection{PREC-FBR - Falta de Base de Recolhimento}

\begin{acaoImediata}
PREC-FBR significa que há contribuições PAGAS, mas sem BASE DE CÁLCULO registrada.

Resultado: Contribuições são IGNORADAS no cálculo do PBC e da RMI.
\end{acaoImediata}

\textbf{O QUE É:}

PREC-FBR significa ``Pendência de Recolhimento - Falta de Base de Recolhimento''. Aparece quando há contribuições pagas, mas o sistema \textbf{não registrou o valor base}.

\textbf{ESTRATÉGIA:}
\begin{enumerate}
    \item Solicite correção via Meu INSS
    \item Anexe GPS pagas + contracheques
    \item Demonstre cálculo: salário $\times$ alíquota = valor pago
    \item Prazo: 30-60 dias
\end{enumerate}

\section{GRUPO 2: Indicadores Informativos}

\subsection{IEAN - Exposição a Agente Nocivo}

\begin{teseRevisional}
IEAN = Atividade especial DETECTADA pelo sistema, mas NÃO CONVERTIDA.

Se o INSS identificou exposição a agente nocivo, você tem base para CONVERSÃO ou APOSENTADORIA ESPECIAL.

\textbf{Impacto potencial:} +25-40\% no tempo de contribuição.
\end{teseRevisional}

\textbf{O QUE É:}

IEAN significa ``Informativo de Exposição a Agente Nocivo''. Aparece quando o sistema \textbf{detectou} exposição a agentes nocivos, mas \textbf{não aplicou automaticamente} a conversão.

\textbf{OPORTUNIDADE:}
\begin{itemize}
    \item[\cmark] INSS já sabe que houve exposição
    \item[\cmark] Base para requerer conversão
    \item[\cmark] Base para aposentadoria especial
    \item[\cmark] Base para revisão
\end{itemize}

\begin{novidade}
\textbf{Tema 1090 STJ: EPI e Ônus da Prova (Abril/2025)}

\textbf{I.} Informação de EPI eficaz no PPP DESCARACTERIZA tempo especial, EXCETO para:
\begin{itemize}
    \item Ruído
    \item Agentes biológicos
    \item Agentes cancerígenos
    \item Periculosidade
\end{itemize}

\textbf{II.} ÔNUS DO SEGURADO comprovar ineficácia do EPI:
\begin{itemize}
    \item Inadequação ao risco
    \item Irregularidade no CA
    \item Descumprimento de normas
    \item Ausência de treinamento
\end{itemize}

\textbf{III.} Em caso de DÚVIDA $\rightarrow$ Conclusão FAVORÁVEL ao segurado

Ver detalhes no MÓDULO 3, Armadilha \#5.
\end{novidade}

\subsection{ICAR - Carência Insuficiente}

\textbf{Tabela de Carências 2026:}

\begin{table}[H]
\centering
\caption{Tabela de Carências Mínimas 2026}
\begin{tabular}{|l|c|}
\hline
\textbf{Benefício} & \textbf{Carência Mínima} \\
\hline
Aposentadoria por Idade & 180 contribuições (15 anos) \\
\hline
Aposentadoria por Tempo & 180 contribuições (15 anos) \\
\hline
Aposentadoria Especial & 180 contribuições (15 anos) \\
\hline
Auxílio por Incapacidade & 12 contribuições \\
\hline
Salário-Maternidade (CI/Fac) & 10 contribuições \\
\hline
\end{tabular}
\end{table}

\textbf{ESTRATÉGIA:}
\begin{itemize}
    \item Verificar contribuições não computadas
    \item Verificar períodos especiais (contam 1,2x ou 1,4x)
    \item Avaliar complementação
\end{itemize}

\subsection{IPREVI - Regime Próprio}

\textbf{O QUE É:}

IPREVI indica que há \textbf{tempo em regime próprio} (servidor público) que pode ser aproveitado via CTC.

\textbf{ESTRATÉGIA:}
\begin{itemize}
    \item Solicitar emissão de CTC ao órgão de origem
    \item Requerer averbação no RGPS
    \item Ver Módulo 3 (Armadilha \#4)
\end{itemize}

\subsection{IAUTVID - Revisão da Vida Toda}

\begin{armadilha}
\textbf{TESE DEFINITIVAMENTE ENCERRADA (Novembro/2025)}

A Revisão da Vida Toda foi CANCELADA pelo STF.

O segurado enquadrado na regra de transição do art. 3\textsuperscript{o} da Lei 9.876/1999 NÃO pode optar pela regra definitiva do art. 29, I e II, da Lei 8.213/1991.
\end{armadilha}

\textbf{Modulação de Efeitos (Marco: 05/04/2024):}

\begin{center}
\begin{tabular}{|p{7cm}|p{5cm}|}
\hline
\textbf{Situação} & \textbf{Consequência} \\
\hline
Valores recebidos até 05/04/2024 & Não precisam ser devolvidos \\
\hline
Honorários pendentes até 05/04/2024 & Não podem ser cobrados \\
\hline
Novas ações & Improcedentes \\
\hline
\end{tabular}
\end{center}

\textbf{Se o indicador IAUTVID aparecer no CNIS, IGNORE.}

\section{GRUPO 3: Acertos e Ajustes}

\subsection{ACNIS - Acerto Solicitado}

Indica que há RAC protocolado e em análise.

\textbf{ESTRATÉGIA:}
\begin{itemize}
    \item Anotar número do protocolo
    \item Monitorar quinzenalmente via Meu INSS ou 135
    \item Prazo médio: 45-90 dias úteis
\end{itemize}

\subsection{AHOM - Acerto Homologado}

Indica que o acerto foi aprovado e aguarda atualização no sistema.

\textbf{ESTRATÉGIA:}
\begin{itemize}
    \item Aguardar atualização do CNIS (15-30 dias)
    \item Confirmar valores corretos após atualização
\end{itemize}

\subsection{APEND - Acerto Pendente}

Indica que o INSS solicitou documentos adicionais.

\textbf{ESTRATÉGIA:}
\begin{itemize}
    \item Verificar exigência específica via Meu INSS
    \item Cumprir exigência no prazo (geralmente 30 dias)
    \item Risco: Se não cumprir, acerto é indeferido
\end{itemize}

\subsection{AINDEFERIDO - Acerto Indeferido}

\textbf{ESTRATÉGIA:}
\begin{itemize}
    \item Avaliar motivo do indeferimento
    \item Preparar recurso administrativo (30 dias)
    \item Se negar novamente: Via judicial
\end{itemize}

\begin{novidade}
\textbf{Conexão com Tema 1124 STJ:}

Se o acerto foi indeferido por falta de documentação:
\begin{itemize}
    \item NÃO ajuíze imediatamente
    \item Complemente a documentação
    \item Protocole NOVO RAC com docs completos
    \item DIB será na DER do segundo requerimento
\end{itemize}

Se ajuizar sem docs completos = DIB na citação (e não DER)
\end{novidade}

\section{Matriz de Priorização: Qual Indicador Resolver Primeiro}

\begin{estrategiaCJP}
\textbf{MATRIZ CJP DE PRIORIZAÇO}

\textbf{PRIORIDADE 1 (Resolver IMEDIATAMENTE):}
\begin{itemize}
    \item PDESFAZ-AJ-EC103 $\rightarrow$ Bloqueia concessão
    \item PSC-MEN-SM-EC103 $\rightarrow$ Período não conta
    \item PEMP-CAD $\rightarrow$ Vínculo ignorado
\end{itemize}

\textbf{PRIORIDADE 2 (Resolver ANTES de pedir benefício):}
\begin{itemize}
    \item PEXT $\rightarrow$ Tempo não conta
    \item PREC-MENOR-MIN $\rightarrow$ PBC reduzido
    \item PREC-FBR $\rightarrow$ Base zerada
\end{itemize}

\textbf{PRIORIDADE 3 (Oportunidade de aumento):}
\begin{itemize}
    \item PVIN-REC-PROC-TRAB $\rightarrow$ Pode adicionar tempo/salário
    \item IEAN $\rightarrow$ Pode converter tempo especial
\end{itemize}

\textbf{PRIORIDADE 4 (Monitorar):}
\begin{itemize}
    \item ACNIS, AHOM, APEND $\rightarrow$ Aguardar
    \item ICAR $\rightarrow$ Planejar complementação
    \item IPREVI $\rightarrow$ Avaliar CTC
\end{itemize}

\textbf{IGNORAR:} IAUTVID $\rightarrow$ Tese cancelada
\end{estrategiaCJP}

\section{Fluxograma: Da Identificação à Resolução Completa}

O processo completo de auditoria CNIS segue este fluxo decisório:

\begin{estrategiaCJP}
\textbf{FLUXOGRAMA DE AUDITORIA CNIS (Método CJP)}

\textbf{ENTRADA:} CNIS Completo Obtido

$\downarrow$

\textbf{ETAPA 1: VARREDURA INICIAL}
\begin{itemize}
    \item Marcar TODOS os indicadores presentes
    \item Contar total de vínculos cadastrados
    \item Anotar períodos com indicadores
\end{itemize}

$\downarrow$

\textbf{ETAPA 2: VERIFICAR IAUTVID}\\
É indicador de Revisão da Vida Toda?
\begin{itemize}
    \item SIM $\rightarrow$ IGNORAR (tese cancelada pelo STF)
    \item NÃO $\rightarrow$ Continuar classificação
\end{itemize}

$\downarrow$

\textbf{ETAPA 3: CLASSIFICAR INDICADORES}
\begin{itemize}
    \item Crítico (Grupo 1): Bloqueia/impacta concessão
    \item Informativo (Grupo 2): Oportunidade ou risco
    \item Acerto (Grupo 3): Correção em andamento
\end{itemize}

$\downarrow$

\textbf{ETAPA 4: APLICAR MATRIZ DE PRIORIZAÇO}
\begin{itemize}
    \item Prioridade 1: PDESFAZ, PSC-MEN-SM, PEMP-CAD
    \item Prioridade 2: PEXT, PREC-MENOR-MIN, PREC-FBR
    \item Prioridade 3: PVIN-REC-PROC-TRAB, IEAN
    \item Prioridade 4: ACNIS, AHOM, APEND, ICAR, IPREVI
\end{itemize}

$\downarrow$

\textbf{ETAPA 5: REUNIR DOCUMENTAÇO COMPLETA}\\
\textbf{(Tema 1124 STJ)}
\begin{itemize}
    \item CTPS, PPP, CTC, certificados
    \item GPS, contracheques, FGTS
    \item Sentença trabalhista (se houver)
\end{itemize}

$\downarrow$

\textbf{ETAPA 6: PROTOCOLAR RAC OU AÇO}
\begin{itemize}
    \item Via administrativa: RAC no Meu INSS
    \item Via judicial: Se bloqueio persistir
\end{itemize}
\end{estrategiaCJP}

\begin{conceitoChave}
\textbf{DECISÕES APÓS ANÁLISE DO RAC:}

\textbf{EXIGÊNCIA?}\\
$\rightarrow$ SIM: Responder em 30 dias com docs adicionais

\textbf{DEFERIDO?}\\
$\rightarrow$ SIM: Baixar novo CNIS atualizado\\
$\rightarrow$ \textbf{CNIS PERFEITO!} Pronto para Módulo 5

\textbf{INDEFERIDO?}\\
$\rightarrow$ Recurso Ordinário em 30 dias\\
$\rightarrow$ Se negar: Considerar via judicial
\end{conceitoChave}

\section{Casos Práticos: 3 CNIS Reais Auditados}

\subsection{CASO 1: O PEXT que Valia R\$ 200.000,00}

\textbf{Perfil:} José, 58 anos, busca Aposentadoria por Pontos

\textbf{CNIS inicial:}
\begin{itemize}
    \item Tempo: 32 anos e 4 meses
    \item Pontuação: 90 pontos (58 + 32)
    \item Necessário 2026: 103 pontos
    \item \textbf{Faltam: 13 pontos}
\end{itemize}

\textbf{Indicadores encontrados:}
\begin{enumerate}
    \item PEXT: 3 anos (2005-2008) na Empresa ABC
    \item PREC-MENOR-MIN: 18 meses com salário abaixo do mínimo
\end{enumerate}

\textbf{Resolução:}
\begin{enumerate}
    \item PEXT: RAC com CTPS + FGTS $\rightarrow$ DEFERIDO (45 dias)
    \item PREC-MENOR-MIN: Correção com contracheques $\rightarrow$ DEFERIDO (30 dias)
    \item Descoberta adicional: Soldador (2008-2012) $\rightarrow$ PPP $\rightarrow$ 4 anos $\times$ 1,4 = 5,6 anos
\end{enumerate}

\textbf{RESULTADO FINAL:}

\begin{center}
\begin{tabular}{|l|c|c|}
\hline
& \textbf{ANTES} & \textbf{DEPOIS} \\
\hline
Tempo & 32a 4m & 36a 7m \\
\hline
Pontos & 90 (faltam 13) & 95 (faltam 8) \\
\hline
Aposentadoria & 2031 & 2028 \\
\hline
RMI & --- & R\$ 4.800,00/mês \\
\hline
\textbf{Ganho antecipação} & & \textbf{R\$ 172.800,00} \\
\hline
\end{tabular}
\end{center}

\subsection{CASO 2: O Bloqueio PDESFAZ}

\textbf{Perfil:} Ana, 54 anos, enfermeira, busca Aposentadoria Especial

\textbf{Problema:} PDESFAZ-AJ-EC103 bloqueando análise

\textbf{Motivo:} Conversão de tempo especial + atividades concomitantes + múltiplas categorias

\textbf{Tentativa administrativa:}
\begin{itemize}
    \item 3 requerimentos online $\rightarrow$ TODOS travaram
    \item Atendimento presencial $\rightarrow$ Servidor não resolveu
    \item \textbf{Tempo perdido: 6 meses}
\end{itemize}

\textbf{Estratégia judicial:}
\begin{enumerate}
    \item Contratação de perito particular
    \item Cálculo completo da RMI preparado
    \item Tutela de urgência: Implantação de 80\%
\end{enumerate}

\textbf{Resultado:}
\begin{itemize}
    \item Tutela CONCEDIDA em 30 dias
    \item Benefício implantado provisoriamente
    \item Sentença final: PROCEDENTE (180 dias)
    \item \textbf{Economia: 12-18 meses vs. via administrativa}
\end{itemize}

\subsection{CASO 3: A Oportunidade Escondida (Processo Trabalhista)}

\textbf{Perfil:} Marco, 62 anos, aposentado desde 2020 com RMI R\$ 2.900,00/mês

\textbf{Indicador:} PVIN-REC-PROC-TRAB

\textbf{Descoberta:} Processo trabalhista de 2019 reconheceu 4 anos de vínculo clandestino (2010-2014) com salários de R\$ 4.500,00/mês

\textbf{Impacto:}
\begin{itemize}
    \item +4 anos de tempo
    \item PBC recalculado
    \item Nova RMI: R\$ 3.580,00/mês (+R\$ 680,00)
\end{itemize}

\textbf{Resultado:}
\begin{itemize}
    \item Retroativo (48 meses): R\$ 32.640,00
    \item Ganho vitalício (20 anos): R\$ 163.200,00
    \item \textbf{TOTAL: R\$ 195.840,00}
\end{itemize}

\section{Checklist Master do Diagnóstico Explícito}

\begin{acaoImediata}
\textbf{CHECKLIST MASTER --- PILAR 2: AUDITORIA CNIS (2026)}

\textbf{ETAPA 1: OBTENÇO DO CNIS}
\begin{itemize}
    \item[$\square$] CNIS completo obtido (via procuração ou documento)
    \item[$\square$] Verificado se é versão mais recente
    \item[$\square$] Todas as páginas conferidas
\end{itemize}

\textbf{ETAPA 2: VARREDURA DE INDICADORES}
\begin{itemize}
    \item[$\square$] Todos os indicadores marcados
    \item[$\square$] Classificados: Crítico/Informativo/Acerto
    \item[$\square$] Matriz de priorização aplicada
\end{itemize}

\textbf{ETAPA 3: ANÁLISE POR INDICADOR}
\begin{itemize}
    \item[$\square$] PEXT identificados $\rightarrow$ Documentação solicitada
    \item[$\square$] PREC-MENOR-MIN identificados $\rightarrow$ Contracheques solicitados
    \item[$\square$] PSC-MEN-SM-EC103 $\rightarrow$ Complementação calculada
    \item[$\square$] PDESFAZ-AJ-EC103 $\rightarrow$ Estratégia judicial avaliada
    \item[$\square$] PEMP-CAD $\rightarrow$ Situação cadastral verificada
    \item[$\square$] PVIN-REC-PROC-TRAB $\rightarrow$ Sentença obtida
    \item[$\square$] IEAN $\rightarrow$ PPP solicitado
\end{itemize}

\textbf{ETAPA 4: DOCUMENTAÇO (Tema 1124)}
\begin{itemize}
    \item[$\square$] Lista de documentos por indicador criada
    \item[$\square$] Prazo para documentação definido
    \item[$\square$] Cliente orientado sobre importância
\end{itemize}

\textbf{ETAPA 5: RESOLUÇO}
\begin{itemize}
    \item[$\square$] RACs protocolados para cada indicador
    \item[$\square$] Prazos de acompanhamento anotados
    \item[$\square$] Via judicial avaliada quando necessário
\end{itemize}

\textbf{ETAPA 6: PRÓXIMO PASSO}
\begin{itemize}
    \item[$\square$] Módulo 3 (Armadilhas Ocultas) agendado
    \item[$\square$] Cliente orientado sobre diagnóstico implícito
\end{itemize}
\end{acaoImediata}

\section{Conclusão do Módulo 2}

Você dominou o \textbf{Diagnóstico Explícito} do Pilar 2.

\textbf{O que você aprendeu:}
\begin{itemize}
    \item[\cmark] A diferença entre consultar e auditar o CNIS
    \item[\cmark] A anatomia dos indicadores (Prefixo + Tipo + Complemento)
    \item[\cmark] Os 15 indicadores prioritários e suas estratégias
    \item[\cmark] A matriz de priorização CJP
    \item[\cmark] 3 casos práticos com ganhos reais
\end{itemize}

\section{Próximo Passo}

No \textbf{Módulo 3}, você aprenderá o \textbf{Diagnóstico Implícito}: as 8 Armadilhas Ocultas que o CNIS NÃO mostra (e como encontrá-las).

%% Continua na Parte 3A (Módulo 3)

%% ============================================================================
%% INFOGRÁFICO DO MÓDULO 2
%% ============================================================================
\clearpage
\backtotoc

\section*{\faImage\ Infográfico de Consolidação}

\begin{figure}[H]
    \centering
    \begin{tcolorbox}[colback=white, colframe=cjpAzulEscuro, title={\textbf{\faBookOpen\ Infográfico: Módulo 2 --- Auditoria do CNIS}}, fonttitle=\bfseries\color{white}, sharp corners=downhill, boxrule=2pt]
        \centering
        \includegraphics[width=0.95\textwidth, keepaspectratio]{modulo2}
    \end{tcolorbox}
    \caption{Resumo Visual do Módulo 2: Auditoria do CNIS}
    \label{fig:modulo2}
\end{figure}
