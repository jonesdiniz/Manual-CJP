\clearpage
\chapter*{Módulo 1: A Entrevista Estratégica}
\addcontentsline{toc}{chapter}{Módulo 1: A Entrevista Estratégica}
\markboth{Módulo 1: A Entrevista Estratégica}{Módulo 1: A Entrevista Estratégica}
\setcounter{chapter}{1}

\begin{center}
{\Large\textit{Onde o Planejamento Começa: O Que o CNIS NÃO Conta}}\\[0.5cm]
\textbf{Sistema CJP | Pilar 1 de 5}
\end{center}

\begin{acaoImediata}
O CNIS é um ponto de partida, NUNCA a resposta final. Um planejamento baseado apenas no CNIS falhará em 90\% dos casos.

A entrevista estratégica é o seu ``raio-x'' para descobrir a vida oculta do cliente.
\end{acaoImediata}

\section{O Mito do CNIS Perfeito}

\subsection{A Mentira Que Todo Advogado Já Acreditou}

Você já viveu este momento?

Cliente entra no escritório. Você solicita o CNIS. Ele traz impresso. São 15 páginas de vínculos, salários, códigos de recolhimento. Você abre seu software de cálculos, digita os dados, aperta ``calcular''... e recebe uma resposta: \textbf{``Aposentadoria por Pontos em 18/05/2027 - RMI R\$ 3.200,00''}

Você mostra para o cliente. Ele sorri. Você fecha o atendimento. Honorários acertados. Caso resolvido.

\textbf{Mas será mesmo?}

E se eu te disser que, naquele CNIS de 15 páginas, existiam \textbf{5 anos de tempo rural da infância} que não constavam? E se aquele cliente trabalhou 3 anos como \textbf{soldador} (atividade especial) em uma empresa que faliu e nunca entregou o PPP? E se ele tem \textbf{vínculo CLT de 2 anos} em uma empresa que não recolheu, mas que está registrado na CTPS?

\textbf{O que muda?}

\begin{itemize}
    \item 5 anos rurais reconhecidos = \textbf{+5 anos de tempo}
    \item 3 anos especiais convertidos em comuns = \textbf{+1,2 anos de bônus}
    \item 2 anos PEXT reconhecidos = \textbf{+2 anos de tempo}
\end{itemize}

\textbf{Total adicional: +8,2 anos de tempo de contribuição}

\textbf{Novo cálculo:}
\begin{itemize}
    \item Data de aposentadoria: \textbf{Hoje} (não precisa esperar até 2027)
    \item RMI: \textbf{R\$ 4.350,00} (não R\$ 3.200,00)
    \item \textbf{Diferença vitalícia:} R\$ 276.000,00 (20 anos de aposentadoria)
\end{itemize}

\textbf{E você perdeu isso porque confiou no CNIS.}

\begin{armadilha}
O maior erro do advogado previdenciário é abrir o CNIS e ACREDITAR nele.

O CNIS não é um extrato de ``direitos''. É um extrato de ``dados brutos'' que o empregador (ou o sistema) enviou.
\end{armadilha}

\subsection{O Que o CNIS NÃO Mostra (E Por Que)}

O CNIS \textbf{mente por omissão}. Ele não foi projetado para ser completo. Ele registra apenas o que foi \textbf{formalmente informado ao INSS} via GFIP, CAGED ou outros sistemas.

\textbf{O CNIS NÃO mostra:}

\begin{enumerate}
    \item \textbf{Tempo rural trabalhado na infância/adolescência}\\
    Por quê? Porque trabalhadores rurais em regime de economia familiar não geram GFIP.
    
    \item \textbf{Tempo de serviço militar obrigatório}\\
    Por quê? Porque o tempo militar precisa ser \textbf{averbado} via CTC do Exército.
    
    \item \textbf{Tempo especial em empresa que faliu sem entregar PPP}\\
    Por quê? Porque o CNIS registra o vínculo, mas não a natureza da atividade.
    
    \item \textbf{Vínculos de empresa que não recolheu FGTS}\\
    Por quê? A empresa infringiu a lei. O INSS trata como ``PEXT''.
    
    \item \textbf{Atividades concomitantes que devem ser somadas}\\
    Por quê? O CNIS mostra vínculos separadamente. Cabe a você somar.
    
    \item \textbf{Períodos de afastamento que não devem entrar no PBC}\\
    Por quê? O CNIS lista o período completo. Cabe a você descontar.
    
    \item \textbf{Contribuições de MEI/facultativo com valores errados}\\
    Por quê? O CNIS registra o que foi declarado/pago, mesmo errado.
    
    \item \textbf{Pecúlios e gratificações que devem integrar o PBC}\\
    Por quê? O CNIS pode omitir verbas variáveis de natureza salarial.
\end{enumerate}

\textbf{Em resumo:} O CNIS é um \textbf{ponto de partida}, nunca a \textbf{resposta final}.

\section{A Revolução da Procuração Eletrônica}

\begin{novidade}
A Portaria Conjunta DIT/DIRBEN/INSS n.\textsuperscript{o} 10/2025 mudou COMPLETAMENTE a forma como você acessa informações do cliente. LEIA ESTA SEÇO COM ATENÇO MÁXIMA.
\end{novidade}

\subsection{O Fim da Dependência de Senhas do Cliente}

Até novembro de 2025, todo advogado previdenciário conhecia este diálogo constrangedor:

\textit{``Cliente, preciso da sua senha do gov.br para acessar seu CNIS...''}

\textit{``Não sei a senha. Meu sobrinho que fez pra mim.''}

\textbf{Isso acabou.}

A \textbf{Portaria Conjunta DIT/DIRBEN/INSS n.\textsuperscript{o} 10/2025}, publicada em 10/11/2025 e vigente desde \textbf{13/11/2025}, instituiu a \textbf{procuração eletrônica} na plataforma Meu INSS.

\begin{estrategiaCJP}
\textbf{IMPACTO PRÁTICO IMEDIATO}

\textbf{ANTES:} ``Cliente, preciso da sua senha do Meu INSS...''

\textbf{DEPOIS:} ``Cliente, você vai me autorizar pela procuração eletrônica e eu acesso direto, sem precisar da sua senha.''

\textbf{Resultado:}
\begin{itemize}
    \item[\cmark] Mais PROFISSIONALISMO (não pede senha de terceiro)
    \item[\cmark] Mais SEGURANÇA (não compartilha credenciais)
    \item[\cmark] Mais AGILIDADE (acesso imediato após autorização)
    \item[\cmark] Mais CONTROLE (cliente sabe exatamente o que autorizou)
\end{itemize}
\end{estrategiaCJP}

\subsection{Requisitos e Funcionamento}

\textbf{REQUISITOS OBRIGATÓRIOS:}

\begin{center}
\begin{tabularx}{\textwidth}{|l|X|l|}
\hline
\textbf{Parte} & \textbf{Requisito} & \textbf{Observação} \\
\hline
Cliente (representado) & Conta gov.br nível \textbf{PRATA} ou \textbf{OURO} & Cadastro pessoal \\
\hline
Advogado (representante) & Conta gov.br nível \textbf{PRATA} ou \textbf{OURO} & CPF como procurador \\
\hline
Cadastro & Feito pelo \textbf{CLIENTE} & Advog. NÃO cadastra \\
\hline
\end{tabularx}
\end{center}

\textbf{SERVIÇOS ACESSÍVEIS (Art. 7\textsuperscript{o} da Portaria):}
\begin{itemize}
    \item[\cmark] Consultas de documentos e serviços online
    \item[\cmark] Consultas de pedidos e benefícios
    \item[\cmark] CNIS COMPLETO (todas as versões)
    \item[\cmark] Extrato de pagamentos
    \item[\cmark] Carta de Concessão
    \item[\cmark] Histórico de benefícios
    \item[\cmark] Simulação de aposentadoria
\end{itemize}

\textbf{LIMITAÇÕES (O Que NÃO É Possível):}
\begin{itemize}
    \item[\xmark] NÃO permite PROTOCOLAR requerimentos (DER, RAC, etc.)
    \item[\xmark] NÃO permite interpor RECURSO ou MANIFESTAÇO
    \item[\xmark] NÃO permite alterar dados cadastrais
    \item[\xmark] Procuração é EXCLUSIVA para plataforma Meu INSS
    \item[\xmark] NÃO tem validade se impressa (apenas digital)
    \item[\xmark] NÃO substitui procuração ``ad judicia''
\end{itemize}

\subsection{Responsabilidades do Advogado (Art. 9\textsuperscript{o})}

\begin{armadilha}
Ao aceitar a procuração eletrônica, você ASSUME responsabilidade por:
\begin{itemize}
    \item GUARDA das informações obtidas
    \item CONFIDENCIALIDADE dos dados previdenciários
    \item USO EXCLUSIVO para fins autorizados pelo cliente
    \item EVENTUAL uso indevido ou compartilhamento
\end{itemize}

Violações podem configurar:
\begin{itemize}
    \item Infração disciplinar (OAB)
    \item Responsabilidade civil
    \item Eventual responsabilidade penal (LGPD)
\end{itemize}
\end{armadilha}

\subsection{Como Usar na Entrevista Estratégica}

\textbf{ANTES da Procuração Eletrônica:}
\begin{enumerate}
    \item Agendar consulta
    \item Pedir para cliente trazer CNIS impresso
    \item Cliente esquece ou traz desatualizado
    \item Reagendar ou trabalhar com dados incompletos
\end{enumerate}

\textbf{DEPOIS da Procuração Eletrônica (2026):}
\begin{enumerate}
    \item Na primeira consulta: orientar cliente a cadastrar procuração
    \item Advogado acessa Meu INSS \textbf{em tempo real} durante a entrevista
    \item CNIS sempre atualizado, sem dependência do cliente
    \item Diagnóstico completo na primeira reunião
\end{enumerate}

\begin{estrategiaCJP}
\textbf{IMPLEMENTAÇO IMEDIATA}

\begin{enumerate}
    \item Crie um ROTEIRO de orientação para o cliente cadastrar a procuração (envie por WhatsApp antes da consulta)
    \item Verifique se sua conta gov.br está nível PRATA ou OURO
    \item Adicione ao seu checklist de primeira consulta: ``Cliente cadastrou procuração eletrônica?''
    \item Demonstre ao cliente que você pode acessar o CNIS dele em tempo real --- isso gera CONFIANÇA IMEDIATA
\end{enumerate}
\end{estrategiaCJP}

\section{Por Que a Entrevista É o Pilar 1}

A maioria dos advogados trabalha nesta ordem:

\begin{enumerate}
    \item Receber CNIS
    \item Digitar no software
    \item Calcular
    \item Apresentar resultado
\end{enumerate}

\textbf{O problema:} Se o input (CNIS) está incompleto, o output (cálculo) será \textbf{matematicamente correto, mas estrategicamente errado}.

O Método CJP inverte essa lógica:

\begin{enumerate}
    \item \textbf{ENTREVISTA ESTRATÉGICA} (Pilar 1) --- Descobrir o que falta no CNIS
    \item \textbf{DIAGNÓSTICO COMPLETO} (Pilar 2) --- Auditar CNIS + descobertas
    \item \textbf{ACERTOS} (Pilar 3) --- Transformar descobertas em provas
    \item \textbf{CÁLCULOS} (Pilar 4) --- Calcular com CNIS corrigido
    \item \textbf{ENTREGA} (Pilar 5) --- Apresentar o melhor caminho
\end{enumerate}

\begin{conceitoChave}
\textbf{90\% dos erros são erros de omissão}, não de cálculo.

O INSS não erra a fórmula da RMI. Ele erra porque deixa de considerar tempo, salários ou períodos que \textbf{deveriam} estar lá.

O Pilar 1 (Entrevista) é sua ferramenta de investigação para encontrar TUDO o que falta no Pilar 2 (Diagnóstico).
\end{conceitoChave}

\section{O Roteiro Investigativo CJP: 8 Perguntas-Gatilho}

\begin{acaoImediata}
Não preencha um formulário. Conduza uma INVESTIGAÇO.

O objetivo deste roteiro é usar perguntas-gatilho para identificar proativamente as 8 Armadilhas Ocultas (detalhadas no Módulo 3).
\end{acaoImediata}

\subsection{Como Usar Este Roteiro}

\textbf{Passo 1:} Peça ao cliente para \textbf{contar a história da vida dele, desde os 12 anos de idade, em ordem cronológica}.

\textbf{Passo 2:} Use as 8 perguntas-gatilho como seu guia. Não leia como questionário. Deixe o cliente falar e \textbf{interrompa} quando ouvir palavras-chave.

\textbf{Passo 3:} Anote TUDO. Não filtre informações. Uma informação ``irrelevante'' agora pode ser crítica depois.

\textbf{Passo 4:} Com a procuração eletrônica, verifique em tempo real se o que o cliente diz ``bate'' com o CNIS.

\subsection{GATILHO 1: Vínculos Rurais (Armadilha 7)}

\textbf{Armadilha Detectada:}\\
Tempo rural trabalhado na infância/adolescência que não está no CNIS.

\textbf{Pergunta-Gatilho:}\\
\textit{``Seus pais moravam no sítio? Você ajudava na roça, mesmo que fosse `só nas férias' ou `só para ajudar'? Em que período?''}

\textbf{Palavras-Chave Para Ficar Alerta:}
\begin{itemize}
    \item ``Morava na fazenda/sítio/chácara''
    \item ``Ajudava meus pais/avós''
    \item ``Trabalhava na lavoura/colheita/plantio''
    \item ``Escola era longe, no interior''
    \item ``Fui para a cidade aos X anos''
\end{itemize}

\textbf{Por Que É Armadilha:}\\
Tempo rural de regime de economia familiar (antes dos 16 anos) \textbf{conta como tempo de contribuição}, mas nunca aparece no CNIS.

Muitos clientes dizem: \textit{``Mas eu não recebia salário, era só ajuda...''}

\textbf{Resposta:} Não importa. A lei considera como tempo de contribuição (Art. 11, VII, Lei 8.213/91).

\textbf{Documentos Para Solicitar:}
\begin{itemize}
    \item[$\square$] Certidão de Nascimento (ver local de nascimento)
    \item[$\square$] Certidão de Casamento dos pais (verificar profissão)
    \item[$\square$] Histórico Escolar (escola rural?)
    \item[$\square$] Declaração de sindicato rural
    \item[$\square$] Notas fiscais de venda de produtos rurais
    \item[$\square$] Testemunhas (irmãos, vizinhos, professores)
\end{itemize}

\textbf{Exemplo Prático:}

\textbf{Cliente:} Maria, 52 anos, professora\\
\textbf{CNIS:} 28 anos de contribuição\\
\textbf{Entrevista:} \textit{``Nasci em Registro-SP. Meus pais tinham um sítio de banana. Eu ajudava até ir para a faculdade aos 18 anos.''}\\
\textbf{Descoberta:} 6 anos de tempo rural (12-18 anos)\\
\textbf{Documentação:} Certidão de casamento dos pais (``Lavrador''), histórico escolar rural\\
\textbf{Impacto:} +6 anos de tempo = Aposentadoria antecipada em 2 anos

\subsection{GATILHO 2: Tempo Militar/CTC (Armadilha 4)}

\textbf{Armadilha Detectada:}\\
Tempo de serviço militar ou tempo de serviço público (CTC) não averbado.

\textbf{Perguntas-Gatilho:}\\
\textit{``Você serviu o exército/Tiro de Guerra? Em que ano?''}\\
\textit{``Você já foi funcionário público concursado?''}

\textbf{Palavras-Chave:}
\begin{itemize}
    \item ``Fiz o exército em [ano]''
    \item ``Servi 12 meses no quartel''
    \item ``Trabalhei na Prefeitura/Governo''
    \item ``Era estatutário, não CLT''
\end{itemize}

\textbf{Por Que É Armadilha:}
\begin{itemize}
    \item \textbf{Tempo militar:} Precisa averbar com Certificado de Reservista
    \item \textbf{Tempo público:} Precisa emitir CTC e averbar no RGPS
\end{itemize}

\textbf{Exemplo Prático:}

\textbf{Cliente:} João, 58 anos, metalúrgico\\
\textbf{CNIS:} 32 anos de contribuição\\
\textbf{Entrevista:} \textit{``Fiz o exército em 1986 (1 ano). Depois trabalhei na Prefeitura de 1987 a 1991, aí pedi demissão.''}\\
\textbf{Descoberta:} 1 ano militar + 4 anos públicos = 5 anos não averbados\\
\textbf{Impacto:} +5 anos = Aposentadoria por pontos \textbf{hoje}

\subsection{GATILHO 3: Atividade Especial (Armadilha 5)}

\textbf{Armadilha Detectada:}\\
Tempo especial (insalubre/perigoso) contado como tempo comum.

\textbf{Perguntas-Gatilho:}\\
\textit{``Você trabalhou com ruído alto, calor, produtos químicos, eletricidade, altura?''}\\
\textit{``Que função você exercia? Usava EPI? Tinha exposição constante?''}

\textbf{Palavras-Chave:}
\begin{itemize}
    \item ``Trabalhava com solda''
    \item ``Mexia com produtos químicos/tóxicos''
    \item ``Operador de máquina pesada/caldeira''
    \item ``Eletricista de alta tensão''
    \item ``Usava protetor auricular/máscara''
\end{itemize}

\begin{novidade}
\textbf{TEMA 1090 STJ (Abril/2025)}

O STJ fixou que a informação de EPI EFICAZ no PPP descaracteriza o tempo especial, EXCETO para:
\begin{itemize}
    \item Ruído (Tema 555 STF)
    \item Agentes biológicos
    \item Agentes cancerígenos
    \item Periculosidade
\end{itemize}

\textbf{ÔNUS DA PROVA:} Cabe ao SEGURADO provar ineficácia do EPI.

Ver detalhes no Módulo 3, Armadilha \#5.
\end{novidade}

\textbf{Documentos Para Solicitar:}
\begin{itemize}
    \item[$\square$] PPP (Perfil Profissiográfico Previdenciário)
    \item[$\square$] PPP Eletrônico (obrigatório a partir de 01/01/2023)
    \item[$\square$] LTCAT (Laudo Técnico das Condições Ambientais)
    \item[$\square$] CTPS (ver função registrada)
    \item[$\square$] Contracheques (ver cargo)
\end{itemize}

\textbf{Exemplo Prático:}

\textbf{Cliente:} Carlos, 54 anos, aposentado\\
\textbf{CNIS:} 30 anos, RMI R\$ 2.800\\
\textbf{Entrevista:} \textit{``Trabalhei 10 anos como soldador. Usava máscara, mas o ruído era alto o tempo todo.''}\\
\textbf{Descoberta:} 10 anos especiais não reconhecidos\\
\textbf{Impacto:} 10 anos $\times$ 1,4 = 14 anos. \textbf{+4 anos de bônus} = Revisão de R\$ 2.800 para R\$ 3.650\\
\textbf{Ganho vitalício:} R\$ 204.000

\subsection{GATILHO 4: Vínculos PEXT (Armadilha 1)}

\textbf{Armadilha Detectada:}\\
Vínculo na CTPS que aparece como ``PEXT'' (pendente) no CNIS.

\textbf{Pergunta-Gatilho:}\\
\textit{``Você tem algum emprego na CTPS que NÃO aparece no CNIS? Empresa que faliu?''}

\textbf{Palavras-Chave:}
\begin{itemize}
    \item ``A empresa faliu''
    \item ``Fechou sem me pagar direitos''
    \item ``Tenho anotação na carteira, mas não sei se o INSS sabe''
\end{itemize}

\textbf{Documentos:}
\begin{itemize}
    \item[$\square$] CTPS (qualificação + contrato + baixa)
    \item[$\square$] Contracheques (mínimo 3 por ano)
    \item[$\square$] TRCT
    \item[$\square$] Extratos FGTS
    \item[$\square$] Declarações IR
\end{itemize}

\subsection{GATILHO 5: Atividades Concomitantes (Armadilha 2)}

\textbf{Armadilha Detectada:}\\
Cliente trabalhou em dois empregos simultâneos, mas INSS somou apenas um salário.

\textbf{Pergunta-Gatilho:}\\
\textit{``Você já trabalhou em dois empregos ao mesmo tempo?''}

\textbf{Jurisprudência:} Tema 1070 STJ --- ambos os salários devem ser \textbf{somados integralmente} (válido para DIB entre Nov/1999 e Jun/2019).

\textbf{Exemplo Prático:}

\textbf{Cliente:} Pedro, 63 anos, aposentado em 2010\\
\textbf{CNIS:} Dois vínculos simultâneos (Empresa A: R\$ 1.200 | Empresa B: R\$ 800)\\
\textbf{Entrevista:} \textit{``Eu trabalhava nas duas. O INSS considerou só R\$ 1.200.''}\\
\textbf{Descoberta:} Deveria ser R\$ 2.000/mês, não R\$ 1.200\\
\textbf{Impacto:} Revisão do PBC = +R\$ 420/mês na RMI\\
\textbf{Ganho:} R\$ 100.800

\subsection{GATILHO 6: Afastamentos (Armadilha 3)}

\textbf{Armadilha Detectada:}\\
Períodos de auxílio-doença foram \textbf{incluídos no PBC} (quando deveriam ser descontados).

\textbf{Pergunta-Gatilho:}\\
\textit{``Ficou afastado por doença ou acidente? Recebeu auxílio-doença? Por quanto tempo?''}

\textbf{Por Que É Armadilha:}\\
Períodos de B31 \textbf{não devem entrar no PBC} porque não houve salário de contribuição.

\subsection{GATILHO 7: Pecúlios e Abonos (Armadilha 8)}

\textbf{Armadilha Detectada:}\\
Gratificações e abonos de natureza salarial não foram incluídos no PBC.

\textbf{Pergunta-Gatilho:}\\
\textit{``Além do salário fixo, você recebia gratificação, comissão, abono?''}

\textbf{Documentos:}
\begin{itemize}
    \item[$\square$] Contracheques completos (todos os meses)
    \item[$\square$] Extrato analítico FGTS
    \item[$\square$] Recibos de comissão/abono
\end{itemize}

\subsection{GATILHO 8: Salário-Maternidade (Armadilha 6)}

\textbf{Armadilha Detectada:}\\
Salário-maternidade (B32) foi incluído no PBC, reduzindo a média.

\textbf{Pergunta-Gatilho (mulheres):}\\
\textit{``Teve filhos durante o período em que trabalhou? Tirou licença-maternidade?''}

\textbf{Por Que É Armadilha:}\\
B32 é benefício, não salário. \textbf{Não deve entrar no PBC}.

\begin{novidade}
\textbf{Lei 15.222/2025 --- Prorrogação por Internação}

Desde 29/09/2025, o salário-maternidade pode ser PRORROGADO se o recém-nascido ficar internado. O período de internação não conta nos 120 dias.

\textbf{Perguntar:} ``Seu bebê ficou internado após o nascimento?''
\end{novidade}

\section{Checklist de Documentação por Tipo de Segurado}

\begin{acaoImediata}
Solicite estes documentos NA PRIMEIRA CONSULTA, mesmo que o cliente diga que ``não tem''.

Documentação completa = diagnóstico completo.\\
Documentação incompleta = oportunidades perdidas.

Com a Procuração Eletrônica, você já tem acesso ao CNIS em tempo real!
\end{acaoImediata}

\subsection{Para TODOS os Segurados}

\begin{itemize}
    \item[$\square$] RG e CPF
    \item[$\square$] Comprovante de residência (atualizado)
    \item[$\square$] CNIS atualizado (máximo 30 dias) OU Procuração Eletrônica
    \item[$\square$] CTPS (TODAS as páginas)
    \item[$\square$] Certidão de Nascimento
    \item[$\square$] Certidão de Casamento (se casado)
\end{itemize}

\subsection{Para Segurados CLT}

\begin{itemize}
    \item[$\square$] CTPS completa (todos os vínculos)
    \item[$\square$] Contracheques (mínimo 3 por ano)
    \item[$\square$] TRCT de todos os empregos
    \item[$\square$] Extrato Analítico do FGTS (completo)
    \item[$\square$] Declarações de IR (últimos 5 anos)
    \item[$\square$] PPP de TODOS os empregadores
    \item[$\square$] Certificado de Reservista
\end{itemize}

\subsection{Para Segurados com Tempo Rural}

\begin{itemize}
    \item[$\square$] Certidão de Nascimento (local: zona rural?)
    \item[$\square$] Certidão de Casamento dos pais (profissão: lavrador?)
    \item[$\square$] Histórico Escolar (escola rural?)
    \item[$\square$] Declaração de sindicato rural
    \item[$\square$] Documentos de posse/propriedade de terra
    \item[$\square$] Lista de testemunhas
\end{itemize}

\subsection{Para Ex-Servidores Públicos}

\begin{itemize}
    \item[$\square$] CTC (Certidão de Tempo de Contribuição)
    \item[$\square$] Portaria de nomeação
    \item[$\square$] Portaria de exoneração
    \item[$\square$] Contracheques do período público
\end{itemize}

\subsection{Para Autônomos/MEI}

\begin{itemize}
    \item[$\square$] Carnês de GPS (todos os anos)
    \item[$\square$] DAS (Documento de Arrecadação do Simples) --- se MEI
    \item[$\square$] Declarações de IR
    \item[$\square$] Contratos de prestação de serviços
    \item[$\square$] Recibos de pagamento
\end{itemize}
