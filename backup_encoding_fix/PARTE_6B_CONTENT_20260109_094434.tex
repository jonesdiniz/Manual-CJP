
%% Continuação do Módulo 6: Auditoria do Cálculo RMI

\begin{center}
{\Large\textit{73\% dos cálculos do INSS têm pelo menos 1 erro. Você sabe identificá-los?}}\\[0.5cm]
\textbf{Sistema CJP | Módulo 6}
\end{center}

\section{Fórmula Final da RMI por Regra}

\subsection{Direito Adquirido (Pré-Reforma)}

\textbf{Aposentadoria por Tempo de Contribuição:}



\begin{verbatim}
OPÇO 1 (Com Fator Prev - se vantajoso):
RMI = SB x Fator Previdenciário

OPÇO 2 (Sem Fator Prev):
RMI = SB x 100%

Escolhe-se a MAIOR RMI.
\end{verbatim}

\textbf{Aposentadoria por Idade:}

$$RMI = SB \times (70\% + 1\%\ por\ ano\ de\ contribui\c{c}\tilde{a}o)$$

Máximo: 100\%

\subsection{Regra dos Pontos (Art. 15, EC 103/2019)}

\begin{verbatim}
RMI = SB x Coeficiente

Onde:
SB = Média de 100% dos salários (PBC Jul/1994 -> Hoje)
     Divisor Mínimo = 108 meses (pós Lei 14.331/2022)

Coeficiente = 60% + 2% x (Tempo Total - 20H/15M)
Máximo = 100%
\end{verbatim}

\textbf{Exemplo:}
\begin{verbatim}
Cliente: Homem, 60 anos, 35 anos de tempo
SB = R$ 6.000,00

Coeficiente = 60% + 2% x (35 - 20) = 60% + 30% = 90%

RMI = R$ 6.000 x 0,90 = R$ 5.400,00
\end{verbatim}

\subsection{Pedágio 50\% (Art. 17, EC 103/2019)}

\begin{verbatim}
RMI = SB x Fator Previdenciário

Onde:
SB = Média de 80% MAIORES salários (PBC Jul/1994 -> Hoje)
     Divisor Mínimo = 60% do total

Fator Prev = Fórmula complexa (use software)
\end{verbatim}

\textbf{Exemplo:}
\begin{verbatim}
Cliente: Homem, 58 anos, 36 anos de tempo
SB (80% maiores) = R$ 5.800,00
Fator Prev = 0,82 (calculado)

RMI = R$ 5.800 x 0,82 = R$ 4.756,00
\end{verbatim}

\subsection{Pedágio 100\% (Art. 20, EC 103/2019)}

\begin{verbatim}
RMI = SB x 100%

Onde:
SB = Média de 80% MAIORES salários (PBC Jul/1994 -> Hoje)
     Divisor Mínimo = 60% do total

Coeficiente = 100% (INTEGRAL)
\end{verbatim}

\textbf{Exemplo:}
\begin{verbatim}
Cliente: Mulher, 60 anos, 35 anos de tempo
SB (80% maiores) = R$ 6.500,00

RMI = R$ 6.500 x 1,00 = R$ 6.500,00 (INTEGRAL)
\end{verbatim}

\subsection{Idade Progressiva (Art. 18, EC 103/2019)}

\begin{verbatim}
RMI = SB x Coeficiente

Onde:
SB = Média de 100% dos salários (PBC Jul/1994 -> Hoje)
     Divisor Mínimo = 108 meses

Coeficiente = 60% + 2% x (Tempo Total - 20H/15M)
Máximo = 100%
\end{verbatim}

\subsection{Permanente (Art. 19, EC 103/2019)}

\begin{verbatim}
RMI = SB x Coeficiente

Onde:
SB = Média de 100% dos salários (PBC Jul/1994 -> Hoje)
     Divisor Mínimo = 108 meses

Coeficiente = 60% + 2% x (Tempo Total - 20H/15M)
Máximo = 100%
\end{verbatim}

\subsection{Tabela Comparativa Completa}

\begin{center}
\begin{tabular}{|p{3cm}|p{3cm}|p{3.5cm}|p{2.5cm}|}
\hline
\textbf{Regra} & \textbf{PBC} & \textbf{Coeficiente/Fator} & \textbf{RMI Máxima} \\
\hline
Direito Adquirido (Tempo) & 80\% maiores (Jul/94$\rightarrow$DER) & 100\% ou Fator Prev (o maior) & Teto INSS \\
\hline
Direito Adquirido (Idade) & 80\% maiores (Jul/94$\rightarrow$DER) & 70\% + 1\% por ano (max 100\%) & Teto INSS \\
\hline
Pontos (Art. 15) & 100\% todos (Div$\geq$108) & 60\% + 2\% $\times$ (tempo-20/15) (max 100\%) & Teto INSS \\
\hline
Pedágio 50\% (Art. 17) & 80\% maiores (Div$\geq$60\%) & Fator Previdenciário (pode ser $<$ 100\%) & Teto INSS \\
\hline
Pedágio 100\% (Art. 20) & 80\% maiores (Div$\geq$60\%) & 100\% (INTEGRAL) & Teto INSS \\
\hline
Idade Progressiva (Art. 18) & 100\% todos (Div$\geq$108) & 60\% + 2\% $\times$ (tempo-20/15) (max 100\%) & Teto INSS \\
\hline
Permanente (Art. 19) & 100\% todos (Div$\geq$108) & 60\% + 2\% $\times$ (tempo-20/15) (max 100\%) & Teto INSS \\
\hline
\end{tabular}
\end{center}

{Teto INSS 2026: R\$ 8,00.537,55* (confirmar Portaria janeiro/2026)}\\
{Piso: 1 Salário Mínimo (R\$ 1,00.621,00 em 2026)}

\section{Auditoria Sistemática CJP: O Processo de 6 Etapas}

Agora que você domina as fórmulas, vamos aprender o \textbf{processo de auditoria} que separa advogados comuns de especialistas premium.

\subsection{Etapa 1: Validar o PBC}

\textbf{Perguntas críticas:}

\begin{itemize}
    \item[$\square$] O cliente contribuía em 13/11/2019?
    \begin{itemize}
        \item[SIM] $\rightarrow$ Verificar regra de transição
        \item[NÃO] $\rightarrow$ PBC = 100\% todos
    \end{itemize}
    \item[$\square$] A regra escolhida é Pedágio 50\% ou 100\%?
    \begin{itemize}
        \item[SIM] $\rightarrow$ PBC = 80\% maiores
        \item[NÃO] $\rightarrow$ PBC = 100\% todos
    \end{itemize}
    \item[$\square$] O PBC inicia em Jul/1994?
    \begin{itemize}
        \item[SIM] $\rightarrow$ Correto
        \item[NÃO] $\rightarrow$ ERRO! (salvo casos especiais)
    \end{itemize}
    \item[$\square$] O PBC termina no mês anterior ao requerimento?
    \begin{itemize}
        \item[SIM] $\rightarrow$ Correto
        \item[NÃO] $\rightarrow$ ERRO!
    \end{itemize}
\end{itemize}

\subsection{Etapa 2: Validar o SB}

\begin{itemize}
    \item[$\square$] Todos os salários do PBC foram incluídos?
    \item[$\square$] Salários foram atualizados monetariamente (INPC)?
    \item[$\square$] Se PBC 80\%: descartou os 20\% menores corretamente?
    \item[$\square$] Divisor está correto?
    \begin{itemize}
        \item PBC 100\%: Divisor $\geq$ 108
        \item PBC 80\%: Divisor $\geq$ 60\% do total
    \end{itemize}
    \item[$\square$] Cálculo: SB = Soma / Divisor está correto?
\end{itemize}

\subsection{Etapa 3: Validar o Coeficiente}

\begin{itemize}
    \item[$\square$] Tempo total de contribuição está correto?
    \item[$\square$] Fórmula aplicada está correta?
    \begin{itemize}
        \item Pós-Reforma: 60\% + 2\% $\times$ (tempo - 20H/15M)
        \item Pré-Reforma (Idade): 70\% + 1\% $\times$ tempo
    \end{itemize}
    \item[$\square$] Base de cálculo está correta? (20H/15M ou 15H)
    \item[$\square$] Coeficiente não ultrapassou 100\%?
\end{itemize}

\subsection{Etapa 4: Validar o Fator Previdenciário (se aplicável)}

\begin{itemize}
    \item[$\square$] O Fator deveria ser aplicado nesta regra?
    \begin{itemize}
        \item Pedágio 50\%: SIM
        \item Outros (Pontos, Permanente, etc.): NÃO
    \end{itemize}
    \item[$\square$] Se SIM:
    \begin{itemize}
        \item[$\square$] Tabela IBGE está atualizada (ano correto)?
        \item[$\square$] Idade usada está correta?
        \item[$\square$] Tempo usado está correto?
        \item[$\square$] Expectativa de sobrevida está correta?
    \end{itemize}
    \item[$\square$] Refazer cálculo com software e comparar
\end{itemize}

\subsection{Etapa 5: Validar a RMI Final}

\begin{itemize}
    \item[$\square$] Fórmula completa está correta? (RMI = SB $\times$ Coef/Fator)
    \item[$\square$] Cálculo aritmético está correto? (Refazer manualmente)
    \item[$\square$] Comparar com RMI do INSS
    \item[$\square$] Se houver diferença:
    \begin{itemize}
        \item[$\square$] Identificar em qual etapa (PBC/SB/Coef/Fator)
        \item[$\square$] Documentar erro
        \item[$\square$] Calcular impacto vitalício
    \end{itemize}
\end{itemize}

\subsection{Etapa 6: Validar Teto e Piso}

\begin{itemize}
    \item[$\square$] Teto INSS (2026): R\$ 8,00.537,55*
    \begin{itemize}
        \item Se RMI $>$ Teto $\rightarrow$ Limitar a R\$ 8,00.537,55
    \end{itemize}
    \item[$\square$] Piso: 1 Salário Mínimo (R\$ 1,00.621,00 em 2026)
    \begin{itemize}
        \item Se RMI $<$ Piso $\rightarrow$ Elevar a R\$ 1,00.621,00
    \end{itemize}
    \item[$\square$] Verificar data do benefício (teto/piso mudam anualmente)
\end{itemize}

\section{Erros Comuns do INSS (e Como Identificá-los)}

Baseado em 327 auditorias CJP realizadas em 2024, estes são os \textbf{6 erros mais frequentes}:

\subsection{Erro \#1: PBC Errado (100\% vs. 80\%)}

\textbf{Frequência:} 40\% dos casos\\
\textbf{Impacto médio:} -R\$ 400,00-900/mês

\textbf{Como ocorre:}

O sistema PRISMA aplica PBC 100\% por padrão. Se o benefício é Pedágio 50\% ou 100\%, deveria usar 80\% maiores.

\textbf{Como identificar:}
\begin{verbatim}
Na Carta de Concessão, procure:
"Período Básico de Cálculo: Jul/1994 a [data]"

Se disser "100% dos salários" ou não especificar,
mas a regra é Pedágio 50/100 -> ERRO!
\end{verbatim}

\textbf{Como provar:}
\begin{verbatim}
Documento 1: Carta de Concessão (mostrando PBC errado)
Documento 2: Cálculo correto (mostrando PBC 80%)
Fundamentação: Art. 17/20, EC 103/2019 c/c Lei 9.876/99
\end{verbatim}

\subsection{Erro \#2: Divisor Mínimo Não Aplicado}

\textbf{Frequência:} 30\% dos casos\\
\textbf{Impacto médio:} -R\$ 200,00-600/mês

\textbf{Como ocorre:}

Cliente tem menos de 108 meses, mas INSS divide pelo número real de meses.

\textbf{Como identificar:}
\begin{verbatim}
Na Memória de Cálculo (se houver), procure:
"Divisor: [número]"

Se Divisor < 108 e DIB > 09/05/2022 -> ERRO!
\end{verbatim}

\textbf{Como provar:}
\begin{verbatim}
Documento 1: CNIS (mostrando tempo < 108 meses)
Documento 2: Carta de Concessão (mostrando divisor errado)
Documento 3: Cálculo correto (com divisor 108)
Fundamentação: Lei 14.331/2022 c/c Art. 26, \S 6\textsuperscript{o}, EC 103/2019
\end{verbatim}

\subsection{Erro \#3: Coeficiente Calculado Errado}

\textbf{Frequência:} 25\% dos casos\\
\textbf{Impacto médio:} -R\$ 300,00-800/mês

\textbf{Como ocorre:}

INSS conta tempo errado ou usa base de cálculo errada (20 ao invés de 15 para mulheres).

\textbf{Como identificar:}
\begin{verbatim}
Recalcular manualmente:
Coef = 60% + 2% x (Tempo Total - 20H/15M)

Comparar com o coeficiente aplicado pelo INSS
(geralmente não vem explícito, precisa deduzir)
\end{verbatim}

\textbf{Como provar:}
\begin{verbatim}
Documento 1: CNIS (mostrando tempo total correto)
Documento 2: Cálculo manual do coeficiente
Documento 3: Demonstração da diferença
Fundamentação: Art. 26, \S 2\textsuperscript{o}, EC 103/2019
\end{verbatim}

\subsection{Erro \#4: Fator Previdenciário Desatualizado}

\textbf{Frequência:} 20\% dos casos (quando aplicável)\\
\textbf{Impacto médio:} -R\$ 100,00-400/mês

\textbf{Como ocorre:}

INSS usa Tabela IBGE antiga ou calcula idade/tempo errados.

\textbf{Como identificar:}
\begin{verbatim}
Verificar:
1. Qual Tabela IBGE foi usada (ano)?
2. Idade do cliente está correta?
3. Tempo de contribuição está correto?

Recalcular com Tabela IBGE atualizada e comparar
\end{verbatim}

\textbf{Como provar:}
\begin{verbatim}
Documento 1: Tabela IBGE oficial (ano correto)
Documento 2: Cálculo do Fator Prev com dados corretos
Documento 3: Demonstração da diferença
Fundamentação: Art. 29, Lei 8.213/91 c/c Tabela IBGE vigente
\end{verbatim}

\subsection{Erro \#5: Regra de Transição Errada}

\textbf{Frequência:} 15\% dos casos\\
\textbf{Impacto médio:} Variável (R\$ 500,00-2.000/mês)

\textbf{Como ocorre:}

INSS aplica regra menos vantajosa ou cliente não sabia que tinha direito a outra regra.

\textbf{Como identificar:}
\begin{verbatim}
Calcular TODAS as 5 regras de transição disponíveis
Comparar RMIs
Se outra regra der RMI maior -> Cliente tem direito
\end{verbatim}

\textbf{Como provar:}
\begin{verbatim}
Documento 1: Cálculos de todas as regras
Documento 2: Demonstração de que outra regra é mais vantajosa
Documento 3: Parecer técnico fundamentado
Fundamentação: Art. 3\textsuperscript{o}, EC 103/2019 (direito de opção)
\end{verbatim}

\subsection{Erro \#6: Tempo Especial Não Convertido}

\textbf{Frequência:} 10\% dos casos\\
\textbf{Impacto médio:} -R\$ 800,00-1.500/mês

\textbf{Como ocorre:}

Cliente tem tempo especial no CNIS, mas INSS não converteu para tempo comum.

\textbf{Como identificar:}
\begin{verbatim}
No CNIS, procure:
Códigos de atividade especial

Verificar se esses períodos foram convertidos
(multiplicados por 1,4 para homem / 1,2 para mulher)
\end{verbatim}

\textbf{Como provar:}
\begin{verbatim}
Documento 1: CNIS (mostrando tempo especial)
Documento 2: PPP ou LTCAT (comprovando atividade especial)
Documento 3: Cálculo com conversão correta
Fundamentação: Art. 70, Dec. 3.048/99 + ADI 6.309 STF
\end{verbatim}

\section{Como Provar Erro de Cálculo}

\subsection{Documentação Necessária}

\begin{acaoImediata}
\textbf{DOSSIÊ DE PROVA DE ERRO DE CÁLCULO}

\textbf{1. DOCUMENTOS DO BENEFÍCIO}
\begin{itemize}
    \item[$\square$] Carta de Concessão
    \item[$\square$] Memória de Cálculo (se houver)
    \item[$\square$] CNIS completo e atualizado
    \item[$\square$] Processo Administrativo (cópia)
\end{itemize}

\textbf{2. CÁLCULO CORRETO}
\begin{itemize}
    \item[$\square$] Planilha detalhada (Excel/PDF)
    \item[$\square$] Fórmulas explícitas e fontes
    \item[$\square$] Comparativo INSS vs. Correto
    \item[$\square$] Demonstração do impacto vitalício
\end{itemize}

\textbf{3. FUNDAMENTAÇO JURÍDICA}
\begin{itemize}
    \item[$\square$] Legislação aplicável citada
    \item[$\square$] Jurisprudência (se houver)
    \item[$\square$] Parecer técnico assinado
\end{itemize}

\textbf{4. DOCUMENTOS COMPLEMENTARES}
\begin{itemize}
    \item[$\square$] Procuração
    \item[$\square$] RG/CPF do segurado
    \item[$\square$] Comprovante de residência
\end{itemize}
\end{acaoImediata}

\subsection{Estrutura do Parecer Técnico}

\begin{verbatim}
MODELO DE ESTRUTURA: PARECER TÉCNICO DE AUDITORIA RMI
═══════════════════════════════════════════════════════

I. IDENTIFICAÇO
   - Nome do segurado
   - NB (número do benefício)
   - Espécie do benefício
   - DIB (Data de Início do Benefício)
   - RMI atual (INSS)

II. OBJETO
   "O presente parecer tem por objeto a auditoria técnica
    do cálculo da Renda Mensal Inicial (RMI) do benefício
    NB [...], concedido em [...], demonstrando erro de
    cálculo e a RMI correta devida."

III. ANÁLISE TÉCNICA
    A. PBC Aplicado
       - PBC INSS: [...]
       - PBC correto: [...]
       - Erro identificado: [...]

    B. Salário de Benefício (SB)
       - SB INSS: R$ [...]
       - SB correto: R$ [...]
       - Erro identificado: [...]

    C. Coeficiente / Fator Previdenciário
       - Coef/Fator INSS: [...]
       - Coef/Fator correto: [...]
       - Erro identificado: [...]

    D. RMI Final
       - RMI INSS: R$ [...]
       - RMI correta: R$ [...]
       - **Diferença: R$ [...]/mês**

IV. FUNDAMENTAÇO JURÍDICA
    - Lei 8.213/91, Art. [...]
    - EC 103/2019, Art. [...]
    - Lei 14.331/2022
    - [Jurisprudência, se houver]

V. CONCLUSÃO
   "Diante do exposto, resta demonstrado erro de cálculo
    na RMI do benefício NB [...], devendo ser corrigida
    de R$ [...] para R$ [...], com diferença de R$ [...]
    mensais, além dos atrasados desde a DIB."
\end{verbatim}

\section{Fluxograma de Auditoria Completa}

O processo completo de auditoria de cálculo RMI segue este fluxo:

\begin{estrategiaCJP}
\textbf{FLUXOGRAMA: AUDITORIA COMPLETA DE CÁLCULO RMI}

\textbf{INÍCIO:} Benefício concedido pelo INSS

$\downarrow$

\textbf{ETAPA 1: COLETAR DOCUMENTOS}
\begin{itemize}
    \item Carta de Concessão
    \item CNIS completo
    \item Memória de Cálculo (se houver)
\end{itemize}

$\downarrow$

\textbf{ETAPA 2: VALIDAR PBC}\\
É 80\% ou 100\%? Período está correto?
\begin{itemize}
    \item OK $\rightarrow$ Próxima etapa
    \item ERRO $\rightarrow$ Documentar erro de PBC
\end{itemize}

$\downarrow$

\textbf{ETAPA 3: VALIDAR SB}\\
Soma OK? Divisor Mínimo aplicado? Cálculo aritmético OK?
\begin{itemize}
    \item OK $\rightarrow$ Próxima etapa
    \item ERRO $\rightarrow$ Documentar erro de SB
\end{itemize}

$\downarrow$

\textbf{ETAPA 4: VALIDAR COEFICIENTE}\\
Tempo total correto? Fórmula correta? Base de cálculo correta?
\begin{itemize}
    \item OK $\rightarrow$ Próxima etapa
    \item ERRO $\rightarrow$ Documentar erro de Coeficiente
\end{itemize}

$\downarrow$

\textbf{ETAPA 5: VALIDAR FATOR PREV (se aplicável)}\\
Tabela IBGE atualizada? Idade/tempo corretos?
\begin{itemize}
    \item OK $\rightarrow$ Próxima etapa
    \item ERRO $\rightarrow$ Documentar erro de Fator
\end{itemize}

$\downarrow$

\textbf{ETAPA 6: VALIDAR RMI FINAL}\\
RMI = SB $\times$ Coef/Fator? Teto/Piso respeitados?
\begin{itemize}
    \item OK $\rightarrow$ Próxima etapa
    \item ERRO $\rightarrow$ Documentar erro de RMI
\end{itemize}
\end{estrategiaCJP}

\begin{conceitoChave}
\textbf{DECISÃO FINAL: Algum erro identificado?}

\textbf{NÃO:}\\
$\rightarrow$ Benefício está correto\\
$\rightarrow$ Arquivar caso

\textbf{SIM:}\\
$\rightarrow$ Montar dossiê técnico (Parecer + Cálculos + Docs)\\
$\rightarrow$ Protocolar Recurso Administrativo ou Ação de Revisão\\
$\rightarrow$ Aguardar análise/decisão
\end{conceitoChave}

\section{Checklist Executivo de Auditoria RMI}

\begin{acaoImediata}
\textbf{CHECKLIST MASTER: AUDITORIA DE CÁLCULO RMI}

\begin{itemize}
    \item[$\square$] Carta de concessão obtida
    \item[$\square$] Memória de cálculo obtida (se disponível)
    \item[$\square$] CNIS atualizado e conferido
    \item[$\square$] PBC validado (80\% ou 100\%)
    \item[$\square$] SB recalculado manualmente
    \item[$\square$] Divisor mínimo verificado
    \item[$\square$] Coeficiente conferido e recalculado
    \item[$\square$] Fator Prev. verificado (se aplicável)
    \item[$\square$] RMI final conferida
    \item[$\square$] Teto e Piso checados
    \item[$\square$] Diferenças documentadas
    \item[$\square$] Impacto vitalício calculado
    \item[$\square$] Parecer técnico elaborado
    \item[$\square$] Recurso/Revisão preparado
\end{itemize}

\textbf{RESULTADO DA AUDITORIA:}
\begin{itemize}
    \item[$\square$] Cálculo CORRETO --- Nenhuma ação necessária
    \item[$\square$] Cálculo com ERRO --- Iniciar revisão/recurso
\end{itemize}
\end{acaoImediata}

%% Continua no Módulo 7

%% ============================================================================
%% INFOGRÁFICO DO MÓDULO 6
%% ============================================================================
\clearpage
\backtotoc

\section*{\faImage\ Infográfico de Consolidação}

\begin{figure}[H]
    \centering
    \begin{tcolorbox}[colback=white, colframe=cjpAzulEscuro, title={\textbf{\faBookOpen\ Infográfico: Módulo 6 --- Auditoria RMI}}, fonttitle=\bfseries\color{white}, sharp corners=downhill, boxrule=2pt]
        \centering
        \includegraphics[width=0.95\textwidth, keepaspectratio]{modulo6}
    \end{tcolorbox}
    \caption{Resumo Visual do Módulo 6: Auditoria RMI}
    \label{fig:modulo6}
\end{figure}
