
\clearpage
\chapter*{Módulo 7: Benefícios Não Programáveis}
\addcontentsline{toc}{chapter}{Módulo 7: Benefícios Não Programáveis}
\markboth{Módulo 7: Benefícios Não Programáveis}{Módulo 7: Benefícios Não Programáveis}
\setcounter{chapter}{7}

\begin{center}
{\Large\textit{``Pensão, Salário-Maternidade e Auxílios: O que Ninguém Te Ensinou''}}\\[0.5cm]
\textbf{Sistema CJP | Pilar 4 de 5 | Cálculos Especializados}
\end{center}

%% \tableofcontents removido - sumário único no master

\section{Pensão por Morte: O Benefício Mais Polêmico}

\subsection{Regras Pré e Pós-Reforma}

A \textbf{Pensão por Morte} passou por uma transformação radical com a EC 103/2019.

\subsubsection{Regra Pré-Reforma (Óbitos até 12/11/2019)}

\begin{verbatim}
CÁLCULO PRÉ-REFORMA:

Se o falecido ERA aposentado:
   RMI = 100% da aposentadoria do falecido

Se o falecido NÃO era aposentado:
   RMI = 100% do SB que ele TERIA direito
         (calculado como Aposentadoria por Invalidez)

SIMPLES: 100% da base, independente de quantos dependentes
\end{verbatim}

\subsubsection{Regra Pós-Reforma (Óbitos a partir de 13/11/2019)}

\begin{verbatim}
CÁLCULO PÓS-REFORMA:

Base de cálculo:
• Se era aposentado: Aposentadoria recebida
• Se não era: SB que TERIA direito + Coeficiente (60% + 2%)

Coeficiente da pensão:
• 50% (cota familiar)
• + 10% por dependente (até 100%)

FÓRMULA:
RMI = Base x (50% + 10% x número de dependentes)
\end{verbatim}

\textbf{Exemplo prático:}

\begin{verbatim}
ÓBITO PÓS-REFORMA (Caso Maria)

Dados:
• Óbito: 20/01/2025
• Aposentadoria do falecido: R$ 6.000,00
• Dependentes: Cônjuge + 2 filhos menores = 3

Cálculo:
Coeficiente = 50% + (3 x 10%) = 80%
RMI = R$ 6.000 x 0,80 = R$ 4.800,00

Cota por dependente:
R$ 4.800 / 3 = R$ 1.600,00 para cada
\end{verbatim}

\subsection{Exceção: Dependente Deficiente = 100\%}

\begin{conceitoChave}
\textbf{EXCEÇO LEGAL: DEFICIÊNCIA}

Se houver dependente com \textbf{deficiência intelectual, mental ou grave}, o coeficiente é SEMPRE 100\%.

\textbf{Base legal:} Art. 23, \S 2\textsuperscript{o}, EC 103/2019

\textbf{Impacto:} Diferença de até 50\% no valor da pensão!
\end{conceitoChave}

\begin{teseRevisional}
\textbf{TESE: Revisão de Pensão (Deficiência)}

\textbf{Requisitos:}
\begin{itemize}
    \item Pensão concedida pós-13/11/2019
    \item Cálculo com coeficiente $<$ 100\%
    \item Existência de dependente inválido/deficiente NO MOMENTO DO ÓBITO
\end{itemize}

\textbf{Fundamento:} Art. 23, \S 2\textsuperscript{o}, EC 103/2019

\textbf{Ganho médio:} R\$ 1.000,00 a R\$ 2.500,00/mês\\
\textbf{Retroativos:} Desde a DER (até 5 anos)\\
\textbf{Taxa de êxito:} 85\% (administrativa)
\end{teseRevisional}

\subsection{Lei 15.108/2025: Menor Sob Guarda Equiparado a Filho}

\begin{novidade}
\textbf{Lei 15.108/2025 (13/03/2025)}

A Lei 15.108/2025 alterou o Art. 16, \S 2\textsuperscript{o} da Lei 8.213/91 para equiparar expressamente o \textbf{menor sob guarda judicial} a filho biológico para fins previdenciários.

\textbf{ANTES da lei:}
\begin{itemize}
    \item[\cmark] Cônjuge
    \item[\cmark] Companheiro(a)
    \item[\cmark] Filho não emancipado até 21 anos
    \item[\cmark] Enteado até 21 anos
    \item[\cmark] Menor sob TUTELA
    \item[\xmark] Menor sob GUARDA $\rightarrow$ NÃO era dependente automático
\end{itemize}

\textbf{DEPOIS da lei (a partir de 13/03/2025):}
\begin{itemize}
    \item[\cmark] Cônjuge
    \item[\cmark] Companheiro(a)
    \item[\cmark] Filho não emancipado até 21 anos
    \item[\cmark] Enteado até 21 anos
    \item[\cmark] Menor sob TUTELA
    \item[\cmark] \textbf{Menor sob GUARDA JUDICIAL} $\leftarrow$ NOVO!
\end{itemize}
\end{novidade}

\subsubsection{Impacto Financeiro da Lei 15.108/2025}

\begin{estrategiaCJP}
\textbf{CASO PRÁTICO: IMPACTO FINANCEIRO}

\textbf{SITUAÇO:}
\begin{itemize}
    \item Óbito: 20/08/2025 (após a lei)
    \item Segurado tinha aposentadoria de R\$ 5.000,00/mês
    \item Dependentes originais: Cônjuge + 2 filhos biológicos
    \item Menor sob guarda desde 2023 (sobrinho do falecido)
\end{itemize}

\textbf{CÁLCULO ANTES DA LEI} (Ignorando o menor sob guarda):
\begin{verbatim}
Base: R$ 5.000
Dependentes contados: 3 (cônjuge + 2 filhos)
Coeficiente: 50% + 30% = 80%
Pensão Total: R$ 5.000 x 80% = R$ 4.000
Por dependente: R$ 4.000 / 3 = R$ 1.333,33
\end{verbatim}

\textbf{CÁLCULO DEPOIS DA LEI} (Incluindo o menor sob guarda):
\begin{verbatim}
Base: R$ 5.000
Dependentes contados: 4 (cônjuge + 2 filhos + menor guarda)
Coeficiente: 50% + 40% = 90%
Pensão Total: R$ 5.000 x 90% = R$ 4.500
Por dependente: R$ 4.500 / 4 = R$ 1.125,00
\end{verbatim}

\textbf{ANÁLISE:}
\begin{itemize}
    \item Pensão TOTAL aumentou: R\$ 500,00/mês (+12,5\%)
    \item Cota INDIVIDUAL diminuiu: -R\$ 208,33/pessoa (-15,6\%)
    \item Mas a FAMÍLIA recebe mais no total
    \item Menor sob guarda agora recebe: R\$ 1.125,00/mês (antes R\$ 0,00)
\end{itemize}
\end{estrategiaCJP}

\begin{teseRevisional}
\textbf{TESE: Inclusão de Menor sob Guarda (Lei 15.108/2025)}

\textbf{Requisitos:}
\begin{itemize}
    \item Pensão concedida APÓS 13/03/2025
    \item Menor sob guarda judicial existente
    \item Menor NÃO foi contado inicialmente
\end{itemize}

\textbf{Fundamento:} Lei 15.108/2025 c/c Art. 16, \S 2\textsuperscript{o}, Lei 8.213/91

\textbf{Ganho médio:} +10\% no coeficiente (R\$ 300,00-800/mês)\\
\textbf{Taxa de êxito:} 95\% (administrativa)\\
\textbf{Prazo médio de análise:} 60-90 dias
\end{teseRevisional}

\subsection{Armadilhas Comuns em Pensão por Morte}

\begin{armadilha}
\textbf{ARMADILHA \#1: Não Contar Todos os Dependentes}

\textbf{Problema:} INSS às vezes não contabiliza menor sob guarda, enteado ou dependente inválido corretamente

\textbf{Impacto:} -10\% a -20\% no coeficiente

\textbf{Como identificar:} Conferir quem foi listado na carta de concessão e comparar com certidões

\textbf{Solução:} Protocolar inclusão de dependente + revisão
\end{armadilha}

\begin{armadilha}
\textbf{ARMADILHA \#2: Aplicar Regra Errada (Pré vs. Pós)}

\textbf{Problema:} Óbito em Nov/2019 (período de transição) pode gerar confusão sobre qual regra aplicar

\textbf{Impacto:} -20\% a -40\% no valor da pensão

\textbf{Como identificar:} Verificar data EXATA do óbito (antes ou depois de 13/11/2019 00h00)

\textbf{Solução:} Recurso administrativo demonstrando data correta
\end{armadilha}

\begin{armadilha}
\textbf{ARMADILHA \#3: Não Reconhecer Deficiência = 100\%}

\textbf{Problema:} Dependente inválido/deficiente grave existe, mas INSS calcula com coeficiente parcial

\textbf{Impacto:} -10\% a -40\% (perdendo o 100\% garantido)

\textbf{Como identificar:} Verificar se há laudo médico atestando invalidez/deficiência grave

\textbf{Solução:} Perícia médica + recurso (100\% garantido por lei)
\end{armadilha}

\subsection{Checklist de Auditoria de Pensão por Morte}

\begin{acaoImediata}
\textbf{CHECKLIST EXECUTIVO --- PENSÃO}

\textbf{DADOS BÁSICOS}
\begin{itemize}
    \item[$\square$] Data do óbito confirmada?
    \item[$\square$] Qual regra se aplica? (Pré/Pós)
    \item[$\square$] Falecido era aposentado? Qual RMI?
    \item[$\square$] Se não, qual seria o SB?
\end{itemize}

\textbf{DEPENDENTES}
\begin{itemize}
    \item[$\square$] Todos dependentes listados?
    \item[$\square$] Menor sob guarda foi contado? (se óbito após 13/03/2025)
    \item[$\square$] Enteados foram contados?
    \item[$\square$] Há inválido/deficiente grave?
\end{itemize}

\textbf{CÁLCULO}
\begin{itemize}
    \item[$\square$] Base correta identificada?
    \item[$\square$] Coeficiente calculado certo?
    \item[$\square$] Se deficiente: coef = 100\%?
    \item[$\square$] RMI final confere?
\end{itemize}

\textbf{OPORTUNIDADES DE REVISÃO}
\begin{itemize}
    \item[$\square$] Lei 15.108/2025 aplicável?
    \item[$\square$] Deficiente não reconhecido?
    \item[$\square$] Dependente não contado?
\end{itemize}

\textbf{DOCUMENTAÇO}
\begin{itemize}
    \item[$\square$] Certidão de óbito
    \item[$\square$] Carta de concessão da pensão
    \item[$\square$] Termo de guarda (se aplicável)
    \item[$\square$] Laudo médico (se deficiente)
    \item[$\square$] Certidões de nascimento/casamento
\end{itemize}
\end{acaoImediata}

\section{Salário-Maternidade: O Benefício Universal}

\subsection{Por Que Salário-Maternidade é Único?}

O \textbf{Salário-Maternidade} é o \textbf{único benefício previdenciário} que:

\begin{enumerate}
    \item \textbf{Não exige carência} (após IN 188/2025)
    \item \textbf{Tem cálculo diferente por categoria} profissional
    \item \textbf{Foi revolucionado DUAS VEZES em 2025} (IN 188 + Lei 15.222)
\end{enumerate}

\begin{conceitoChave}
\textbf{UNIVERSALIDADE}

Toda segurada tem direito, desde que:
\begin{itemize}
    \item Seja segurada do RGPS (qualquer categoria)
    \item Tenha dado à luz/adotado
    \item Tenha pelo menos 1 contribuição (após IN 188/2025)
\end{itemize}

Não importa se é empregada, autônoma, MEI, rural, doméstica --- TODAS têm direito. O que muda é COMO calcular.
\end{conceitoChave}

\subsection{Cálculo por Categoria Profissional}

\subsubsection{Categoria 1: Empregada CLT (Urbana ou Rural)}

\begin{verbatim}
REGRA GERAL:
RMI = Remuneração integral do mês de afastamento

Se remuneração é FIXA:
• Exemplo: Salário de R$ 3.500/mês
• RMI = R$ 3.500

Se remuneração é VARIÁVEL (comissões, horas extras, etc.):
• Média dos 6 últimos meses
• Exemplo: Últimos 6 meses: R$ 3.200, R$ 3.800, R$ 3.500,
           R$ 3.600, R$ 3.400, R$ 3.700
• Soma: R$ 21.200
• Média: R$ 21.200 / 6 = R$ 3.533,33
• RMI = R$ 3.533,33
\end{verbatim}

\textbf{Quem paga:}
\begin{itemize}
    \item Empresa PAGA à empregada
    \item Empresa DEDUZ da GPS
    \item INSS reembolsa a empresa
    \item A empregada NÃO precisa pedir ao INSS! É automático via folha de pagamento.
\end{itemize}

\subsubsection{Categoria 2: Empregada Doméstica}

\begin{verbatim}
REGRA:
RMI = Último salário-de-contribuição

Exemplo:
• Salário registrado na carteira: R$ 1.800/mês
• RMI = R$ 1.800

Se o salário variou nos últimos meses:
• Usa-se o último salário-de-contribuição
• (não é média, é o último mesmo)
\end{verbatim}

\subsubsection{Categoria 3: Contribuinte Individual / Facultativa / MEI}

\begin{verbatim}
REGRA:
RMI = 1/12 da soma dos 12 últimos salários-de-contribuição
      (dentro de período máximo de 15 meses antes do afastamento)

FÓRMULA:
RMI = (Soma dos 12 últimos SC) / 12

Exemplo (MEI):
• Contribuições: R$ 1.621,00 (salário mínimo) x 12 meses
• Soma: R$ 19.452
• RMI: R$ 19.452 / 12 = R$ 1.621,00

Exemplo (Autônoma - contribui sobre valor maior):
• Contribuições variáveis nos últimos 12 meses
• Soma: R$ 37.600
• RMI: R$ 37.600 / 12 = R$ 3.133,33
\end{verbatim}

\subsubsection{Categoria 4: Segurada Especial (Rural sem CNPJ)}

\begin{verbatim}
REGRA SIMPLES:
RMI = 1 salário mínimo

Valor 2026: R$ 1.621,00

NÃO importa:
• Quanto ela produz
• Quantos hectares tem
• Renda familiar

Salário-maternidade da segurada especial é SEMPRE 1 SM.
\end{verbatim}

\subsection{IN 188/2025: A Revolução da Carência Zero}

\begin{novidade}
\textbf{IN PRES/INSS 188/2025 (10/07/2025)}

A IN 188/2025 regulamentou a decisão do STF na ADI 2.110 que declarou INCONSTITUCIONAL a exigência de carência para salário-maternidade.

\textbf{ANTES (até 09/07/2025):}
\begin{itemize}
    \item Empregada/Doméstica: SEM carência
    \item Contribuinte Individual/Facultativa: 10 MESES de carência
    \item Segurada Especial: SEM carência
\end{itemize}

\textbf{DEPOIS (a partir de 10/07/2025):}

\textbf{TODAS AS CATEGORIAS: SEM CARÊNCIA}

\textbf{Requisito único:}
\begin{itemize}
    \item Ter pelo menos 1 contribuição válida antes do parto/adoção
    \item Estar na qualidade de segurada
\end{itemize}

\textbf{Aplicação:}
\begin{itemize}
    \item Pedidos feitos APÓS 05/04/2024 (data da decisão STF)
    \item Retroage até essa data
\end{itemize}
\end{novidade}

\begin{teseRevisional}
\textbf{TESE: Salário-Maternidade (Isenção de Carência --- IN 188)}

\textbf{Requisitos:}
\begin{itemize}
    \item Pedido INDEFERIDO por falta de carência
    \item Pedido feito APÓS 05/04/2024
    \item Segurada tinha pelo menos 1 contribuição
\end{itemize}

\textbf{Fundamento:} STF (ADI 2.110) c/c IN PRES/INSS 188/2025

\textbf{Valor médio:} R\$ 1.621,00 a R\$ 3.500,00/mês $\times$ 4-5 meses (120-150 dias)\\
\textbf{Retroativos:} Desde a DER original (pode ser 12-18 meses)\\
\textbf{Taxa de êxito:} 98\% (administrativa)\\
\textbf{Prazo:} 30-60 dias
\end{teseRevisional}

\subsection{Lei 15.222/2025: Prorrogação por Internação}

\begin{novidade}
\textbf{Lei 15.222/2025 (30/09/2025)}

A Lei 15.222/2025 alterou o Art. 71 da Lei 8.213/91 e o Art. 392 da CLT para prorrogar o salário-maternidade em casos de \textbf{internação hospitalar prolongada}.

\textbf{REQUISITO:}

Internação hospitalar da mãe OU do recém-nascido por MAIS DE 2 SEMANAS (14 dias) em decorrência de complicações relacionadas à gestação ou ao parto.

\textbf{NOVA REGRA DE CÁLCULO DO PERÍODO:}
\begin{enumerate}
    \item Salário-maternidade durante TODA a internação
    \item + 120 dias APÓS a alta hospitalar
    \item - DESCONTADO o tempo já recebido antes do parto
\end{enumerate}

\textbf{PERÍODO MÁXIMO:}

Pode ultrapassar os 120 dias tradicionais, não há limite enquanto durar a internação + 120 dias pós-alta.
\end{novidade}

\subsubsection{Exemplo Prático}

\begin{estrategiaCJP}
\textbf{CASO: Internação Prolongada}

\textbf{DADOS:}
\begin{itemize}
    \item Parto: 01/10/2025
    \item Bebê nasceu prematuro (7 meses)
    \item Internação do bebê: 01/10 a 05/11 (35 dias)
    \item Alta: 06/11/2025
\end{itemize}

\textbf{CÁLCULO DO PERÍODO:}

\textbf{ANTES DA LEI} (regra tradicional):
\begin{itemize}
    \item 120 dias corridos do parto
    \item Fim: 29/01/2026
    \item Total de dias: 120
\end{itemize}

\textbf{DEPOIS DA LEI 15.222/2025:}
\begin{itemize}
    \item Durante internação: 35 dias (01/10 a 05/11)
    \item Pós-alta: 120 dias (06/11 a 05/03/2026)
    \item Total de dias: 155 dias
    \item Diferença: +35 dias
\end{itemize}

\textbf{VALOR ADICIONAL (exemplo):}
\begin{itemize}
    \item RMI: R\$ 3.500,00/mês
    \item Diária: R\$ 3.500,00 $\div$ 30 = R\$ 116,67
    \item Ganho: 35 dias $\times$ R\$ 116,67 = R\$ 4,00.083,45
\end{itemize}
\end{estrategiaCJP}

\begin{teseRevisional}
\textbf{TESE: Prorrogação Salário-Maternidade (Internação --- Lei 15.222/2025)}

\textbf{Requisitos:}
\begin{itemize}
    \item Parto APÓS 30/09/2025
    \item Internação $>$ 14 dias (mãe ou bebê)
    \item Benefício já concedido com 120 dias (sem prorrogação)
\end{itemize}

\textbf{Fundamento:} Lei 15.222/2025 c/c Art. 71, Lei 8.213/91

\textbf{Ganho médio:} 15-90 dias adicionais (R\$ 1.500,00 a R\$ 10.000,00)

\textbf{Documentação necessária:}
\begin{itemize}
    \item Relatório médico da internação
    \item Data de alta hospitalar
    \item Atestado de complicação gestacional
\end{itemize}

\textbf{Taxa de êxito:} 90\% (administrativa)
\end{teseRevisional}

%% Continua na Parte 7B
