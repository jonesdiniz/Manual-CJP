%% ============================================================================
%% CAPA
%% ============================================================================
\begin{titlepage}
\begin{center}
    \vspace*{1cm}

    {\fontsize{48}{58}\selectfont\bfseries\textcolor{cjpAzulEscuro}{SISTEMA CJP}}
    \vspace{0.1cm}
    {\fontsize{28}{34}\selectfont\textcolor{cjpAzulClaro}{COMPLETO}}

    \vspace{0.8cm}
    {\Large\textcolor{cjpCinzaEscuro}{Maestria em Planejamento Previdenciário Estratégico}}
    \vspace{0.5cm}

    \begin{tikzpicture}
    \draw[cjpDourado, line width=2pt] (0,0) -- (12,0);
    \end{tikzpicture}

    \vspace{0.5cm}
    {\large\textit{\textcolor{cjpCinzaEscuro}{Os 5 Pilares do Método CJP: Da Entrevista ao Parecer Irrefutável}}}

    \vspace{1.5cm}
    \textbf{Dr. Jones Weslley Bueno Diniz}\\
    OAB/SP 377.329\\
    \textit{Especialista em Direito Previdenciário}

    \vspace{0.5cm}
    Método CJP | Cálculos Jurídicos Precisos

    \vfill

    % BLOCO DE ATUALIZAÇO LEGISLATIVA COMPACTO
    \begin{tcolorbox}[colback=cjpCinzaEscuro!10, colframe=cjpAzulEscuro, width=0.95\textwidth, arc=0mm, boxrule=0.5pt]
        \centering
        \textbf{EDIÇO 2026 (REVISADA E ATUALIZADA)}\\[0.2cm]
        \footnotesize
        Incluindo: EC 103/2019 $\cdot$ Lei 14.331/2022 $\cdot$ Leis 15.108 e 15.222/2025\\
        Tema 1300 STF $\cdot$ Portaria Conjunta DIT/DIRBEN/INSS n.\textsuperscript{o} 10/2025\\
        IN 188/2025 $\cdot$ Portaria DIRBEN/INSS n.\textsuperscript{o} 1.316/2025\\
        Temas STF/STJ 2025-2026
    \end{tcolorbox}
\end{center}
\end{titlepage}

%% ============================================================================
%% DEDICATÓRIA
%% ============================================================================
\newpage
\thispagestyle{empty}
\vspace*{5cm}
\begin{flushright}
\textit{\large A todos os advogados que não se contentam com o superficial.}\\[0.3cm]
\textit{\large Aos que entendem que o Direito Previdenciário não é sorte,}\\[0.3cm]
\textit{\large mas sim uma ciência exata de estratégia, cálculo e justiça.}\\[0.5cm]
\textit{\large Este sistema é para vocês.}
\end{flushright}
\vfill

%% ============================================================================
%% EPÍGRAFE
%% ============================================================================
\newpage
\thispagestyle{empty}
\vspace*{8cm}
\begin{flushright}
\textit{\large Dê-me seis horas para derrubar uma árvore}\\[0.3cm]
\textit{\large e passarei as quatro primeiras afiando o machado.}\\[1cm]
\textbf{--- Abraham Lincoln}
\end{flushright}
\vfill

%% ============================================================================
%% SOBRE O AUTOR
%% ============================================================================
\chapter*{Sobre o Autor}
\addcontentsline{toc}{chapter}{Sobre o Autor}

\begin{center}
{\Large\bfseries Dr. Jones Weslley Bueno Diniz}\\[0.3cm]
{\large OAB/SP 377.329 | 9 Anos de Experiência em Planejamento Previdenciário Estratégico}
\end{center}

\vspace{1cm}

Você já perdeu o sono pensando se realmente entregou o \textbf{melhor caminho} para seu cliente?

Eu passei por isso. Centenas de vezes.

Um cliente entra no escritório com uma pasta cheia de documentos. CNIS impresso. Carta de Concessão. CTPS com vários vínculos. Ele pergunta: \textit{``Doutor, quando posso me aposentar? E por quanto?''}

E você, advogado sério, abre seu software de cálculos, digita os dados e... \textbf{recebe uma resposta.}

Mas será que aquela é a \textbf{resposta certa}? Será que você não deixou dinheiro na mesa? Será que não ignorou um vínculo rural da infância dele? Será que aquele tempo especial foi convertido corretamente? Será que o software aplicou a regra de transição mais vantajosa?

\textbf{A verdade incômoda:} A maioria dos softwares te dá \textit{uma} resposta, não a \textit{melhor} resposta.

E foi para resolver isso — essa insegurança profissional que todo previdenciarista sente mas poucos admitem — que criei o \textbf{Método CJP (Cálculos Jurídicos Precisos)}.

Este e-book não é um manual para ``aprender a calcular''. É um \textbf{sistema completo de planejamento previdenciário estratégico} que te ensina, passo a passo, a:

\begin{enumerate}
    \item \textbf{Entrevistar estrategicamente} para descobrir o que o CNIS não mostra
    \item \textbf{Diagnosticar impecavelmente} todos os erros e oportunidades ocultos
    \item \textbf{Acertar vínculos e documentação} com precisão jurídica
    \item \textbf{Calcular sistematicamente} todas as possibilidades (não apenas uma)
    \item \textbf{Entregar um parecer irrefutável} que justifica honorários premium
\end{enumerate}

\textbf{Este é o método que me posicionou como referência em Planejamento Previdenciário no Vale do Paraíba.}

E agora está disponível para você.

\subsection*{Formação e Especialização}

\begin{itemize}
    \item \textbf{Especialista} em Direito Previdenciário pela Universidade Municipal de São Caetano do Sul (USCS)
    \item \textbf{Especialista} em Planejamento Previdenciário Estratégico
    \item \textbf{Criador} do Sistema CJP de Planejamento Previdenciário
    \item \textbf{Criador} do Método CJP (Cálculos Jurídicos Precisos)
    \item \textbf{Pioneiro} na integração de IA e automação no Direito Previdenciário
\end{itemize}

\subsection*{Valores Profissionais}

\textbf{Honestidade | Respeito | Responsabilidade | Integridade | Inovação | Excelência}

\subsection*{Especialidades Core (Método CJP)}

\begin{itemize}
    \item[\cmark] Planejamento Previdenciário Estratégico (5 Pilares)
    \item[\cmark] Entrevista e Diagnóstico Previdenciário
    \item[\cmark] Análise e Correção Jurídica do CNIS
    \item[\cmark] Auditoria de Cálculos Previdenciários
    \item[\cmark] Acertos de Vínculos e Documentação Administrativa
    \item[\cmark] Atividade Especial \& Conversão de Tempos
    \item[\cmark] Teses Revisionais e Contenciosos Estratégicos
    \item[\cmark] Tecnologia Jurídica \& Automação de Processos
\end{itemize}



%% ============================================================================
%% COPYRIGHT & DISCLAIMER
%% ============================================================================
\chapter*{Copyright \& Disclaimer}
\addcontentsline{toc}{chapter}{Copyright \& Disclaimer}

\begin{center}
{\large\bfseries \textcopyright{} 2026 Jones Weslley Bueno Diniz}\\[0.3cm]
\textbf{Método CJP | Cálculos Jurídicos Precisos}
\end{center}

\vspace{0.5cm}

Todos os direitos reservados. Nenhuma parte deste material pode ser reproduzida, distribuída ou transmitida sem permissão expressa e por escrito do autor.

\subsection*{\faBalanceScale\ Aviso Legal Importante}

\textbf{1. Finalidade Educacional:} Este E-book foi desenvolvido como material educativo e de referência técnica para profissionais do Direito Previdenciário. \textbf{NÃO CONSTITUI CONSULTORIA JURÍDICA ESPECÍFICA} ou parecer jurídico para caso concreto.

\textbf{2. Não Substitui Análise Personalizada:} Cada situação previdenciária é única e requer análise individualizada por advogado especializado. O Método CJP é uma estrutura de trabalho, não uma fórmula automática.

\textbf{3. Validade Temporal:} Este material reflete a legislação e jurisprudência vigentes até \textbf{Janeiro de 2026}, incluindo:

\begin{center}
\begin{tabular}{cl}
\xmark & Tema 1300 STF — Tese REJEITADA (6$\times$5) \\
\cmark & Cancelamento definitivo da Revisão da Vida Toda (Tema 1102) \\
\cmark & Tema 1124 STJ — Interesse de agir e DIB \\
\cmark & Tema 1090 STJ — EPI e tempo especial \\
\cmark & Procuração Eletrônica Meu INSS (Portaria 10/2025) \\
\cmark & Indicadores CNIS atualizados (Portaria 1.316/2025) \\
\cmark & Regras de transição 2026 (93/103 pontos) \\
\cmark & Valores previdenciários 2026 (previsão oficial) \\
\end{tabular}
\end{center}

\begin{armadilha}
\textbf{NOTA IMPORTANTE:} O salário mínimo de 2026 foi confirmado em R\$ 1.621,00. O teto INSS tem previsão de R\$ 8.537,55* (aguardando confirmação oficial).
\end{armadilha}

\textbf{4. Responsabilidade Limitada:} O autor não se responsabiliza por perdas, danos ou consequências diretas/indiretas decorrentes do uso ou interpretação deste material sem orientação profissional adequada ou sem análise crítica do contexto específico.

\textbf{5. Atualização Contínua Recomendada:} A legislação previdenciária é dinâmica. STF e STJ realizam julgamentos constantemente. Recomenda-se acompanhamento jurisprudencial regular e participação em comunidades de atualização profissional.

\textbf{6. Aplicação Prática com Discernimento:} Qualquer aplicação prática das informações aqui contidas deve ser feita sob responsabilidade do profissional, com análise técnica específica do caso concreto. O método ensina ``como pensar'', não ``o que pensar''.

\subsection*{\faLock\ Propriedade Intelectual}

\textbf{Método CJP (Cálculos Jurídicos Precisos)} é metodologia proprietária desenvolvida sob autoria e propriedade intelectual do Dr. Jones Weslley Bueno Diniz.

\textbf{Sistema CJP Completo}, \textbf{Os 5 Pilares do Planejamento Previdenciário} e \textbf{Dossiê Estratégico CJP} são marcas e frameworks proprietários protegidos por direitos autorais.

\subsection*{\faShieldAlt\ Nota Legal e Isenção de Responsabilidade}

\begin{tcolorbox}[colback=gray!5, colframe=cjpCinzaEscuro, title={\textbf{DISCLAIMER JURÍDICO}}, fonttitle=\bfseries\color{white}]

\textbf{Natureza Educacional:} Este material tem caráter estritamente didático e educacional. O ``Sistema CJP'' é uma metodologia de organização e raciocínio jurídico desenvolvida pelo autor com base na legislação vigente até a data de fechamento desta edição (Janeiro/2026).

\textbf{Ausência de Vinculação:} As teses, modelos de cálculo e estratégias aqui apresentadas não garantem resultado em demandas judiciais ou administrativas, uma vez que o Direito é uma ciência interpretativa e sujeita à convicção de magistrados e servidores, bem como a alterações legislativas supervenientes.

\textbf{Responsabilidade Profissional:} O uso deste manual \textbf{não substitui} a análise técnica individualizada de cada caso concreto. Cabe exclusivamente ao profissional (leitor) a responsabilidade pela conferência dos dados, prazos e aplicação da legislação adequada aos seus clientes. O autor se exime de qualquer responsabilidade por prejuízos diretos ou indiretos decorrentes da utilização prática das informações contidas nesta obra sem a devida cautela profissional.

\textbf{Direitos Autorais:} É vedada a reprodução total ou parcial desta obra, por qualquer meio ou processo, inclusive quanto às características gráficas e editoriais, sem a expressa autorização do autor (Lei n.\textsuperscript{o} 9.610/98).

\end{tcolorbox}


%% ============================================================================
%% COMO USAR ESTE SISTEMA
%% ============================================================================
\chapter*{Como Usar Este Sistema}
\addcontentsline{toc}{chapter}{Como Usar Este Sistema}

\section*{\faTarget\ Este NÃO é um Manual para LER. É um SISTEMA para APLICAR.}

O advogado previdenciário não lê manuais de forma linear do início ao fim. Ele os \textbf{consulta no momento da dor}, quando um cliente específico apresenta uma situação que exige decisão imediata.

Mas este e-book vai além da consulta pontual. Ele te ensina um \textbf{método sistemático} que você aplica em \textbf{todos os seus casos}, da primeira consulta até a entrega do parecer final.

\section*{\faBuilding\ Arquitetura do Sistema CJP: 5 Pilares + 9 Módulos}

Este e-book está organizado em \textbf{5 Pilares Sequenciais}, cada um cobrindo uma etapa essencial do planejamento previdenciário:

\begin{estrategiaCJP}
\textbf{SISTEMA CJP COMPLETO - ARQUITETURA}

\textbf{\faMicrophone\ PILAR 1: A ENTREVISTA ESTRATÉGICA}\\
└─ Módulo 1: Como descobrir o que o CNIS não mostra\\
\hspace*{1cm}\textcolor{cjpLaranja}{\textbf{NOVIDADE 2026:}} Procuração Eletrônica Meu INSS!

\textbf{\faSearch\ PILAR 2: O DIAGNÓSTICO IMPECÁVEL}\\
├─ Módulo 2: Auditoria do CNIS (Diagnóstico Explícito)\\
└─ Módulo 3: Armadilhas Ocultas (Diagnóstico Implícito)

\textbf{\faWrench\ PILAR 3: OS ACERTOS DE VÍNCULOS}\\
└─ Módulo 4: Acertos e Documentação na Prática

\textbf{\faCalculator\ PILAR 4: OS CÁLCULOS SISTEMATIZADOS}\\
├─ Módulo 5: PBC e Regras de Transição (2026)\\
├─ Módulo 6: Auditoria do Cálculo RMI\\
└─ Módulo 7: Benefícios Não Programáveis (2026)

\textbf{\faFileAlt\ PILAR 5: A ENTREGA DE VALOR}\\
└─ Módulo 8: O Parecer Irrefutável e a Consultoria

\textbf{\faGift\ BÔNUS}\\
└─ Módulo 9: Teses Revisionais e Modelos Práticos\\
\hspace*{1cm}\textcolor{cjpVinho}{\xmark\ INCLUI: Tema 1300 (REJEITADO) + Atualização Vida Toda}
\end{estrategiaCJP}

\section*{\faChartBar\ Três Formas de Usar Este E-book}

\subsection*{Forma 1: Consulta Rápida (2-5 minutos)}

\begin{itemize}
    \item \textbf{Quando:} Você tem uma dúvida específica e precisa de resposta imediata
    \item \textbf{Como:} Use o sumário interativo ou o índice de navegação por ``dor''
    \item \textbf{Exemplo:} ``Meu cliente tem um PEXT no CNIS, o que eu faço?''
    \begin{itemize}
        \item Solução: Ir direto ao Módulo 2, Seção 2.3.1 (PEXT)
        \item Tempo: 2-3 minutos para ler a solução completa
    \end{itemize}
\end{itemize}

\subsection*{Forma 2: Aplicação do Método Completo (1-2 horas/cliente)}

\begin{itemize}
    \item \textbf{Quando:} Você quer aplicar o Sistema CJP completo em um caso real
    \item \textbf{Como:} Siga os 5 Pilares na ordem sequencial
    \item \textbf{Fluxo:}
    \begin{enumerate}
        \item \textbf{Pilar 1 (Módulo 1):} Faça a entrevista estratégica com o roteiro CJP (30-40 min)
        \item \textbf{Pilar 2 (Módulos 2-3):} Audite o CNIS e identifique armadilhas (20-30 min)
        \item \textbf{Pilar 3 (Módulo 4):} Monte o dossiê de acertos (se necessário) (15-20 min)
        \item \textbf{Pilar 4 (Módulos 5-6-7):} Calcule todos os cenários possíveis (20-30 min)
        \item \textbf{Pilar 5 (Módulo 8):} Monte e apresente o Parecer Estratégico (30-40 min)
    \end{enumerate}
\end{itemize}

\textbf{Resultado:} Cliente impressionado, decisão clara, honorários justificados.

\subsection*{Forma 3: Estudo para Domínio do Método (3-4 semanas)}

\begin{itemize}
    \item \textbf{Quando:} Você quer \textbf{dominar} o Sistema CJP para se tornar especialista
    \item \textbf{Como:} Estude um pilar por semana, com prática deliberada
\end{itemize}

\textbf{Cronograma Sugerido:}

\textbf{Semana 1 - Diagnóstico:}
\begin{itemize}
    \item Segunda: Módulo 1 (Entrevista) + praticar roteiro em 2 casos
    \item Terça: Módulo 2 (CNIS) - estudar os 15 indicadores
    \item Quarta: Módulo 3 (Armadilhas) - memorizar as 8 armadilhas
    \item Quinta/Sexta: Aplicar Pilares 1-2 em 3 casos reais
\end{itemize}

\textbf{Semana 2 - Acertos e Cálculos:}
\begin{itemize}
    \item Segunda: Módulo 4 (Acertos) + montar 1 RAC prático
    \item Terça: Módulo 5 (PBC e Transições)
    \item Quarta: Módulo 6 (RMI)
    \item Quinta: Módulo 7 (Benefícios 2026)
    \item Sexta: Aplicar Pilares 3-4 em 2 casos reais
\end{itemize}

\textbf{Semana 3 - Entrega e Teses:}
\begin{itemize}
    \item Segunda: Módulo 8 (Parecer) + montar 1 Parecer real
    \item Terça: Módulo 9 Parte A (Teses Revisionais)
    \item Quarta: Módulo 9 Parte B (Modelos)
    \item Quinta/Sexta: Aplicar Sistema CJP completo em 1 caso
\end{itemize}

\textbf{Semana 4 - Consolidação:}
\begin{itemize}
    \item Aplicar Sistema CJP em 5 casos diferentes
    \item Criar seus próprios templates personalizados
    \item Documentar dificuldades e criar FAQs pessoais
    \item Revisar módulos onde teve mais dúvidas
\end{itemize}

\textbf{Ao final das 4 semanas:} Você terá \textbf{dominado} o Método CJP e estará aplicando-o de forma natural e automatizada.

\clearpage
\section*{\faPalette\ Legenda Visual (Sistema de Cores e Ícones)}

Este e-book utiliza \textbf{Legal Design} intensivamente. Cada cor e ícone tem um \textbf{propósito funcional}, não decorativo.

\subsection*{Cores e Significados:}

\begin{itemize}
    \item \textcolor{cjpAzulEscuro}{\textbf{Azul (\#1A365D):}} Conceitos-chave, ações imediatas, links de navegação
    \item \textcolor{cjpVinho}{\textbf{Vinho/Marrom:}} ALERTA DE ARMADILHA | TESE REVISIONAL (atenção máxima)
    \item \textcolor{cjpCinzaEscuro}{\textbf{Cinza Escuro (\#58595B):}} Texto principal, conteúdo técnico
    \item \textcolor{gray}{\textbf{Cinza Claro (\#D1D3D4):}} Boxes informativos, destaques secundários
\end{itemize}

\subsection*{Callouts (Caixas de Destaque):}

Você verá estas caixas ao longo de todo o e-book. \textbf{Nunca ignore um callout.} Eles condensam anos de experiência prática em avisos de 2-3 linhas.

\begin{armadilha}
Indica erro comum que prejudica o cálculo, a estratégia ou a execução.

\textbf{Uso:} Alerta para erros que você \textbf{pode estar cometendo agora} sem saber. Leia com atenção dobrada.
\end{armadilha}

\begin{teseRevisional}
Oportunidade de aumentar o valor do benefício ou corrigir erro do INSS.

\textbf{Uso:} Identifica oportunidades de \textbf{gerar receita adicional} com revisões. Anote os requisitos de cada tese.
\end{teseRevisional}

\begin{acaoImediata}
Passo prático e acionável que deve ser executado agora para resolver o caso.

\textbf{Uso:} Roteiro passo a passo do que fazer. Copie, cole no seu checklist, execute.
\end{acaoImediata}

\begin{conceitoChave}
Fundamento técnico ou jurídico essencial para compreensão do tema.

\textbf{Uso:} Teoria necessária para entender o ``porquê'' antes do ``como''. Leia devagar, internalize.
\end{conceitoChave}

\begin{novidade}
Atualização legislativa, jurisprudencial ou procedimental vigente a partir de 2025/2026. Atenção especial!

\textbf{Uso:} Marca alterações recentes que impactam a prática. Verifique se você está atualizado.
\end{novidade}

\clearpage
\section*{\faMap\ Mapa de Navegação por Dor}

Use esta tabela quando tiver uma \textbf{dor específica} e não souber por onde começar:

\begin{center}
\begin{tabular}{|p{7cm}|p{5cm}|c|}
\hline
\textbf{Sua Dor Atual} & \textbf{Módulo Recomendado} & \textbf{Tempo} \\
\hline
``Não sei o que perguntar na consulta inicial'' & Módulo 1 - Entrevista Estratégica & 10 min \\
\hline
``Como acessar o CNIS do cliente sem pedir senha?'' & Módulo 1 - Procuração Eletrônica & 5 min \\
\hline
``CNIS cheio de indicadores estranhos'' & Módulo 2 - Auditoria do CNIS & 10 min \\
\hline
``Suspeito que algo está faltando no CNIS'' & Módulo 3 - Armadilhas Ocultas & 15 min \\
\hline
``Não sei como fazer um RAC'' & Módulo 4 - Acertos e Documentação & 15 min \\
\hline
``Dúvida sobre média de salários (PBC)'' & Módulo 5 - PBC e Transições & 15 min \\
\hline
``Qual regra de transição se aplica em 2026?'' & Módulo 5 - PBC e Transições & 10 min \\
\hline
``Divisor/coeficiente está correto?'' & Módulo 6 - Auditoria RMI & 15 min \\
\hline
``Cliente com aposentadoria por incapacidade'' & Módulo 7 - Benefícios Não Programáveis & 10 min \\
\hline
``Como apresentar os cenários para o cliente?'' & Módulo 8 - Parecer e Consultoria & 20 min \\
\hline
``Como aumentar o valor do benefício?'' & Módulo 9 - Teses Revisionais & 15 min \\
\hline
``Tema 1300 — posso revisar AIP do cliente?'' & Módulo 9 - Teses Revisionais \xmark & 5 min \\
\hline
``Preciso de modelos de petição/recurso'' & Módulo 9 - Modelos Práticos & 5 min \\
\hline
\end{tabular}
\end{center}

%% ============================================================================
%% SUMÁRIO E LISTAS
%% ============================================================================
\hypertarget{toc}{\tableofcontents}
\clearpage
\listoffigures
\clearpage
\listoftables
\clearpage
\pagestyle{fancy}
\markboth{}{} % Limpa o cabeçalho para não vazar


%% ============================================================================
%% MÓDULO 0: INTRODUÇO AO SISTEMA CJP
%% ============================================================================
\clearpage
\chapter*{Módulo 0: Introdução ao Sistema CJP}
\addcontentsline{toc}{chapter}{Módulo 0: Introdução ao Sistema CJP}
\markboth{Módulo 0: Introdução ao Sistema CJP}{Módulo 0: Introdução ao Sistema CJP}
\setcounter{chapter}{0}

\begin{center}
{\Large\textit{``Por Que Cálculo Sozinho Não Basta (E O Que Fazer a Respeito)''}}
\end{center}

\section{A Ilusão da Automação: Por Que Softwares Falham}

Você usa software de cálculos previdenciários?

Provavelmente sim. Todo escritório de advocacia previdenciária hoje tem ao menos um. Alguns têm vários: Cálculo Jurídico, Prev Virtual, CJ Desktop, planilhas do Excel herdadas de outros colegas...

E todos eles têm algo em comum: \textbf{são ferramentas úteis, mas nunca suficientes.}

Por quê?

\subsection{O Problema das ``Caixas Pretas''}

Um software de cálculos faz exatamente o que você manda. Nada mais, nada menos.

\begin{itemize}
    \item Você digita: ``Tempo de contribuição: 35 anos''
    \item O software calcula baseado em 35 anos
    \item \textbf{Mas ele não pergunta:} ``Tem certeza que são 35? Você verificou se aquele vínculo PEXT de 1998 foi considerado? E o tempo especial que pode ser convertido em comum? E o tempo rural da infância?''
\end{itemize}

\begin{armadilha}
O software assume que seus inputs estão corretos. E quase nunca estão.

Se você alimenta o software com dados incompletos ou errados (porque não fez uma entrevista estratégica completa), o resultado será tecnicamente preciso... \textbf{mas estrategicamente inútil.}
\end{armadilha}

\subsection{Os 3 Problemas Estruturais da Automação}

\textbf{Problema \#1: Garbage In, Garbage Out}

Se você alimenta o software com dados incompletos ou errados (porque não fez uma entrevista estratégica completa), o resultado será tecnicamente preciso... \textbf{mas estrategicamente inútil.}

\textbf{Problema \#2: Uma Resposta vs A Melhor Resposta}

A maioria dos softwares te dá \textit{uma} resposta. Geralmente a primeira regra que se enquadra.

Mas e se existem 5 regras possíveis? E se a Regra de Pontos dá R\$ 3.500/mês mas o Pedágio 100\% dá R\$ 4.200/mês se o cliente esperar 18 meses?

\textbf{O software não faz planejamento estratégico. Você faz.}

\textbf{Problema \#3: Zero Auditoria Crítica}

Um software não questiona o CNIS. Ele não pergunta: ``Por que este cliente com 20 anos de carteira assinada tem apenas 15 anos no CNIS?''

\textbf{Ele simplesmente processa o que você inseriu.}

\subsection{A Solução: Validação Estratégica Humana (Método CJP)}

O Método CJP não substitui softwares. \textbf{Ele os valida.}

O sistema CJP ensina você a:

\begin{enumerate}
    \item \textbf{Fazer as perguntas certas} antes de digitar qualquer número (Pilar 1)
    \item \textbf{Detectar erros que o software não vê} (Pilares 2 e 3)
    \item \textbf{Calcular todas as possibilidades, não apenas uma} (Pilar 4)
    \item \textbf{Entregar valor que justifica honorários premium} (Pilar 5)
\end{enumerate}

\begin{conceitoChave}
\textbf{Em resumo:}
\begin{itemize}
    \item Softwares fazem \textbf{cálculos}
    \item O Método CJP faz \textbf{planejamento estratégico}
\end{itemize}

E é o planejamento que seu cliente está pagando.
\end{conceitoChave}

\section{Os 5 Pilares do Método CJP Explicados}

\begin{conceitoChave}
\textbf{Os 5 Pilares}

O Método CJP é uma ARQUITETURA de trabalho, não uma fórmula mágica. Cada pilar depende do anterior. Pular etapas = deixar dinheiro na mesa.
\end{conceitoChave}

\subsection{PILAR 1: A ENTREVISTA ESTRATÉGICA}

\textbf{O que é:}\\
Uma consulta inicial estruturada com perguntas investigativas que mapeiam a ``vida previdenciária completa'' do cliente, não apenas o que está no CNIS.

\textbf{Por que importa:}\\
90\% dos erros do INSS são \textbf{erros de omissão} (tempo que deveria constar mas não consta). Se você não descobrir na entrevista, não verá no CNIS.

\begin{novidade}
\textbf{PROCURAÇO ELETRÔNICA MEU INSS}

Com a Portaria Conjunta DIT/DIRBEN/INSS n.\textsuperscript{o} 10/2025, vigente desde 13/11/2025, você pode acessar DIRETAMENTE o CNIS e demais informações do cliente SEM solicitar senha. Isso REVOLUCIONA o processo de entrevista e diagnóstico inicial!

\textbf{ANTES:} ``Cliente, preciso da sua senha do Meu INSS...''

\textbf{DEPOIS:} ``Cliente, acesso direto via procuração eletrônica''

Ver detalhes completos no Módulo 1, Seção 1.2.
\end{novidade}

\textbf{O que você aprende:}
\begin{itemize}
    \item Roteiro de 8 perguntas-gatilho que detectam as 8 Armadilhas Ocultas
    \item Checklist de documentação inicial por tipo de segurado
    \item Como priorizar o que investigar primeiro
    \item Como utilizar a Procuração Eletrônica para acesso direto ao Meu INSS
\end{itemize}

\textbf{Resultado:}\\
Lista completa de ``pontos de atenção'' para auditar no Pilar 2.

\subsection{PILAR 2: O DIAGNÓSTICO IMPECÁVEL}

\textbf{O que é:}\\
Auditoria dupla do CNIS: (A) Diagnóstico \textbf{Explícito} via indicadores prioritários, e (B) Diagnóstico \textbf{Implícito} via detecção das 8 Armadilhas Ocultas.

\textbf{Por que importa:}\\
O CNIS do seu cliente tem, em média, 3-5 erros significativos. Cada erro corrigido pode representar R\$ 50.000 - R\$ 200.000 em ganho vitalício.

\textbf{O que você aprende:}
\begin{itemize}
    \item Módulo 2: Como decodificar cada um dos indicadores CNIS prioritários (nomenclatura atualizada pela Portaria 1.316/2025)
    \item Módulo 3: Como detectar armadilhas que não aparecem como ``indicador'' (ex: tempo rural, concomitâncias)
\end{itemize}

\textbf{Resultado:}\\
Relatório completo de erros encontrados + estimativa de impacto financeiro.

\subsection{PILAR 3: OS ACERTOS DE VÍNCULOS}

\textbf{O que é:}\\
Transformar diagnóstico em \textbf{prova documental} via Requerimento de Acerto de CNIS (RAC) ou processo judicial de reconhecimento.

\textbf{Por que importa:}\\
Diagnóstico sem execução é teoria sem prática. Este pilar te ensina a ``traduzir'' cada armadilha detectada em um protocolo de correção administrativo ou judicial.

\textbf{O que você aprende:}
\begin{itemize}
    \item Como montar um RAC que o INSS aceite (ou que o juiz não possa negar)
    \item Matriz de documentos: o que anexar para cada tipo de erro
    \item Quando ir na via administrativa vs judicial
    \item Impacto do Tema 1124 STJ na instrução do processo administrativo
\end{itemize}

\textbf{Resultado:}\\
CNIS corrigido, pronto para ser usado no Pilar 4 (cálculos).

\subsection{PILAR 4: OS CÁLCULOS SISTEMATIZADOS}

\textbf{O que é:}\\
Cálculo de \textbf{todos os cenários possíveis} (não apenas o primeiro que aparece), incluindo PBC, regras de transição, RMI final e benefícios não programáveis.

\textbf{Por que importa:}\\
Seu cliente não quer ``uma aposentadoria''. Ele quer a \textbf{melhor aposentadoria possível}. E ``melhor'' pode significar esperar 6 meses para ganhar R\$ 800/mês a mais pelo resto da vida.

\textbf{O que você aprende:}
\begin{itemize}
    \item Módulo 5: Como auditar o PBC e escolher a regra de transição mais vantajosa (atualizado para 2026: 93/103 pontos)
    \item Módulo 6: Como validar o divisor mínimo e o coeficiente da RMI
    \item Módulo 7: Como calcular benefícios não programáveis (inclui Tema 1300 — AIP 100\%)
\end{itemize}

\textbf{Resultado:}\\
Tabela comparativa com 3-5 cenários diferentes, cada um com RMI, data de aposentadoria e ganho vitalício estimado.

\subsection{PILAR 5: A ENTREGA DE VALOR}

\textbf{O que é:}\\
Transformar números brutos em um \textbf{Parecer Estratégico} (Dossiê CJP) que você apresenta pessoalmente ao cliente em uma Consulta de Entrega.

\textbf{Por que importa:}\\
Cliente não paga por cálculos. Ele paga por \textbf{clareza, segurança e plano de ação}. O Parecer é o produto final que justifica honorários premium (R\$ 1.500-3.000/caso).

\textbf{O que você aprende:}
\begin{itemize}
    \item Template estrutural do Dossiê CJP em 4 blocos
    \item Roteiro de apresentação da Consulta de Entrega
    \item Framework de precificação (3 modelos de venda)
\end{itemize}

\textbf{Resultado:}\\
Cliente impressionado, decisão clara, honorários pagos sem objeção.

\section{Glossário Essencial do Sistema CJP}

\subsection{Termos Técnicos Previdenciários}

\textbf{PBC (Período Básico de Cálculo)}\\
Período utilizado para calcular a média dos salários de contribuição. Varia conforme a regra aplicada.

\textbf{SC (Salário de Contribuição)}\\
Valor sobre o qual incide a contribuição previdenciária em cada mês.

\textbf{SB (Salário de Benefício)}\\
Média aritmética dos salários de contribuição dentro do PBC. Base para cálculo da RMI.

\textbf{RMI (Renda Mensal Inicial)}\\
Valor do primeiro pagamento do benefício. Fórmula: RMI = SB $\times$ Coeficiente.

\textbf{RMA (Renda Mensal Atual)}\\
Valor atual do benefício após reajustes anuais pelo INPC.

\textbf{DIB (Data de Início do Benefício)}\\
Data a partir da qual o benefício começa a ser pago.

\textbf{DER (Data de Entrada do Requerimento)}\\
Data em que o pedido foi protocolado no INSS.

\textbf{DDB (Data de Despacho do Benefício)}\\
Data em que o INSS concedeu (ou indeferiu) o benefício.

\textbf{Direito Adquirido}\\
Situação em que o segurado já cumpria todos os requisitos antes da Reforma (12/11/2019), podendo usar as regras antigas.

\textbf{Regras de Transição}\\
5 regras criadas pela EC 103/2019 para quem estava próximo de se aposentar mas não tinha direito adquirido.

\subsection{Termos do Método CJP}

\textbf{RAC (Requerimento de Acerto de CNIS)}\\
Pedido administrativo para corrigir erros no CNIS. Principal ferramenta do Pilar 3.

\textbf{Armadilha (CJP)}\\
Erro ou omissão no CNIS que reduz o valor do benefício e que softwares comuns não detectam. As 8 Armadilhas estão detalhadas no Módulo 3.

\textbf{Indicador CNIS}\\
Código alfanumérico que aparece no extrato CNIS sinalizando pendência, informação ou acerto. Os indicadores prioritários estão no Módulo 2.

\textbf{Parecer CJP (ou Dossiê Estratégico CJP)}\\
Documento de 4 blocos entregue ao cliente na Consulta de Entrega. Compila Pilares 1-4 em formato vendável (Módulo 8).

\textbf{Validação Estratégica Humana}\\
Processo de auditoria crítica que complementa (e valida) os cálculos automatizados, essência do Método CJP.

\subsection{Termos Novos 2025/2026}

\begin{novidade}
\textbf{Procuração Eletrônica Meu INSS}

Mecanismo digital instituído pela \textbf{Portaria Conjunta DIT/DIRBEN/INSS n.\textsuperscript{o} 10/2025}, que permite ao segurado autorizar representante (advogado) a consultar serviços digitais do INSS \textbf{sem compartilhamento de senha}. Exige conta gov.br nível \textbf{prata ou ouro} para ambas as partes. Vigente desde \textbf{13/11/2025}. Permite acesso a CNIS completo, extrato de pagamentos, carta de concessão e consultas gerais. \textbf{Não permite protocolar requerimentos ou recursos}.
\end{novidade}

\textbf{Tema 1300 STF}\\
Repercussão geral (RE 1.469.150) que analisou a constitucionalidade do coeficiente reduzido (60\%+2\%) para aposentadoria por incapacidade permanente não acidentária. \textbf{Resultado final (Dezembro/2025): 6$\times$5 pela CONSTITUCIONALIDADE} - tese rejeitada. O coeficiente de 60\%+2\% foi mantido. \textbf{NÃO há revisão disponível}.

\textbf{Tema 1124 STJ}\\
Tese sobre \textbf{interesse de agir previdenciário} (julgado em outubro/2025). Estabelece que o segurado deve apresentar \textbf{requerimento administrativo com documentação suficiente}. Define que a DIB será fixada na \textbf{DER} se as provas já estavam no processo administrativo, ou na \textbf{citação válida} se as provas surgiram apenas em juízo.

\textbf{Tema 1090 STJ}\\
Tese sobre \textbf{EPI e tempo especial} (julgado em abril/2025). Estabelece que a informação de EPI eficaz no PPP \textbf{descaracteriza o tempo especial}, EXCETO para ruído, agentes biológicos, cancerígenos e periculosidade. O \textbf{ônus de provar a ineficácia do EPI é do SEGURADO}.

\textbf{Divisor Mínimo de 108}\\
Regra da \textbf{Lei 14.331/2022} que estabelece divisor mínimo de \textbf{108 contribuições} no cálculo da média. Se houver menos de 108 contribuições no PBC, a soma é dividida por 108. Encerrou o ``milagre da contribuição única''.

\begin{armadilha}
\textbf{Revisão da Vida Toda — CANCELADA}

A tese da Revisão da Vida Toda foi \textbf{definitivamente cancelada} pelo STF em 10/04/2025 (embargos de declaração do Tema 1102) e pelas ADIs 2110 e 2111 em novembro/2025.

\textbf{Modulação de efeitos (marco 05/04/2024):} valores recebidos até esta data por decisões favoráveis não precisam ser devolvidos; honorários e custas em ações pendentes até esta data não podem ser cobrados.

\textbf{Novas ações são improcedentes.}
\end{armadilha}

\section{As 3 Grandes Eras do Cálculo Previdenciário}

\begin{conceitoChave}
Entender as ``Eras'' é essencial para aplicar a regra correta. Cliente com DIB em cada era tem cálculo diferente.
\end{conceitoChave}

\subsection{Era 1: Pré-Reforma (até 12/11/2019)}

\textbf{Regra aplicada:}\\
Regra antiga (Lei 8.213/91 com redação original + alterações até 2019)

\textbf{Características:}
\begin{itemize}
    \item Aposentadoria por tempo de contribuição: 35 anos (H) / 30 anos (M)
    \item Fator Previdenciário: Opcional para aposentadoria por tempo, obrigatório por idade
    \item PBC: Média de 80\% maiores salários desde julho/1994
    \item Divisor Mínimo: 60\% do PBC
\end{itemize}

\textbf{Quem se enquadra:}
\begin{itemize}
    \item Segurados que \textbf{já estavam aposentados} antes de 13/11/2019
    \item Segurados com \textbf{direito adquirido} (completaram requisitos antes da Reforma)
\end{itemize}

\textbf{Regras de cálculo:}
\begin{itemize}
    \item Aposentadoria por Tempo: SB $\times$ Fator Previdenciário (se vantajoso)
    \item Aposentadoria por Idade: SB $\times$ (70\% + 1\% por ano de contribuição)
\end{itemize}

\subsection{Era 2: Regras de Transição (13/11/2019 - presente)}

\textbf{Regra aplicada:}\\
Uma das 5 Regras de Transição da EC 103/2019

\textbf{Características:}
\begin{itemize}
    \item Coeficiente novo: 60\% + 2\% por ano acima de 15/20 anos
    \item Requisitos híbridos (tempo + idade + pedágio)
    \item 5 possibilidades diferentes
    \item Não existe mais ``aposentadoria por tempo puro''
\end{itemize}

\textbf{Quem se enquadra:}
\begin{itemize}
    \item Segurados que \textbf{estavam contribuindo} em 13/11/2019 mas \textbf{não tinham direito adquirido}
    \item Maioria absoluta dos casos atuais (2026)
\end{itemize}

\textbf{5 Regras (Valores 2026):}
\begin{enumerate}
    \item \textbf{Pontos (Art. 15):} Tempo + Idade = \textbf{93 pontos (M) / 103 pontos (H)} em 2026
    \item \textbf{Pedágio 50\% (Art. 17):} Faltavam até 2 anos em 2019, cumprir 50\% de pedágio
    \item \textbf{Pedágio 100\% (Art. 20):} Cumprir 100\% de pedágio + idade mínima (57M/60H)
    \item \textbf{Idade Progressiva (Art. 18):} \textbf{59 anos (M) / 64 anos (H)} em 2026
    \item \textbf{Permanente (Art. 19):} Idade fixa (62M/65H) + tempo mínimo (15 anos)
\end{enumerate}

\subsection{Era 3: Regra Permanente (futuro)}

\textbf{Regra aplicada:}\\
Art. 19, EC 103/2019 (já vigente, mas afeta principalmente quem começou a contribuir pós-Reforma)

\textbf{Características:}
\begin{itemize}
    \item Idade: 65 anos (H) / 62 anos (M)
    \item Tempo mínimo: 15 anos (H) / 15 anos (M)
    \item Coeficiente: 60\% + 2\% por ano acima de 15/20
    \item Sem opções alternativas
\end{itemize}

\textbf{Quem se enquadra:}
\begin{itemize}
    \item Segurados que começaram a contribuir \textbf{após 13/11/2019}
    \item Segurados em transição que ``desistiram'' de cumprir pedágios e preferiram esperar a idade permanente
\end{itemize}

\section{Checklist Pré-Consulta}

Use este checklist \textbf{antes} de aplicar o Método CJP:

\begin{acaoImediata}
\textbf{CHECKLIST PRÉ-CONSULTA (ATUALIZADO 2026)}

\textbf{DOCUMENTAÇO MÍNIMA OBRIGATÓRIA}
\begin{itemize}
    \item[$\square$] CNIS atualizado (máx 30 dias) OU acesso via Procuração Eletrônica
    \item[$\square$] CTPS (todas as páginas)
    \item[$\square$] RG e CPF
    \item[$\square$] Comprovante de residência
\end{itemize}

\textbf{DOCUMENTAÇO CONDICIONAL}
\begin{itemize}
    \item[$\square$] Carta de Concessão (se já aposentado)
    \item[$\square$] PPP (se trabalhou em ativ. especial) — PPP Eletrônico obrigatório desde 2023
    \item[$\square$] Certidão de Tempo de Contribuição (se foi servidor público)
    \item[$\square$] Certificado de Reservista (se fez serviço militar)
    \item[$\square$] Declaração de Tempo Rural (se trabalhou em zona rural)
\end{itemize}

\textbf{VALIDAÇÕES INICIAIS}
\begin{itemize}
    \item[$\square$] Cliente entende que o processo pode levar 60-90 dias?
    \item[$\square$] Cliente concorda com honorários?
    \item[$\square$] Expectativa está alinhada (não prometer ``resultado milagroso'')?
\end{itemize}

\textbf{VERIFICAÇÕES 2026}
\begin{itemize}
    \item[$\square$] Cliente tem conta gov.br prata/ouro? (para procuração eletrônica)
    \item[$\square$] AIP pós-2019? $\rightarrow$ Verificar Tema 1300
    \item[$\square$] Processo judicial em curso? $\rightarrow$ Tema 1124
\end{itemize}
\end{acaoImediata}

\section{Mensagem Final do Módulo 0}

Você acabou de ser apresentado à \textbf{arquitetura completa do Sistema CJP}.

Agora você sabe:
\begin{itemize}
    \item[\cmark] Por que softwares sozinhos falham
    \item[\cmark] Quais são os 5 Pilares do Método
    \item[\cmark] Como navegar este e-book por Pilar
    \item[\cmark] O glossário essencial para entender o sistema (incluindo termos 2025/2026)
    \item[\cmark] As 3 Eras do cálculo previdenciário
\end{itemize}

\textbf{Novidades que você encontrará nesta edição:}
\begin{itemize}
    \item \textbf{Procuração Eletrônica Meu INSS} — Acesso direto sem solicitar senha
    \item \textbf{Tema 1300 STF} — Revisão de AIP para 100\% (REJEITADA)
    \item \textbf{Tema 1124 STJ} — DIB na DER se provas estavam no PA
    \item \textbf{Tema 1090 STJ} — Ônus de provar ineficácia do EPI é do segurado
    \item \textbf{Regras de Transição 2026} — 93/103 pontos
    \item[\xmark] \textbf{Revisão da Vida Toda} — Definitivamente cancelada
\end{itemize}

\textbf{Próximo passo:}

Se você vai \textbf{aplicar o método em um caso real agora}, vá para o \textbf{Módulo 1} (Pilar 1: Entrevista).

Se você está \textbf{estudando para dominar o sistema}, continue para o \textbf{Módulo 1} e siga a sequência linear até o Módulo 9.

\begin{conceitoChave}
\textbf{Lembre-se:} O Sistema CJP não é uma fórmula mágica. É uma \textbf{disciplina de trabalho} que, quando aplicada consistentemente, transforma advogados comuns em especialistas temidos.

\textbf{Vamos começar.}
\end{conceitoChave}

\vfill

\begin{center}
\textit{``Planejamento não é sobre prever o futuro. É sobre estar preparado para qualquer futuro que vier.''}

\textbf{— Sistema CJP, 2026}
\end{center}

\vspace{1cm}

%% Continua na Parte 1B

%% ============================================================================
%% INFOGRÁFICO DO MÓDULO 0
%% ============================================================================
\clearpage
\backtotoc

\section*{\faImage\ Infográfico de Consolidação}

\begin{figure}[H]
    \centering
    \begin{tcolorbox}[colback=white, colframe=cjpAzulEscuro, title={\textbf{\faBookOpen\ Infográfico: Módulo 0 --- Introdução ao Sistema CJP}}, fonttitle=\bfseries\color{white}, sharp corners=downhill, boxrule=2pt]
        \centering
        \includegraphics[width=0.95\textwidth, keepaspectratio]{modulo0}
    \end{tcolorbox}
    \caption{Resumo Visual do Módulo 0: Introdução ao Sistema CJP}
    \label{fig:modulo0}
\end{figure}
