
% ============================================================================
% MÓDULO 9 - PARTE B: MODELOS, QUESITOS E FINALIZAÇÃO
% ============================================================================

\section{Modelo: Pedido de Aposentadoria Programada}

\textbf{QUANDO USAR:}
\begin{itemize}
    \item Após correção do CNIS (RAC deferido)
    \item Cliente completou requisitos para aposentadoria
    \item Após planejamento completo (Módulos 1-8)
\end{itemize}

\textbf{TIPOS DE APOSENTADORIA:}
\begin{itemize}
    \item Por Idade
    \item Por Tempo de Contribuição (Regras de Transição)
    \item Especial
\end{itemize}

\begin{verbatim}
REQUERIMENTO DE CONCESSÃO DE BENEFÍCIO PREVIDENCIÁRIO
APOSENTADORIA {POR IDADE | POR TEMPO DE CONTRIBUIÇÃO | ESPECIAL}

REQUERENTE: [NOME COMPLETO DO CLIENTE]
CPF: [XXX.XXX.XXX-XX]
RG: [XX.XXX.XXX-X]
Data de Nascimento: [DD/MM/AAAA]
Idade atual: [XX] anos
Sexo: {Masculino | Feminino}
Telefone: [XX] XXXXX-XXXX
E-mail: [email@exemplo.com]
Endereço: [Rua/Av], [Número], [Bairro], [Cidade]-[UF], CEP [XXXXX-XXX]

ADVOGADO(A): [SEU NOME COMPLETO]
OAB/[UF]: [NÚMERO]

---

ASSUNTO: REQUERIMENTO DE CONCESSÃO DE APOSENTADORIA {POR IDADE | POR TEMPO 
DE CONTRIBUIÇÃO | ESPECIAL} - {REGRA PERMANENTE | REGRA DE TRANSIÇÃO [ESPECIFICAR]}

---

I - DOS FATOS

O(A) Requerente acima qualificado(a) vem, por meio de seu(sua) advogado(a), 
requerer a este Instituto a concessão de Aposentadoria {POR IDADE | POR TEMPO 
DE CONTRIBUIÇÃO | ESPECIAL}, pela {REGRA PERMANENTE | REGRA DE TRANSIÇÃO 
[NOME]}, nos termos da legislação previdenciária vigente.

I.1 - DO HISTÓRICO CONTRIBUTIVO

Conforme extrato do CNIS atualizado (Anexo I), o(a) Requerente possui o 
seguinte histórico contributivo:

RESUMO DO TEMPO DE CONTRIBUIÇÃO:
• Tempo Total de Contribuição: [XX] anos, [XX] meses e [XX] dias
• Tempo de Contribuição até 12/11/2019: [XX] anos, [XX] meses e [XX] dias
• Tempo de Contribuição após 13/11/2019: [XX] anos, [XX] meses e [XX] dias
• Carência: [XXX] contribuições mensais

(SE HOUVER TEMPO ESPECIAL)
• Tempo Especial (antes de 13/11/2019): [XX] anos, [XX] meses
• Tempo Especial convertido em comum: [XX] anos, [XX] meses

(SE HOUVER TEMPO RURAL)
• Tempo Rural: [XX] anos, [XX] meses

I.2 - DO CUMPRIMENTO DOS REQUISITOS

O(A) Requerente cumpre INTEGRALMENTE os requisitos para a concessão da 
aposentadoria requerida, conforme demonstração abaixo:

[ESCOLHA A REGRA APLICÁVEL E PREENCHA:]

---

OPÇÃO A - APOSENTADORIA POR IDADE (REGRA PERMANENTE - EC 103/2019)

Requisitos (Art. 19, EC 103/2019):
✅ Idade mínima: 65 anos (homem) / 62 anos (mulher)
   Idade do(a) Requerente: [XX] anos (data de nascimento: [DD/MM/AAAA])

✅ Tempo mínimo de contribuição: 15 anos (mulher) / 20 anos (homem)
   Tempo do(a) Requerente: [XX] anos, [XX] meses

✅ Carência mínima: 180 contribuições mensais
   Carência do(a) Requerente: [XXX] contribuições

---

OPÇÃO B - APOSENTADORIA POR TEMPO DE CONTRIBUIÇÃO - REGRA DE PONTOS 
(Art. 15, EC 103/2019)

Requisitos:
✅ Tempo mínimo de contribuição: 35 anos (homem) / 30 anos (mulher)
   Tempo do(a) Requerente: [XX] anos, [XX] meses

✅ Pontuação mínima: [XXX] pontos em [ANO ATUAL]
   (Tabela progressiva: 2024 = 100H/90M; 2025 = 101H/91M; etc.)
   
   Pontuação do(a) Requerente:
   • Idade: [XX] anos = [XX] pontos
   • Tempo de contribuição: [XX] anos = [XX] pontos
   • TOTAL: [XXX] PONTOS ✅

✅ Carência: 180 contribuições mensais
   Carência do(a) Requerente: [XXX] contribuições

---

OPÇÃO C - APOSENTADORIA POR TEMPO DE CONTRIBUIÇÃO - REGRA DO PEDÁGIO 50% 
(Art. 17, EC 103/2019)

Requisitos (cumulativos):
✅ Faltavam no máximo 2 anos para completar 35/30 anos em 13/11/2019
   Situação do(a) Requerente em 13/11/2019: [XX] anos, [XX] meses
   Faltavam: [X] anos, [XX] meses ✅

✅ Cumprimento do pedágio de 50% do tempo faltante
   Tempo faltante em 13/11/2019: [X] anos, [XX] meses = [XX] meses
   Pedágio 50%: [XX] meses $\times$ 0,5 = [XX] meses
   Tempo trabalhado após 13/11/2019: [XX] meses ✅

✅ Tempo total mínimo: 35 anos (homem) / 30 anos (mulher)
   Tempo total do(a) Requerente: [XX] anos, [XX] meses ✅

✅ Carência: 180 contribuições
   Carência do(a) Requerente: [XXX] contribuições ✅

---

OPÇÃO D - APOSENTADORIA POR TEMPO DE CONTRIBUIÇÃO - REGRA DO PEDÁGIO 100% 
(Art. 20, EC 103/2019)

Requisitos (cumulativos):
✅ Idade mínima: 60 anos (homem) / 57 anos (mulher)
   Idade do(a) Requerente: [XX] anos ✅

✅ Tempo mínimo: 35 anos (homem) / 30 anos (mulher)
   Tempo do(a) Requerente: [XX] anos, [XX] meses ✅

✅ Cumprimento do pedágio de 100% do tempo faltante em 13/11/2019
   Tempo faltante em 13/11/2019: [X] anos, [XX] meses = [XX] meses
   Pedágio 100%: [XX] meses $\times$ 1,0 = [XX] meses
   Tempo trabalhado após 13/11/2019: [XX] meses ✅

✅ Carência: 180 contribuições
   Carência do(a) Requerente: [XXX] contribuições ✅

---

OPÇÃO E - APOSENTADORIA POR TEMPO DE CONTRIBUIÇÃO - REGRA PERMANENTE 
(Art. 19, \S 1\textsuperscript{o}, EC 103/2019)

Requisitos (cumulativos):
✅ Idade mínima: 65 anos (homem) / 62 anos (mulher)
   Idade do(a) Requerente: [XX] anos ✅

✅ Tempo mínimo: 20 anos (homem) / 15 anos (mulher)
   Tempo do(a) Requerente: [XX] anos, [XX] meses ✅

✅ Carência: 180 contribuições
   Carência do(a) Requerente: [XXX] contribuições ✅

---

OPÇÃO F - APOSENTADORIA ESPECIAL (Art. 19, EC 103/2019)

Requisitos (cumulativos):
✅ Tempo especial mínimo: {15 | 20 | 25} anos de exposição
   Tempo especial do(a) Requerente: [XX] anos, [XX] meses ✅

✅ Idade mínima: {55 | 58 | 60} anos (conforme tempo especial)
   Idade do(a) Requerente: [XX] anos ✅

✅ Carência: 180 contribuições
   Carência do(a) Requerente: [XXX] contribuições ✅

✅ Comprovação de exposição a agentes nocivos:
   PPP anexo (Anexo [X]) comprova exposição a {AGENTE NOCIVO} no período 
   de [DD/MM/AAAA] a [DD/MM/AAAA].

---

II - DO DIREITO

II.1 - DA BASE LEGAL DA APOSENTADORIA REQUERIDA

[ESCOLHA CONFORME A REGRA:]

• Lei n\textsuperscript{o} 8.213/91 (Lei de Benefícios da Previdência Social)
• Emenda Constitucional n\textsuperscript{o} 103/2019 (Reforma da Previdência)
• Art. {ARTIGO ESPECÍFICO} da EC 103/2019
• Instrução Normativa PRES/INSS n\textsuperscript{o} 128/2022

II.2 - DO CÁLCULO DA RENDA MENSAL INICIAL (RMI)

Conforme planilha de cálculo anexa (Anexo [X]), elaborada nos termos dos 
arts. 26, 29 e 29-A da Lei n\textsuperscript{o} 8.213/91, com as alterações da EC 103/2019, 
a Renda Mensal Inicial (RMI) do benefício será de:

CÁLCULO DETALHADO:

1. PERÍODO BÁSICO DE CÁLCULO (PBC):
   Média aritmética simples dos 80% maiores salários de contribuição desde 
   julho de 1994 (ou desde a filiação, se posterior).
   
   Total de contribuições desde 07/1994: [XXX] meses
   80% das maiores: [XXX] meses
   Soma dos salários (atualizados): R$ [XXXXXX,XX]
   Divisor: {108 meses mínimo | [XXX] meses real}
   
   PBC = R$ [XXXXXX,XX] $\div$ [XXX] = R$ [XXXX,XX]

2. COEFICIENTE DE CÁLCULO:
   [ESCOLHA CONFORME A REGRA:]
   
   PARA REGRA PERMANENTE/PONTOS/PEDÁGIO 100%:
   • Base: 60% + 2% por ano acima de {20 anos (homem) / 15 anos (mulher)}
   • Tempo do(a) Requerente: [XX] anos
   • Tempo excedente: [XX] anos
   • Coeficiente: 60% + ([XX] $\times$ 2%) = [XX]%
   
   OU
   
   PARA PEDÁGIO 50%:
   • 100% do PBC (sem redutor) - Art. 17, \S 2\textsuperscript{o}, EC 103/2019
   • Coeficiente: 100%
   
   OU
   
   PARA APOSENTADORIA ESPECIAL:
   • 60% + 2% por ano acima de {15 | 20} anos
   • Tempo especial: [XX] anos
   • Coeficiente: 60% + ([XX] $\times$ 2%) = [XX]%

3. RENDA MENSAL INICIAL (RMI):
   RMI = PBC $\times$ Coeficiente
   RMI = R$ [XXXX,XX] $\times$ [XX]% = R$ [XXXX,XX]

(Se RMI calculada estiver acima do teto)
Aplicação do teto constitucional: R$ [TETO VIGENTE]
RMI FINAL: R$ [XXXX,XX]

III - DOS PEDIDOS

Diante do exposto, REQUER-SE a Vossa Senhoria que:

a) RECEBA e PROCESSE o presente Requerimento de Aposentadoria {TIPO};

b) CONCEDA o benefício de Aposentadoria {TIPO} ao(à) Requerente, com Data 
   de Início do Benefício (DIB) em [DD/MM/AAAA], nos termos da {REGRA 
   APLICÁVEL};

c) CALCULE a Renda Mensal Inicial (RMI) conforme demonstrado na planilha 
   anexa, no valor estimado de R$ [XXXX,XX];

d) INICIE o pagamento do benefício no prazo legal de 45 dias, conforme 
   art. 101, II, da Lei n\textsuperscript{o} 8.213/91;

e) (OPCIONAL) DEFIRA a implantação do benefício mesmo havendo pendências 
   documentais sanáveis, nos termos do art. 124-A da IN 128/2022, 
   evitando-se prejuízo ao segurado.

Termos em que,
Pede Deferimento.

[CIDADE]-[UF], [DD] de [MÊS] de [ANO].

___________________________________
[NOME DO(A) ADVOGADO(A)]
OAB/[UF] [NÚMERO]

___________________________________
[NOME DO(A) REQUERENTE]
CPF: [XXX.XXX.XXX-XX]
\end{verbatim}

\section{Modelo: Pedido de Revisão de Benefício Concedido}

\textbf{QUANDO USAR:}
\begin{itemize}
    \item Benefício foi concedido com erro de cálculo
    \item Nova legislação beneficia o segurado (ex: Tema 1070, Lei 15.108/2025)
    \item CNIS foi corrigido após concessão (RAC deferido posterior)
\end{itemize}

\begin{verbatim}
REQUERIMENTO DE REVISÃO DE BENEFÍCIO PREVIDENCIÁRIO

REQUERENTE: [NOME COMPLETO DO CLIENTE]
CPF: [XXX.XXX.XXX-XX]
Número do Benefício (NB): [XXX.XXX.XXX-X]
Espécie do Benefício: [CÓDIGO] - {APOSENTADORIA | PENSÃO | AUXÍLIO}
Data de Início do Benefício (DIB): [DD/MM/AAAA]
Renda Mensal Inicial (RMI) original: R$ [XXXX,XX]
Renda Mensal Atual (RMA): R$ [XXXX,XX]

ADVOGADO(A): [SEU NOME COMPLETO]
OAB/[UF]: [NÚMERO]

---

ASSUNTO: REQUERIMENTO DE REVISÃO DE BENEFÍCIO - {TEMA 1070 STJ | TEMA 555 
STJ | TEMA 1152 STJ | LEI 15.108/2025 | ERRO MATERIAL NO CÁLCULO}

---

I - DOS FATOS

O(A) Requerente, titular do benefício NB [XXX.XXX.XXX-X], concedido em 
[DD/MM/AAAA], vem, por meio de seu(sua) advogado(a), requerer a REVISÃO 
do benefício, nos termos do art. 103 da Lei n\textsuperscript{o} 8.213/91, pelas razões de 
fato e de direito que passa a expor.

I.1 - DO BENEFÍCIO CONCEDIDO

O benefício foi concedido com os seguintes parâmetros (conforme Carta de 
Concessão anexa - Anexo I):

• Data de Início do Benefício (DIB): [DD/MM/AAAA]
• Renda Mensal Inicial (RMI): R$ [XXXX,XX]
• Período Básico de Cálculo (PBC): R$ [XXXX,XX]
• Coeficiente aplicado: [XX]%
• Tempo de contribuição computado: [XX] anos, [XX] meses

I.2 - DO ERRO IDENTIFICADO

Após análise técnica detalhada do processo de concessão, identificou-se 
{ERRO MATERIAL NO CÁLCULO | APLICAÇÃO INCORRETA DA LEGISLAÇÃO}, 
conforme descrito abaixo:

[ESCOLHA E PREENCHA CONFORME O CASO:]

---

OPÇÃO A - TEMA 1070 STJ (ATIVIDADES CONCOMITANTES)

ERRO IDENTIFICADO:
O INSS calculou o Salário de Benefício aplicando apenas percentual sobre 
a atividade secundária, em desacordo com o Tema 1070 do STJ.

FUNDAMENTAÇÃO:
Conforme decidido em definitivo pelo Superior Tribunal de Justiça no 
julgamento do Tema 1070, devem ser somados 100% dos salários de contribuição 
das atividades concomitantes para benefícios com DIB a partir de 29/11/1999.

DEMONSTRAÇÃO DO ERRO:
CÁLCULO INCORRETO DO INSS:
Salário da atividade principal: R$ [XXXX,XX]
+ Percentual da atividade secundária (80%): R$ [XXXX,XX] $\times$ 0,8 = R$ [XXXX,XX]
= Soma computada pelo INSS: R$ [XXXX,XX]

CÁLCULO CORRETO (TEMA 1070):
Salário da atividade principal: R$ [XXXX,XX]
+ Salário da atividade secundária (100%): R$ [XXXX,XX]
= Soma correta: R$ [XXXX,XX]

DIFERENÇA: R$ [XXXX,XX] por mês

---

II - DOS PEDIDOS

Diante do exposto, REQUER-SE a Vossa Senhoria que:

a) CONHEÇA e PROCESSE o presente pedido de REVISÃO;

b) RECALCULE a Renda Mensal Inicial (RMI) do benefício NB [XXX.XXX.XXX-X], 
   corrigindo o erro apontado;

c) PAGUE as diferenças apuradas desde a Data de Início do Benefício (DIB), 
   respeitada a prescrição quinquenal, devidamente atualizadas;

d) IMPLANTE a nova Renda Mensal Atual (RMA) corrigida.

Termos em que,
Pede Deferimento.

[CIDADE]-[UF], [DD] de [MÊS] de [ANO].

___________________________________
[NOME DO(A) ADVOGADO(A)]
OAB/[UF] [NÚMERO]
\end{verbatim}

\section{Quesitos Periciais Estratégicos}

\subsection{Quesitos para Exposição a Ruído}

\begin{acaoImediata}
\textbf{QUESITOS PARA PERÍCIA --- RUÍDO}

\begin{enumerate}
    \item Quais eram as funções executadas pelo autor no período de [DATA] a [DATA]?
    \item O autor era exposto a ruído ocupacional? Em qual intensidade (dB)?
    \item O nível de ruído ultrapassava o limite de tolerância vigente à época?
    \item Quais EPIs eram fornecidos? Eram suficientes para neutralizar o agente?
    \item Havia Programa de Prevenção de Riscos Ambientais (PPRA)?
    \item O ambiente de trabalho possuía Laudo Técnico (LTCAT)?
    \item A exposição era habitual e permanente?
    \item Havia medidas de proteção coletiva?
    \item O ruído era contínuo, intermitente ou de impacto?
    \item A atividade se enquadra como especial nos termos do Decreto 53.831/64 ou Decreto 3.048/99?
\end{enumerate}
\end{acaoImediata}

\subsection{Quesitos para Agentes Químicos}

\begin{acaoImediata}
\textbf{QUESITOS PARA PERÍCIA --- AGENTES QUÍMICOS}

\begin{enumerate}
    \item Quais agentes químicos estavam presentes no ambiente de trabalho?
    \item Havia manipulação direta de substâncias químicas?
    \item Quais as concentrações dos agentes no ambiente?
    \item Ultrapassavam os limites de tolerância das NRs?
    \item O contato era por via cutânea, respiratória ou digestiva?
    \item Havia monitoramento biológico?
    \item Quais EPIs eram fornecidos e utilizados?
    \item A exposição era habitual, permanente e não ocasional?
    \item Os agentes são classificados como cancerígenos pelo IARC?
    \item A atividade se enquadra nos Anexos do Decreto 3.048/99?
\end{enumerate}
\end{acaoImediata}

\subsection{Quesitos para Temperatura Extrema (Calor)}

\begin{acaoImediata}
\textbf{QUESITOS PARA PERÍCIA --- EXPOSIÇÃO A CALOR}

\textbf{Aplicação:} Forneiros, fundidores, metalúrgicos. Limite: 26,7$^\circ$C (IBUTG).

\begin{enumerate}
    \item Confirme a função exercida pelo(a) Autor(a) e o período trabalhado.
    \item Havia fonte de calor no ambiente (fornos, caldeiras, fundição)?
    \item Qual a temperatura média (IBUTG --- Índice de Bulbo Úmido Termômetro de Globo) no ambiente de trabalho?
    \item A temperatura ultrapassava 26,7$^\circ$C (limite da NR-15, Anexo 3)?
    \item A exposição ao calor era permanente durante a jornada?
    \item Havia medidas de controle (ventilação, pausas, rotatividade)?
    \item As medidas adotadas eram EFICAZES para reduzir a temperatura a níveis toleráveis?
    \item O período caracteriza-se como atividade especial (Decreto 3.048/99, Anexo IV, código 1.1.3)?
    \item \textbf{(CONCLUSÃO)} O período deve ser reconhecido como tempo especial?
\end{enumerate}
\end{acaoImediata}

\subsection{Quesitos para Atividade Insalubre (Categoria Profissional)}

\begin{acaoImediata}
\textbf{QUESITOS PARA PERÍCIA --- ENQUADRAMENTO POR CATEGORIA}

\textbf{Aplicação:} Eletricista (alta tensão), vigilante armado, mineiro. Válido até 28/04/1995 (presunção legal).

\begin{enumerate}
    \item Confirme que o(a) Autor(a) exerceu a função de [ELETRICISTA | VIGILANTE | MINEIRO | OUTRA] no período de [DATA] a [DATA].
    \item A função efetivamente exercida corresponde à categoria profissional prevista no Decreto 53.831/64 ou Decreto 83.080/79?
    \item As atividades desenvolvidas pelo(a) Autor(a) eram compatíveis com a descrição da categoria profissional prevista na legislação?
    \item (PARA ELETRICISTA) O(a) Autor(a) trabalhava com sistemas elétricos de potência superior a 250 volts (alta tensão)?
    \item (PARA VIGILANTE) O(a) Autor(a) portava arma de fogo durante o exercício da função?
    \item (PARA MINEIRO) O(a) Autor(a) exercia atividades no subsolo (interior de minas)?
    \item \textbf{(CONCLUSÃO)} O período de [DATA] a [DATA] enquadra-se como atividade especial por categoria profissional?
\end{enumerate}
\end{acaoImediata}

\section{Checklist Master Unificado}

\textbf{O Checklist que integra TODOS os 5 Pilares do Método CJP.}
Use ANTES de protocolar ou entregar parecer.

\subsection{Pilar 1: Entrevista Estratégica}
\begin{enumerate}
    \item[\faSquare] \textbf{Dados Cadastrais:} Nome, CPF, RG, Nascimento, Estado Civil, Endereço, Contato, Profissão.
    \item[\faSquare] \textbf{Histórico Profissional:} Lista cronológica de TODOS os vínculos (início/fim/função/tipo).
    \item[\faSquare] \textbf{Gatilhos das 8 Armadilhas:}
    \begin{itemize}
        \item[\faSquare] Tempo rural na infância?
        \item[\faSquare] Serviço militar?
        \item[\faSquare] Função pública (CTC)?
        \item[\faSquare] Exposição a agentes nocivos?
        \item[\faSquare] Atividades concomitantes?
        \item[\faSquare] Contribuições como autônomo?
        \item[\faSquare] Afastamentos/Benefícios?
        \item[\faSquare] Empresas extintas?
    \end{itemize}
    \item[\faSquare] \textbf{Documentação Inicial:} CTPS, CNIS, FGTS, GPS, PPP, Certidões.
    \item[\faSquare] \textbf{Expectativas:} Aposentar rápido ou valor maior? Orçamento disponível.
\end{enumerate}

\subsection{Pilar 2: Diagnóstico Impecável}
\begin{enumerate}
    \item[\faSquare] \textbf{Auditoria CNIS:} Extrato recente.
    \item[\faSquare] \textbf{15 Indicadores Verificados:} PEXT, IEAN, PSC-MEN-SM, AUX-DT, EXTEMP, VINC-NC, CONCOM, CT-DES, DEC-ADM, SEM-REM, RURAL, ESP-IMP, MIL-CTC, PENS-REC, GAP-TMP.
    \item[\faSquare] \textbf{Quantificação:} Tempo ANTES x DEPOIS das correções. Impacto na RMI.
\end{enumerate}

\subsection{Pilar 3: Acertos e Documentação}
\begin{enumerate}
    \item[\faSquare] \textbf{RAC Protocolado:} Se necessário (PEXT, Vínculos não computados).
    \item[\faSquare] \textbf{Tempo Especial:} PPPs de todos os períodos obtidos e conferidos (ART, Agentes).
    \item[\faSquare] \textbf{Tempo Rural:} Certidões, histórico escolar, declaração sindicato.
    \item[\faSquare] \textbf{Complementação:} GPS emitidas e pagas (se abaixo do mínimo).
    \item[\faSquare] \textbf{Tríade Probatória:} Prova plena ou início de prova material + testemunhal.
\end{enumerate}

\subsection{Pilar 4: Cálculos Sistematizados}
\begin{enumerate}
    \item[\faSquare] \textbf{PBC:} Salários desde 07/94, 80\% maiores, divisor mínimo (conforme época).
    \item[\faSquare] \textbf{Regra Definida:} Requisitos cumpridos (Idade, Tempo, Pontos, Pedágio, Carência).
    \item[\faSquare] \textbf{RMI Calculada:} Coeficiente correto aplicado. Teto respeitado.
    \item[\faSquare] \textbf{Cenários:} Rápido, Equilibrado, Otimizado.
    \item[\faSquare] \textbf{Planilhas:} PBC, RMI, Comparativo.
\end{enumerate}

\subsection{Pilar 5: Parecer e Consultoria}
\begin{enumerate}
    \item[\faSquare] \textbf{Dossiê Estratégico:} Capa, Diagnóstico, Correções, Cenários, Recomendação.
    \item[\faSquare] \textbf{Entrega:} Reunião agendada, apresentação ensaiada.
    \item[\faSquare] \textbf{Precificação:} Modelo definido, Contrato elaborado, ROI demonstrado.
\end{enumerate}

\subsection{Checklist Pré-Protocolo (Validação Final)}
\begin{enumerate}
    \item[\faSquare] \textbf{Documentação:} Legível, numerada, índice, cópias autênticas.
    \item[\faSquare] \textbf{Cálculos:} Conferidos por segunda pessoa.
    \item[\faSquare] \textbf{Requisitos:} Todos marcados como CUMPRIDOS.
    \item[\faSquare] \textbf{Cliente:} Informado, autorizou, assinou contrato/procuração.
    \item[\faSquare] \textbf{Protocolo:} Via definida (Meu INSS/Presencial), data agendada.
\end{enumerate}

\section{Glossário Técnico Completo A-Z}
\textbf{O Dicionário Definitivo do Previdenciarista.}

\begin{description}
    \item[ADI] Ação Direta de Inconstitucionalidade. Ex: ADI 6.309 (conversão especial pós-reforma).
    \item[Agente Nocivo] Elemento prejudicial à saúde (ruído, calor, químico).
    \item[Aposentadoria Especial] Benefício para 15/20/25 anos de exposição a agentes nocivos.
    \item[Aposentadoria por Idade] Regra: 62 anos (M) / 65 anos (H) + 15/20 anos de contribuição.
    \item[ART] Anotação de Responsabilidade Técnica (essencial para LTCAT).
    \item[Atividades Concomitantes] Dois ou mais trabalhos simultâneos. Ver Tema 1070 STJ (soma integral).
    \item[Carência] Número mínimo de meses pagos exigidos (ex: 180 meses).
    \item[CNIS] Cadastro Nacional de Informações Sociais. A "vida" do segurado no INSS.
    \item[CTC] Certidão de Tempo de Contribuição (para levar tempo público ao INSS).
    \item[DER] Data de Entrada do Requerimento. Fixa os direitos na data do protocolo.
    \item[DIB] Data de Início do Benefício. A data que começa a pagar (pode ser igual à DER).
    \item[Divisor Mínimo] Divisor fixo no cálculo do PBC. Atualmente 108. (Vácuo 11/2019 a 05/2022).
    \item[EC 103/2019] Reforma da Previdência (13/11/2019). Mudou quase tudo.
    \item[EPI] Equipamento de Proteção Individual. Se eficaz, pode afastar especialidade (exceto ruído).
    \item[Fator Previdenciário] Fórmula antiga (redutor). Ainda vale em direito adquirido ou regra de pedágio 50\%.
    \item[GPS] Guia da Previdência Social. Para pagar autônomo/facultativo.
    \item[LTCAT] Laudo Técnico das Condições Ambientais de Trabalho. Prova técnica da especialidade.
    \item[MEI] Microempreendedor Individual (5\%). Não conta para aposentadoria por tempo, só idade.
    \item[PBC] Período Básico de Cálculo. Todos os salários de 07/1994 até a DER.
    \item[PPP] Perfil Profissiográfico Previdenciário. Formulário padrão para prova de tempo especial.
    \item[RAC] Requerimento de Acerto de CNIS. O pedido para corrigir erros no cadastro.
    \item[RMI] Renda Mensal Inicial. O valor do primeiro pagamento.
    \item[Salário de Benefício (SB)] Valor base calculado a partir do PBC.
    \item[Tema Repetitivo] Decisão do STJ/STF que vale para todos (ex: Tema 1070).
    \item[Tempo Especial] Trabalho com risco/nocividade.
    \item[Teto INSS] Valor máximo pago.
    \item[Vácuo Legislativo] Período sem lei específica (ex: divisor mínimo pós-reforma).
\end{description}

\section{Conclusão do Sistema CJP}

\begin{estrategiaCJP}
\textbf{VOCÊ COMPLETOU O SISTEMA CJP!}

\textbf{O que você domina agora:}
\begin{itemize}
    \item[\cmark] Entrevista Estratégica com 8 gatilhos
    \item[\cmark] Auditoria CNIS com 15 indicadores
    \item[\cmark] Detecção de 8 armadilhas ocultas
    \item[\cmark] Acertos de vínculos com RAC
    \item[\cmark] Cálculos das 5 transições EC 103/2019
    \item[\cmark] Auditoria completa de RMI
    \item[\cmark] Benefícios não programáveis
    \item[\cmark] Parecer comercial premium
    \item[\cmark] Teses revisionais 2025-2026
\end{itemize}

\textbf{Seu diferencial competitivo:}

Enquanto 95\% dos advogados previdenciários ``rodam software'', você AUDITA, VALIDA e ENTREGA VALOR.

Isso vale R\$ 1.500-15.000 por cliente, não R\$ 500.

\textbf{Próximo passo:}

Aplique o Sistema CJP no seu PRÓXIMO cliente e veja a diferença na sua receita e na satisfação do cliente.
\end{estrategiaCJP}

\vfill

\begin{center}
\textbf{FIM DO SISTEMA CJP COMPLETO}\\[0.5cm]
\textit{``A Maestria em Planejamento Previdenciário''}\\[1cm]
\textbf{Dr. Jones Weslley Bueno Diniz}\\
OAB/SP 377.329\\
Método CJP | Cálculos Jurídicos Precisos\\[0.5cm]
\textbf{Edição Janeiro 2026}
\end{center}

%% ============================================================================
%% INFOGRÁFICO DO MÓDULO 9
%% ============================================================================
\clearpage
\backtotoc

\section*{\faImage\ Infográfico de Consolidação}

\begin{figure}[H]
    \centering
    \begin{tcolorbox}[colback=white, colframe=cjpAzulEscuro, title={\textbf{\faBookOpen\ Infográfico: Módulo 9 --- Teses Revisionais}}, fonttitle=\bfseries\color{white}, sharp corners=downhill, boxrule=2pt]
        \centering
        \includegraphics[width=0.95\textwidth, keepaspectratio]{modulo9}
    \end{tcolorbox}
    \caption{Resumo Visual do Módulo 9: Teses Revisionais}
    \label{fig:modulo9}
\end{figure}
