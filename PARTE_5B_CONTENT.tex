%% Continuação do Módulo 5: PBC e Regras de Transição

\begin{center}
{\Large\textit{Cada regra é um caminho. O advogado CJP escolhe o melhor.}}\\[0.5cm]
\textbf{Sistema CJP | Módulo 5}
\end{center}

\begin{conceitoChave}
\textbf{Por Que 5 Regras?}

A EC 103/2019 \textbf{acabou} com a aposentadoria por tempo de contribuição ``pura'' (35H/30M).

Mas criou \textbf{5 caminhos alternativos} para quem já estava contribuindo em 13/11/2019.

\textbf{Qual o melhor?}

Depende do perfil do cliente. Por isso você precisa dominar todas as 5.
\end{conceitoChave}

\begin{conceitoChave}
\textbf{TRANSIÇÃO}

As regras de transição são ``pontes'' entre a regra antiga e a regra nova.

Quem já estava contribuindo em 2019 pode escolher uma dessas 5 pontes.

Quem começou a contribuir DEPOIS de 2019 só tem a Regra Permanente.
\end{conceitoChave}

\section{Regra 1: Regra dos Pontos (Art. 15, EC 103/2019)}

\subsection{Como Funciona}

Somar \textbf{Idade + Tempo de Contribuição} até atingir uma pontuação mínima.

\subsection{Requisitos por Ano}

\begin{table}[H]
\centering
\caption{Pontuação da Regra dos Pontos por Ano (2019-2033)}
\begin{tabular}{|c|c|c|}
\hline
\textbf{Ano} & \textbf{Homens} & \textbf{Mulheres} \\
\hline
2019 & 96 pontos & 86 pontos \\
\hline
2020 & 97 pontos & 87 pontos \\
\hline
2021 & 98 pontos & 88 pontos \\
\hline
2022 & 99 pontos & 89 pontos \\
\hline
2023 & 100 pontos & 90 pontos \\
\hline
2024 & 101 pontos & 91 pontos \\
\hline
2025 & 102 pontos & 92 pontos \\
\hline
\rowcolor{yellow!30} \textbf{2026} & \textbf{103 pontos} & \textbf{93 pontos} \\
\hline
2027 & 104 pontos & 94 pontos \\
\hline
2028 (Teto H) & 105 pontos & 95 pontos \\
\hline
2033 (Teto M) & 105 pontos & 100 pontos \\
\hline
\end{tabular}
\end{table}

\textbf{Tempo mínimo obrigatório:}
\begin{itemize}
    \item Homem: 35 anos
    \item Mulher: 30 anos
\end{itemize}

\subsection{Cálculo da RMI}

$$RMI = SB \times Coeficiente$$

$$Coeficiente = 60\% + 2\% \times (anos - 20H/15M)$$

Máximo: 100\%

\subsection{Exemplo Prático (Valores 2026)}

\textbf{Cliente:} Maria, 58 anos, 33 anos de contribuição

\textbf{Cálculo:}
$$58 (idade) + 33 (tempo) = 91\ pontos$$

\textbf{Situação em 2026:}
\begin{itemize}
    \item Pontuação necessária = 93 pontos
    \item Faltam 2 pontos
\end{itemize}

\textbf{Estratégia CJP:}

Maria pode:
\begin{enumerate}
    \item Trabalhar mais 2 anos (chegar aos 60 anos com 35 de tempo = 95 pontos) \cmark
    \item Trabalhar mais 1 ano (chegar aos 59 com 34 = 93 pontos) \cmark
\end{enumerate}

\textbf{RMI (quando atingir 93 pontos em 2026):}

\begin{verbatim}
SB = R$ 6.000,00 (exemplo)
Coeficiente = 60% + 2% × (34 - 15) = 60% + 38% = 98%
RMI = R$ 6.000 × 0,98 = R$ 5.880,00
\end{verbatim}

\begin{estrategiaCJP}
\textbf{PONTOS --- Perfil Ideal}

\textbf{Vantagens:}
\begin{itemize}
    \item[\cmark] RMI alta (coeficiente até 100\%)
    \item[\cmark] Progressão anual (fica mais fácil)
    \item[\cmark] Flexibilidade (trabalhe até atingir)
\end{itemize}

\textbf{Desvantagens:}
\begin{itemize}
    \item[\xmark] Exige tempo mínimo alto (35H/30M)
    \item[\xmark] Pode demorar anos para atingir
\end{itemize}

\textbf{Perfil ideal:}
\begin{itemize}
    \item Cliente com 30-34 anos de tempo
    \item Cliente que ainda quer/pode trabalhar
    \item Cliente que busca RMI máxima
\end{itemize}
\end{estrategiaCJP}

\section{Regra 2: Pedágio de 50\% (Art. 17, EC 103/2019)}

\subsection{Como Funciona}

Se você estava a \textbf{menos de 2 anos} de completar 35H/30M em 13/11/2019, pode cumprir um \textbf{pedágio de 50\%} do tempo que faltava.

\subsection{Requisitos}

\begin{enumerate}
    \item \textbf{Faltavam no máximo 2 anos em 13/11/2019}
    \item Cumprir 50\% de pedágio sobre o tempo que faltava
    \item \textbf{Sem idade mínima}
\end{enumerate}

\subsection{Cálculo da RMI}

$$RMI = SB \times Fator\ Previdenci\acute{a}rio$$

(Usa a regra ANTIGA de cálculo)

\subsection{Exemplo Prático}

\textbf{Cliente:} Carlos, tinha 33 anos de contribuição em 13/11/2019

\textbf{Cálculo:}
\begin{itemize}
    \item Faltavam: 35 - 33 = 2 anos (\cmark pode usar)
    \item Pedágio: 2 × 50\% = 1 ano
    \item Tempo total necessário: 33 + 2 + 1 = 36 anos
\end{itemize}

\textbf{Data de elegibilidade:} Quando completar 36 anos de contribuição (sem idade mínima)

\textbf{RMI:}
\begin{verbatim}
SB = R$ 5.800,00
Fator Previdenciário (estimado): 0,85
RMI = R$ 5.800 × 0,85 = R$ 4.930,00
\end{verbatim}

\begin{estrategiaCJP}
\textbf{PEDÁGIO 50\% --- Perfil Ideal}

\textbf{Vantagens:}
\begin{itemize}
    \item[\cmark] Sem idade mínima (aposenta mais jovem)
    \item[\cmark] Pedágio reduzido (50\%)
\end{itemize}

\textbf{Desvantagens:}
\begin{itemize}
    \item[\xmark] Usa Fator Previdenciário (RMI pode ser baixa)
    \item[\xmark] Só para quem faltava $\leq$ 2 anos em 2019
\end{itemize}

\textbf{Perfil ideal:}
\begin{itemize}
    \item Cliente que tinha 33-34 anos em 2019
    \item Cliente que quer aposentar SEM idade mínima
    \item Cliente com Fator Previdenciário favorável ($>$ 0,90)
\end{itemize}
\end{estrategiaCJP}

\begin{armadilha}
\textbf{FATOR PREV NO PEDÁGIO 50\%}

Muitos advogados recomendam o Pedágio 50\% porque ``é rápido'' (menos tempo de espera).

Mas ESQUECEM que o Pedágio 50\% OBRIGA o uso do Fator Previdenciário.

\textbf{Resultado:} Cliente se aposenta rápido, mas com RMI 20-40\% MENOR do que poderia ter esperando 2-3 anos para usar outra regra.

\textbf{AÇÃO CJP:} SEMPRE calcule o Fator antes de recomendar Pedágio 50\%. Se Fator $<$ 0,85, evite.
\end{armadilha}

\section{Regra 3: Pedágio de 100\% (Art. 20, EC 103/2019)}

\subsection{Como Funciona}

Cumprir \textbf{100\% de pedágio} sobre o tempo que faltava em 13/11/2019, mais idade mínima.

\subsection{Requisitos}

\begin{enumerate}
    \item Cumprir 100\% de pedágio sobre o tempo que faltava
    \item \textbf{Idade mínima:}
    \begin{itemize}
        \item Homem: 60 anos
        \item Mulher: 57 anos
    \end{itemize}
\end{enumerate}

\subsection{Cálculo da RMI}

$$RMI = SB \times 100\%\ (INTEGRAL)$$

\textbf{BÔNUS:} Usa PBC 80\% (descarta 20\% menores salários)!

\subsection{Exemplo Prático}

\textbf{Cliente:} Ana, tinha 27 anos de contribuição em 13/11/2019

\textbf{Cálculo:}
\begin{itemize}
    \item Faltavam: 30 - 27 = 3 anos
    \item Pedágio: 3 × 100\% = 3 anos
    \item Tempo total necessário: 27 + 3 + 3 = 33 anos
    \item Idade mínima: 57 anos
\end{itemize}

\textbf{Data de elegibilidade:} Quando completar 33 anos de contribuição \textbf{E} 57 anos de idade

\textbf{RMI (quando completar 57 anos e 33 de tempo):}
\begin{verbatim}
SB = R$ 6.200,00
RMI = R$ 6.200 × 100% = R$ 6.200,00 (INTEGRAL)
\end{verbatim}

\begin{estrategiaCJP}
\textbf{PEDÁGIO 100\% --- Perfil Ideal}

\textbf{Vantagens:}
\begin{itemize}
    \item[\cmark] RMI INTEGRAL (100\% do SB)
    \item[\cmark] Usa PBC 80\% (geralmente melhor SB)
    \item[\cmark] Previsibilidade (não tem Fator Prev)
\end{itemize}

\textbf{Desvantagens:}
\begin{itemize}
    \item[\xmark] Pedágio dobrado (demora mais)
    \item[\xmark] Exige idade mínima (60H/57M)
\end{itemize}

\textbf{Perfil ideal:}
\begin{itemize}
    \item Cliente que faltavam 3-5 anos em 2019
    \item Cliente que busca RMI MÁXIMA
    \item Cliente com paciência (vai demorar)
\end{itemize}
\end{estrategiaCJP}

\section{Regra 4: Idade Progressiva (Art. 18, EC 103/2019)}

\subsection{Como Funciona}

Combina \textbf{tempo mínimo + idade progressiva}.

\subsection{Requisitos por Ano}

\begin{table}[H]
\centering
\caption{Idade Mínima da Regra Idade Progressiva (2019-2031)}
\begin{tabular}{|c|c|c|}
\hline
\textbf{Ano} & \textbf{Idade Homem} & \textbf{Idade Mulher} \\
\hline
2019 & 61 anos & 56 anos \\
\hline
2020 & 61,5 anos & 56,5 anos \\
\hline
2021 & 62 anos & 57 anos \\
\hline
2022 & 62,5 anos & 57,5 anos \\
\hline
2023 & 63 anos & 58 anos \\
\hline
2024 & 63,5 anos & 58,5 anos \\
\hline
2025 & 64 anos & 59 anos \\
\hline
\rowcolor{yellow!30} \textbf{2026} & \textbf{64,5 anos} & \textbf{59,5 anos} \\
\hline
2027 (Teto H) & 65 anos & 60 anos \\
\hline
2031+ (Teto M) & 65 anos & 62 anos \\
\hline
\end{tabular}
\end{table}

\textbf{Tempo mínimo obrigatório:}
\begin{itemize}
    \item Homem: 35 anos
    \item Mulher: 30 anos
\end{itemize}

\subsection{Cálculo da RMI}

$$RMI = SB \times Coeficiente$$

$$Coeficiente = 60\% + 2\% \times (anos - 20H/15M)$$

\subsection{Exemplo Prático (Valores 2026)}

\textbf{Cliente:} Roberto, 64 anos, 35 anos de contribuição em 2026

\textbf{Cálculo:}
\begin{itemize}
    \item Idade = 64 anos
    \item Idade necessária em 2026 = 64,5 anos
    \item \textbf{Falta meio ano}
\end{itemize}

\textbf{Opções:}
\begin{enumerate}
    \item Trabalhar mais 6 meses (chegar aos 64,5 em 2026) \cmark
    \item Avaliar outra regra
\end{enumerate}

\textbf{RMI (quando atingir 64,5 anos em 2026):}
\begin{verbatim}
SB = R$ 5.800,00
Coeficiente = 60% + 2% × (35 - 20) = 60% + 30% = 90%
RMI = R$ 5.800 × 0,90 = R$ 5.220,00
\end{verbatim}

\begin{estrategiaCJP}
\textbf{IDADE PROGRESSIVA --- Perfil Ideal}

\textbf{Vantagens:}
\begin{itemize}
    \item[\cmark] Mais rápida que Pontos (em geral)
    \item[\cmark] RMI boa (coeficiente até 100\%)
\end{itemize}

\textbf{Desvantagens:}
\begin{itemize}
    \item[\xmark] Exige idade relativamente alta
    \item[\xmark] Tempo mínimo alto (35H/30M)
\end{itemize}

\textbf{Perfil ideal:}
\begin{itemize}
    \item Cliente que já tem o tempo (35/30)
    \item Cliente que está perto da idade
    \item Cliente que quer aposentar logo
\end{itemize}
\end{estrategiaCJP}

\section{Regra 5: Permanente (Art. 19, EC 103/2019)}

\subsection{Como Funciona}

A regra ``definitiva'' pós-Reforma. Sem transição.

\subsection{Requisitos}

\begin{itemize}
    \item \textbf{Idade:}
    \begin{itemize}
        \item Homem: 65 anos
        \item Mulher: 62 anos
    \end{itemize}
    \item \textbf{Tempo mínimo:}
    \begin{itemize}
        \item Homem: 15 anos
        \item Mulher: 15 anos
    \end{itemize}
\end{itemize}

\subsection{Cálculo da RMI}

$$RMI = SB \times Coeficiente$$

$$Coeficiente = 60\% + 2\% \times (anos - 20H/15M)$$

\subsection{Exemplo Prático (Valores 2026)}

\textbf{Cliente:} Paula, 62 anos, 18 anos de contribuição

\textbf{Cálculo:}
\begin{itemize}
    \item Idade = 62 anos \cmark
    \item Tempo = 18 anos \cmark
\end{itemize}

\textbf{RMI:}
\begin{verbatim}
SB = R$ 4.500,00
Coeficiente = 60% + 2% × (18 - 15) = 60% + 6% = 66%
RMI = R$ 4.500 × 0,66 = R$ 2.970,00
\end{verbatim}

\begin{estrategiaCJP}
\textbf{PERMANENTE --- Perfil Ideal}

\textbf{Vantagens:}
\begin{itemize}
    \item[\cmark] Tempo mínimo baixo (15 anos)
    \item[\cmark] Simples (sem transições complexas)
\end{itemize}

\textbf{Desvantagens:}
\begin{itemize}
    \item[\xmark] Idade alta (65H/62M)
    \item[\xmark] RMI baixa se tiver pouco tempo
\end{itemize}

\textbf{Perfil ideal:}
\begin{itemize}
    \item Cliente que começou a contribuir tarde
    \item Cliente com menos de 25 anos de tempo
    \item Cliente que não se enquadra nas outras
\end{itemize}
\end{estrategiaCJP}

\section{Matriz de Decisão: Qual Regra Aplicar?}

\subsection{Tabela Comparativa das 5 Regras (Valores 2026)}

\begin{table}[H]
\centering
\caption{Matriz de Decisão: As 5 Regras de Transição (2026)}
\begin{tabular}{|p{2.2cm}|p{2cm}|p{2.5cm}|p{2cm}|p{4cm}|}
\hline
\textbf{Regra} & \textbf{Tempo Mín.} & \textbf{Idade Mín.} & \textbf{RMI} & \textbf{Melhor Para} \\
\hline
Pontos & 35H/30M & 103H/93M pts & 60-100\% & Cliente 30+ anos, busca RMI alta \\
\hline
Pedágio 50\% & 33H+ped/28M+ped & Nenhuma & Fator Prev & Faltava $\leq$ 2 anos em 2019 \\
\hline
Pedágio 100\% & Tempo+ped & 60H/57M & 100\% & Busca RMI MÁXIMA \\
\hline
Idade Prog. & 35H/30M & 64,5H/59,5M & 60-100\% & Tem tempo, perto da idade \\
\hline
Permanente & 15 & 65H/62M & 60-100\% & $<$ 25 anos de tempo \\
\hline
\end{tabular}
\end{table}

\subsection{Exemplos Práticos por Perfil de Cliente}

\subsubsection{Perfil 1: João, 62 anos, 32 anos de contribuição, SB = R\$ 7.000}

\begin{center}
\begin{tabularx}{\textwidth}{|l|c|c|c|X|}
\hline
\textbf{Regra} & \textbf{Elegível?} & \textbf{Data} & \textbf{RMI} & \textbf{Observação} \\
\hline
Pontos & \xmark & --- & --- & Faltam 9 pontos (94 atual vs. 103) \\
\hline
Ped. 50\% & \xmark & --- & --- & Tinha 30 anos em 2019 (faltavam 5) \\
\hline
Ped. 100\% & \cmark & Daqui 3 anos & R\$ 7.000 & \textbf{Melhor opção!} \\
\hline
Idade Prog. & \xmark & --- & --- & Precisa 64,5 anos + 35 tempo \\
\hline
Permanente & \xmark & --- & --- & Precisa 65 anos \\
\hline
\end{tabularx}
\end{center}

\textbf{Recomendação CJP:} Pedágio 100\% (aguardar completar tempo + idade)

\subsubsection{Perfil 2: Maria, 58 anos, 33 anos de contribuição, SB = R\$ 6.000}

\begin{center}
\begin{tabularx}{\textwidth}{|l|c|c|c|X|}
\hline
\textbf{Regra} & \textbf{Elegível?} & \textbf{Data} & \textbf{RMI} & \textbf{Observação} \\
\hline
Pontos & \cmark & 1 ano & R\$ 5.880 & 59 + 34 = 93 pontos \\
\hline
Ped. 50\% & \cmark & Agora & R\$ 4.800 & Fator Prev. baixo \\
\hline
Ped. 100\% & \xmark & --- & --- & Precisa 57 (tem), falta tempo \\
\hline
Idade Prog. & \cmark & 1,5 anos & R\$ 5.760 & 59,5 + 34,5 tempo \\
\hline
Permanente & \xmark & --- & --- & Precisa 62 anos \\
\hline
\end{tabularx}
\end{center}

\textbf{Recomendação CJP:} Pontos (melhor custo-benefício)

\subsubsection{Perfil 3: Carlos, 66 anos, 18 anos de contribuição, SB = R\$ 4.500}

\begin{center}
\begin{tabularx}{\textwidth}{|l|c|c|c|X|}
\hline
\textbf{Regra} & \textbf{Elegível?} & \textbf{Data} & \textbf{RMI} & \textbf{Observação} \\
\hline
Pontos & \xmark & --- & --- & Não tem tempo mínimo (35) \\
\hline
Ped. 50\% & \xmark & --- & --- & Não se aplica \\
\hline
Ped. 100\% & \xmark & --- & --- & Não tem tempo mínimo \\
\hline
Idade Prog. & \xmark & --- & --- & Não tem tempo mínimo (35) \\
\hline
Permanente & \cmark & Agora & R\$ 2.970 & \textbf{Única opção} \\
\hline
\end{tabularx}
\end{center}

\textbf{Recomendação CJP:} Permanente (única viável)

\section{Mensagem Final do Módulo 5}

Você acabou de dominar o \textbf{núcleo matemático} do Método CJP.

Agora você sabe:
\begin{itemize}
    \item[\cmark] As 3 Eras do PBC e quando aplicar cada uma
    \item[\cmark] Como calcular o Salário de Benefício (SB) com Divisor Mínimo
    \item[\cmark] A Tese do Vácuo Previdenciário e como revisar benefícios
    \item[\cmark] As 5 Regras de Transição da EC 103/2019 (valores 2026)
    \item[\cmark] Como escolher estrategicamente a melhor rota para cada cliente
\end{itemize}

\textbf{Mas ainda faltam duas etapas críticas:}

\begin{enumerate}
    \item \textbf{Módulo 6:} Calcular o \textbf{Coeficiente} e a \textbf{RMI final}, e auditar o cálculo completo (Fator Previdenciário, pedágios, coeficientes, Tema 1300 para aposentadoria por incapacidade).
    \item \textbf{Módulo 7:} Aplicar essas mesmas técnicas em \textbf{Benefícios Não Programáveis} (Pensão por Morte, Auxílio-Incapacidade, Salário-Maternidade).
\end{enumerate}

\begin{novidade}
\textbf{IMPORTANTE: TESES CANCELADAS}

A ``Revisão da Vida Toda'' foi DEFINITIVAMENTE CANCELADA pelo STF em abril/2025 (Tema 1102) e confirmada pelas ADIs 2110/2111 em novembro/2025.

\xmark\ Novas ações são IMPROCEDENTES.

Detalhes: Módulo 9 (Teses Revisionais)
\end{novidade}

\vfill

\begin{center}
\textit{``Calcular é fácil. Calcular CERTO é arte. Calcular a MELHOR opção é maestria.''}\\
\textbf{--- Sistema CJP, 2026}
\end{center}

%% Continua no Módulo 6

%% ============================================================================
%% INFOGRÁFICO DO MÓDULO 5
%% ============================================================================
\clearpage
\backtotoc

\section*{\faImage\ Infográfico de Consolidação}

\begin{figure}[H]
    \centering
    \begin{tcolorbox}[colback=white, colframe=cjpAzulEscuro, title={\textbf{\faBookOpen\ Infográfico: Módulo 5 --- Regras de Transição}}, fonttitle=\bfseries\color{white}, sharp corners=downhill, boxrule=2pt]
        \centering
        \includegraphics[width=0.95\textwidth, keepaspectratio]{modulo5}
    \end{tcolorbox}
    \caption{Resumo Visual do Módulo 5: Regras de Transição}
    \label{fig:modulo5}
\end{figure}
