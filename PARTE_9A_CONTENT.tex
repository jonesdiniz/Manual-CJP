\clearpage
\chapter*{Módulo 9: Bônus - Teses Revisionais e Modelos Práticos}
\addcontentsline{toc}{chapter}{Módulo 9: Bônus - Teses Revisionais e Modelos Práticos}
\markboth{Módulo 9: Bônus - Teses Revisionais e Modelos Práticos}{Módulo 9: Bônus - Teses Revisionais e Modelos Práticos}
\setcounter{chapter}{9}

\begin{center}
{\Large\textit{``A Caixa de Ferramentas Executiva: Da Teoria à Execução Imediata''}}\\[0.5cm]
\textbf{Sistema CJP | Módulo Bônus | Ferramentas de Execução}
\end{center}

\begin{acaoImediata}
\textbf{ESTE É O BÔNUS INESPERADO}

``Conhecimento sem execução é apenas teoria acadêmica.''

Os Módulos 0-8 te ensinaram O QUÊ e COMO fazer. Este módulo te entrega as FERRAMENTAS PRONTAS para executar.

Aqui você encontra:
\begin{itemize}
    \item Teses revisionais atualizadas 2025-2026
    \item Modelos de petições copy/paste
    \item Templates de recursos prontos
    \item Quesitos periciais estratégicos
    \item Checklist master completo
    \item Glossário técnico de consulta rápida
\end{itemize}

Este módulo transforma você de ``advogado que sabe'' em ``advogado que EXECUTA''.
\end{acaoImediata}

%% \tableofcontents removido - sumário único no master

\section{A Lógica do Bônus: Por Que Este Módulo É Diferente}
\nopagebreak[4]
\subsection{O Que Separa Teoria de Execução?}

Você acabou de dominar 8 módulos completos do Sistema CJP:
\begin{itemize}
    \item Sabe fazer entrevista estratégica (M1)
    \item Sabe auditar CNIS (M2)
    \item Sabe identificar armadilhas ocultas (M3)
    \item Sabe corrigir vínculos (M4)
    \item Sabe calcular PBC e transições (M5)
    \item Sabe auditar RMI (M6)
    \item Sabe calcular benefícios não programáveis (M7)
    \item Sabe montar parecer premium (M8)
\end{itemize}

\textbf{Pergunta brutal:} Mas quando chegar segunda-feira e um cliente entrar no seu escritório, você tem os MODELOS PRONTOS para executar?

\begin{conceitoChave}
\textbf{A ARMADILHA DA TEORIA}

Advogados investem anos estudando COMO fazer as coisas, mas quando chegam na prática:
\begin{itemize}
    \item Ficam 2h tentando redigir RAC
    \item Esquecem requisitos obrigatórios
    \item Perdem prazos por falta de checklist
    \item Não sabem quais quesitos fazer
    \item Não lembram da base legal exata
\end{itemize}

Este módulo elimina TODA essa fricção.

Você não vai ``aprender a fazer''. Você vai COPIAR, COLAR e EXECUTAR.
\end{conceitoChave}

\section{Teses Revisionais 2025-2026: Catálogo Completo}

\subsection{Teses VITORIOSAS (Aplicar Imediatamente)}

\begin{teseRevisional}
\textbf{TEMA 1070 STJ --- Atividades Concomitantes (VITORIOSA)}

\textbf{O que é:} Todas as remunerações concomitantes devem ser somadas (100\%), não apenas proporcionalmente.

\textbf{Fundamento:} STJ, REsp 1.870.793/RS --- Tema 1070

\textbf{Impacto:} +R\$ 500 a +R\$ 2.000/mês para quem tinha 2+ vínculos

\textbf{Aplicável a:} Benefícios concedidos com cálculo proporcional antes do Tema

\textbf{Prazo:} Até 10 anos após a DIB

\textbf{Taxa de êxito:} 95\%+
\end{teseRevisional}

\begin{teseRevisional}
\textbf{TEMA 555 STJ --- Revisão de Teto Constitucional (VITORIOSA)}

\textbf{O que é:} Aplicação do teto das ECs 20/1998 e 41/2003 para quem teve benefício limitado ao teto anterior.

\textbf{Fundamento:} STJ, REsp 1.149.316/SP --- Tema 555

\textbf{Impacto:} +R\$ 200 a +R\$ 800/mês para benefícios de 1988-2003

\textbf{Aplicável a:} Benefícios com RMI limitada ao teto na DIB

\textbf{Prazo:} Sem prescrição do fundo de direito (só parcelas $>$ 5 anos)

\textbf{Taxa de êxito:} 90\%+
\end{teseRevisional}

\begin{teseRevisional}
\textbf{TEMA 1152 STJ --- Vácuo Legislativo do Divisor (VITORIOSA)}

\textbf{O que é:} Entre 13/11/2019 e 08/05/2022, não havia divisor mínimo de 108.

\textbf{Fundamento:} STJ, REsp 1.956.543/SP --- Tema 1152

\textbf{Impacto:} +R\$ 500 a +R\$ 2.500/mês para benefícios no período

\textbf{Aplicável a:} DIB entre 13/11/2019 e 08/05/2022 com divisor errado

\textbf{Prazo:} Até 10 anos após a DIB

\textbf{Taxa de êxito:} 90\%+
\end{teseRevisional}

\subsection{Tese REJEITADA (NÃO Aplicar)}

\begin{teseRejeitada}
\textbf{TEMA 1300 STF --- AIP com Coeficiente 100\% (REJEITADA --- 6×5)}

\textbf{O que era:} Pleiteava coeficiente de 100\% para aposentadoria por invalidez comum (não acidentária) pós-2019.

\textbf{Decisão:} STF, RE 1.469.150 --- Tema 1300 --- Dezembro/2025

\textbf{Placar:} 6×5 pela CONSTITUCIONALIDADE (tese rejeitada)

\textbf{Resultado:}
\begin{itemize}
    \item[\xmark] A regra de 60\%+2\% para AIP comum foi MANTIDA
    \item[\xmark] NÃO há direito a revisão para coeficiente de 100\%
    \item[\xmark] Ações judiciais com essa tese tendem a ser improcedentes
\end{itemize}

\textbf{ORIENTAÇÃO:}
\begin{itemize}
    \item[\xmark] NÃO ajuizar ações pleiteando coeficiente de 100\%
    \item[\cmark] Informar clientes que a tese foi REJEITADA
    \item[\cmark] Encerrar processos pendentes sobre esta matéria
    \item[\cmark] Estratégia alternativa: Nexo causal (B32 → B92)
\end{itemize}
\end{teseRejeitada}

\subsection{Teses NOVAS 2025 (Aplicar com Cuidado)}

\begin{novidade}
\textbf{Lei 15.108/2025 --- Menor Sob Guarda em Pensão (NOVA)}

\textbf{Vigência:} 13/03/2025

\textbf{O que mudou:} Menor sob guarda judicial é equiparado a filho para fins de pensão por morte.

\textbf{Impacto:} +10\% no coeficiente (+R\$ 300-800/mês)

\textbf{Aplicável a:} Óbitos APÓS 13/03/2025

\textbf{Taxa de êxito:} 95\%+ (administrativa)
\end{novidade}

\begin{novidade}
\textbf{Lei 15.222/2025 --- Prorrogação Salário-Maternidade (NOVA)}

\textbf{Vigência:} 30/09/2025

\textbf{O que mudou:} Internação $>$ 14 dias prorroga o benefício.

\textbf{Impacto:} +15 a +90 dias de benefício (R\$ 1.500-10.000)

\textbf{Aplicável a:} Partos APÓS 30/09/2025

\textbf{Taxa de êxito:} 90\%+ (administrativa)
\end{novidade}

\begin{novidade}
\textbf{IN 188/2025 --- Isenção de Carência Salário-Maternidade (NOVA)}

\textbf{Vigência:} 10/07/2025 (retroage a 05/04/2024)

\textbf{O que mudou:} ZERO carência para salário-maternidade (todas as categorias).

\textbf{Impacto:} Benefício integral para quem tinha $<$ 10 contribuições

\textbf{Aplicável a:} Pedidos APÓS 05/04/2024 indeferidos por carência

\textbf{Taxa de êxito:} 98\%+ (administrativa)
\end{novidade}

\subsection{Tabela Síntese de Aplicabilidade}

\begin{center}
\begin{tabular}{|p{3.5cm}|c|c|c|}
\hline
\textbf{Tese} & \textbf{Status} & \textbf{Impacto} & \textbf{Via} \\
\hline
Tema 1070 (Concomitantes) & \cmark\ Vitoriosa & Alto & Administrativa/Judicial \\
\hline
Tema 555 (Teto EC 20/41) & \cmark\ Vitoriosa & Médio & Administrativa/Judicial \\
\hline
Tema 1152 (Vácuo Divisor) & \cmark\ Vitoriosa & Alto & Administrativa/Judicial \\
\hline
Tema 1300 (AIP 100\%) & \xmark\ Rejeitada & --- & NÃO AJUIZAR \\
\hline
Lei 15.108 (Menor Guarda) & \cmark\ Nova & Médio & Administrativa \\
\hline
Lei 15.222 (Prorrog. Mat.) & \cmark\ Nova & Médio & Administrativa \\
\hline
IN 188 (Carência Zero) & \cmark\ Nova & Alto & Administrativa \\
\hline
\end{tabular}
\end{center}

\section{Modelo de Petição: RAC (Requerimento de Acerto de CNIS)}

\begin{verbatim}
REQUERIMENTO DE ACERTO DE CADASTRO NACIONAL DE
INFORMAÇÕES SOCIAIS (RAC)

AO INSTITUTO NACIONAL DO SEGURO SOCIAL - INSS
Agência da Previdência Social de [CIDADE/UF]

REQUERENTE: [NOME COMPLETO]
CPF: [XXX.XXX.XXX-XX]
NIT: [XXX.XXXXX.XX-X]
Data de Nascimento: [DD/MM/AAAA]
Endereço: [ENDEREÇO COMPLETO]
Telefone: [TELEFONE]
E-mail: [E-MAIL]

I - DO OBJETO

Vem, respeitosamente, requerer o ACERTO DE CADASTRO
no CNIS (Cadastro Nacional de Informações Sociais),
para inclusão/retificação dos seguintes dados:

1. PERÍODO: [DD/MM/AAAA a DD/MM/AAAA]
   EMPREGADOR: [NOME/CNPJ]
   ALTERAÇÃO SOLICITADA: [DESCREVER]
   DOCUMENTOS ANEXADOS: [LISTAR]

2. [REPETIR PARA CADA PERÍODO]

II - DOS FUNDAMENTOS

Nos termos do Art. 39 da Lei 13.846/2019 c/c Portaria
INSS 10/2025, o segurado tem direito à retificação de
dados cadastrais que não correspondam à realidade de
suas contribuições.

Os documentos anexados comprovam inequivocamente o
direito do requerente.

III - DOS DOCUMENTOS ANEXADOS

[  ] Cópia RG e CPF
[  ] CNIS atualizado
[  ] CTPS (páginas relevantes)
[  ] Declaração firmada pelo empregador
[  ] Extrato analítico FGTS
[  ] GPS/GFIP (carnês)
[  ] [OUTROS DOCUMENTOS]

IV - DO PEDIDO

Requer seja deferido o presente pedido de acerto de
CNIS, determinando-se a inclusão/retificação dos
períodos descritos no item I.

Termos em que,
Pede deferimento.

[CIDADE], [DATA]

___________________________
[NOME DO REQUERENTE]
ou
___________________________
[NOME DO ADVOGADO]
OAB/[UF] [NÚMERO]
\end{verbatim}

\section{Quesitos Periciais Estratégicos}

\subsection{Quesitos para Exposição a Ruído}

\begin{acaoImediata}
\textbf{QUESITOS PARA PERÍCIA --- RUÍDO}

\begin{enumerate}
    \item Quais eram as funções executadas pelo autor no período de [DATA] a [DATA]?
    \item O autor era exposto a ruído ocupacional? Em qual intensidade (dB)?
    \item O nível de ruído ultrapassava o limite de tolerância vigente à época?
    \item Quais EPIs eram fornecidos? Eram suficientes para neutralizar o agente?
    \item Havia Programa de Prevenção de Riscos Ambientais (PPRA)?
    \item O ambiente de trabalho possuía Laudo Técnico (LTCAT)?
    \item A exposição era habitual e permanente?
    \item Havia medidas de proteção coletiva?
    \item O ruído era contínuo, intermitente ou de impacto?
    \item A atividade se enquadra como especial nos termos do Decreto 53.831/64 ou Decreto 3.048/99?
\end{enumerate}
\end{acaoImediata}

\subsection{Quesitos para Agentes Químicos}

\begin{acaoImediata}
\textbf{QUESITOS PARA PERÍCIA --- AGENTES QUÍMICOS}

\begin{enumerate}
    \item Quais agentes químicos estavam presentes no ambiente de trabalho?
    \item Havia manipulação direta de substâncias químicas?
    \item Quais as concentrações dos agentes no ambiente?
    \item Ultrapassavam os limites de tolerância das NRs?
    \item O contato era por via cutânea, respiratória ou digestiva?
    \item Havia monitoramento biológico?
    \item Quais EPIs eram fornecidos e utilizados?
    \item A exposição era habitual, permanente e não ocasional?
    \item Os agentes são classificados como cancerígenos pelo IARC?
    \item A atividade se enquadra nos Anexos do Decreto 3.048/99?
\end{enumerate}
\end{acaoImediata}

\section{Checklist Master Consolidado}

\begin{acaoImediata}
\textbf{CHECKLIST MASTER DOS 5 PILARES CJP}

\textbf{PILAR 1 --- ENTREVISTA ESTRATÉGICA}
\begin{itemize}
    \item[$\square$] 8 perguntas-gatilho aplicadas
    \item[$\square$] Procuração eletrônica (Portaria 10/2025)
    \item[$\square$] CNIS completo obtido
    \item[$\square$] Objetivos do cliente documentados
\end{itemize}

\textbf{PILAR 2 --- DIAGNÓSTICO EXPLÍCITO + IMPLÍCITO}
\begin{itemize}
    \item[$\square$] 15 indicadores CNIS verificados
    \item[$\square$] 8 armadilhas ocultas checadas
    \item[$\square$] Tema 1090 STJ verificado (especial)
    \item[$\square$] Impacto financeiro calculado
\end{itemize}

\textbf{PILAR 3 --- ACERTOS E DOCUMENTAÇÃO}
\begin{itemize}
    \item[$\square$] Dossiê de provas montado
    \item[$\square$] RAC protocolado (se necessário)
    \item[$\square$] Documentação hierarquizada
    \item[$\square$] Protocolo acompanhado
\end{itemize}

\textbf{PILAR 4 --- CÁLCULOS SISTEMATIZADOS}
\begin{itemize}
    \item[$\square$] PBC correto identificado (80\%/100\%)
    \item[$\square$] Divisor mínimo verificado
    \item[$\square$] 5 regras de transição testadas
    \item[$\square$] Coeficiente/Fator calculado
    \item[$\square$] RMI final validada
    \item[$\square$] Teto/Piso respeitados
\end{itemize}

\textbf{PILAR 5 --- ENTREGA DE VALOR}
\begin{itemize}
    \item[$\square$] Dossiê 4 blocos montado
    \item[$\square$] Reunião agendada (NÃO e-mail)
    \item[$\square$] Roteiro 5 etapas preparado
    \item[$\square$] ROI calculado
    \item[$\square$] Contrato fechado
\end{itemize}
\end{acaoImediata}

\section{Conclusão do Sistema CJP}

\begin{estrategiaCJP}
\textbf{VOCÊ COMPLETOU O SISTEMA CJP!}

\textbf{O que você domina agora:}
\begin{itemize}
    \item[\cmark] Entrevista Estratégica com 8 gatilhos
    \item[\cmark] Auditoria CNIS com 15 indicadores
    \item[\cmark] Detecção de 8 armadilhas ocultas
    \item[\cmark] Acertos de vínculos com RAC
    \item[\cmark] Cálculos das 5 transições EC 103/2019
    \item[\cmark] Auditoria completa de RMI
    \item[\cmark] Benefícios não programáveis
    \item[\cmark] Parecer comercial premium
    \item[\cmark] Teses revisionais 2025-2026
\end{itemize}

\textbf{Seu diferencial competitivo:}

Enquanto 95\% dos advogados previdenciários ``rodam software'', você AUDITA, VALIDA e ENTREGA VALOR.

Isso vale R\$ 1.500-15.000 por cliente, não R\$ 500.

\textbf{Próximo passo:}

Aplique o Sistema CJP no seu PRÓXIMO cliente e veja a diferença na sua receita e na satisfação do cliente.
\end{estrategiaCJP}

\vfill

\begin{center}
\textbf{FIM DO SISTEMA CJP COMPLETO}\\[0.5cm]
\textit{``A Maestria em Planejamento Previdenciário''}\\[1cm]
\textbf{Dr. Jones Weslley Bueno Diniz}\\
OAB/SP 377.329\\
Método CJP | Cálculos Jurídicos Precisos\\[0.5cm]
\textbf{Edição Janeiro 2026}
\end{center}

%% ============================================================================
%% INFOGRÁFICO DO MÓDULO 9
%% ============================================================================
\clearpage
\backtotoc

\section*{\faImage\ Infográfico de Consolidação}

\begin{figure}[H]
    \centering
    \begin{tcolorbox}[colback=white, colframe=cjpAzulEscuro, title={\textbf{\faBookOpen\ Infográfico: Módulo 9 --- Teses Revisionais}}, fonttitle=\bfseries\color{white}, sharp corners=downhill, boxrule=2pt]
        \centering
        \includegraphics[width=0.95\textwidth, keepaspectratio]{modulo9}
    \end{tcolorbox}
    \caption{Resumo Visual do Módulo 9: Teses Revisionais}
    \label{fig:modulo9}
\end{figure}

