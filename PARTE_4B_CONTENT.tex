%% Continuação do Módulo 4: Acertos e Documentação

\section{Guia Prático: Taxa de Sucesso por Tipo de Acerto}

\begin{center}
\begin{tabular}{|l|c|l|}
\hline
\textbf{Tipo de Acerto} & \textbf{Taxa de Sucesso} & \textbf{Documentos Chave} \\
\hline
PEXT com CTPS + FGTS & 95\% & CTPS, Extrato FGTS \\
\hline
PEXT (empresa extinta) & 85\% & CTPS, Certidão falência \\
\hline
Complementação GPS (EC103) & 98\% & GPS pagas, comprovantes \\
\hline
Tempo Especial (PPP completo) & 90\% & PPP assinado, LTCAT \\
\hline
Concomitantes (Tema 1070) & 90\% & CTPS ambos vínculos \\
\hline
Rural (prova + testemunhas) & 80\% & Certidões, declarações \\
\hline
Tempo Militar & 98\% & Certificado Reservista \\
\hline
CTC (RPPS) & 95\% & CTC emitida pelo órgão \\
\hline
Salário-Maternidade PBC & 70\% & Carta de Concessão \\
\hline
\end{tabular}
\end{center}

\section{Checklist de Execução por Tipo de Acerto}

\subsection{Vínculos PEXT}

\begin{itemize}
    \item[$\square$] Obter CTPS original (digitalizar 300 DPI)
    \item[$\square$] Solicitar Extrato FGTS à CEF
    \item[$\square$] Verificar CNPJ da empresa (Receita Federal)
    \item[$\square$] Se empresa falida: Certidão da Junta Comercial
    \item[$\square$] Preencher RAC com período exato e CNPJ
    \item[$\square$] Protocolar e acompanhar (45-60 dias)
\end{itemize}

\subsection{Contribuições Abaixo do Mínimo (EC103)}

\begin{itemize}
    \item[$\square$] Identificar competências com valor abaixo do mínimo
    \item[$\square$] Calcular GPS complementares (SAL)
    \item[$\square$] Emitir GPS (Código 1929 ou 1147)
    \item[$\square$] Pagar e guardar comprovantes
    \item[$\square$] Anexar GPS + comprovantes ao RAC
    \item[$\square$] Protocolar e acompanhar
\end{itemize}

\subsection{Tempo Especial}

\begin{itemize}
    \item[$\square$] Identificar períodos com exposição a agentes nocivos
    \item[$\square$] Obter PPP da empresa (ou PPP Eletrônico pós-2023)
    \item[$\square$] Analisar campo EPI (aplicar Tema 1090 se necessário)
    \item[$\square$] Verificar assinaturas e validade
    \item[$\square$] Se EPI ``eficaz'': preparar impugnação
    \item[$\square$] Anexar PPP + LTCAT ao RAC
\end{itemize}

\begin{novidade}
\textbf{TEMA 1090 STJ (Abril/2025)}

Se PPP indica EPI eficaz, verifique:
\begin{itemize}
    \item Agente é ruído, biológico, cancerígeno ou periculosidade? $\rightarrow$ EPI não afasta
    \item Outros agentes? $\rightarrow$ Ônus do segurado provar ineficácia
    \item Em caso de dúvida $\rightarrow$ Favorável ao segurado
\end{itemize}
\end{novidade}

\subsection{Atividades Concomitantes}

\begin{itemize}
    \item[$\square$] Confirmar períodos de sobreposição (CTPS ambos vínculos)
    \item[$\square$] Verificar regra aplicável (antes/depois de 28/11/1999)
    \item[$\square$] Obter holerites de ambos os empregos (mesmo mês)
    \item[$\square$] Fundamentar com Tema 1070 STJ
    \item[$\square$] Protocolar RAC
\end{itemize}

\subsection{Vínculos Rurais}

\begin{itemize}
    \item[$\square$] Delimitar período rural (início trabalho até 1\textsuperscript{o} emprego urbano)
    \item[$\square$] Obter Certidão de Casamento dos pais (profissão lavrador)
    \item[$\square$] Obter Histórico Escolar de escola rural
    \item[$\square$] Obter Declaração do Sindicato Rural
    \item[$\square$] Preparar 2-3 testemunhas com declarações escritas
    \item[$\square$] Protocolar RAC
    \item[$\square$] Se indeferido: Requerer Audiência de Justificação
\end{itemize}

\begin{estrategiaCJP}
\textbf{ESTRATÉGIA GOLD: Múltiplas Provas Rurais}

Para tempo rural, NUNCA confie em 1 só documento.

\textbf{Estratégia vencedora:}
\begin{itemize}
    \item Certidão de Casamento dos pais (prova ruralidade)
    \item Histórico Escolar de escola rural (prova presença)
    \item Declaração do Sindicato (prova reconhecimento)
    \item 2 testemunhas (prova trabalho efetivo)
\end{itemize}

Com esse conjunto, chance de deferimento: \textbf{90\%+}
\end{estrategiaCJP}

\subsection{Tempo Militar}

\begin{itemize}
    \item[$\square$] Obter Certificado de Reservista
    \item[$\square$] Obter Certidão de Tempo de Serviço Militar
    \item[$\square$] Protocolar RAC
    \item[$\square$] Taxa de sucesso: 98\%
\end{itemize}

\subsection{CTC (Certidão Tempo de Contribuição)}

\begin{itemize}
    \item[$\square$] Identificar RPPS de origem (municipal, estadual, federal)
    \item[$\square$] Solicitar CTC ao órgão (prazo: 30-90 dias)
    \item[$\square$] Verificar se CTC contém período, cargo, salários, assinatura
    \item[$\square$] Anexar ao RAC e protocolar
\end{itemize}

\begin{armadilha}
\textbf{Demora na Emissão da CTC}

A emissão de CTC pode demorar 60-120 dias.

\textbf{ESTRATÉGIA CJP:} Solicite a CTC IMEDIATAMENTE após o diagnóstico. NÃO espere para fazer o restante do planejamento.
\end{armadilha}

\section{Organização Visual do Dossiê}

\begin{estrategiaCJP}
\textbf{ESTRUTURA DO DOSSIÊ RAC}

\texttt{DOSSIÊ RAC - [Nome do Cliente] - [CPF]}\\
\texttt{├── 01\_REQUERIMENTO\_RAC.pdf}\\
\texttt{├── 02\_DOCUMENTOS\_PESSOAIS/}\\
\texttt{│   ├── RG\_CPF.pdf}\\
\texttt{│   ├── Comprovante\_Residencia.pdf}\\
\texttt{│   └── Procuracao.pdf}\\
\texttt{├── 03\_ACERTO\_1\_PEXT/}\\
\texttt{│   ├── CTPS.pdf}\\
\texttt{│   ├── Extrato\_FGTS.pdf}\\
\texttt{│   └── Holerites.pdf}\\
\texttt{├── 04\_ACERTO\_2\_TEMPO\_ESPECIAL/}\\
\texttt{│   ├── PPP.pdf}\\
\texttt{│   └── LTCAT.pdf}\\
\texttt{└── 05\_RELACAO\_DOCUMENTOS.pdf}

\textbf{Regras de Nomenclatura:}
\begin{itemize}
    \item Numere sequencialmente todas as pastas
    \item Use nomes descritivos (não ``doc1.pdf'')
    \item PDFs com menos de 10MB cada
\end{itemize}
\end{estrategiaCJP}

\section{Checklist Executivo Final do Módulo 4}

\begin{acaoImediata}
\textbf{CHECKLIST MASTER --- PILAR 4: ACERTOS DE VÍNCULOS (2026)}

\textbf{FASE 1: PREPARAÇÃO}
\begin{itemize}
    \item[$\square$] Lista de acertos identificados (Módulos 2 e 3)
    \item[$\square$] Procuração eletrônica ativa (se aplicável)
    \item[$\square$] Acesso ao Meu INSS do cliente
\end{itemize}

\textbf{FASE 2: INSTRUÇÃO (Tema 1124 STJ)}
\begin{itemize}
    \item[$\square$] Documentos de Nível 1 para cada acerto
    \item[$\square$] Documentos complementares (Nível 2)
    \item[$\square$] Testemunhas preparadas (se rural)
    \item[$\square$] GPS complementares pagas (se EC103)
\end{itemize}

\textbf{FASE 3: PROTOCOLO}
\begin{itemize}
    \item[$\square$] RAC redigido com estrutura completa
    \item[$\square$] Dossiê organizado e numerado
    \item[$\square$] Digitalização em 300 DPI mínimo
    \item[$\square$] Protocolo via Meu INSS (ou presencial)
    \item[$\square$] Número de protocolo anotado
\end{itemize}

\textbf{FASE 4: ACOMPANHAMENTO}
\begin{itemize}
    \item[$\square$] Alerta configurado para exigências
    \item[$\square$] Monitoramento quinzenal
    \item[$\square$] Resposta a exigências em até 30 dias
    \item[$\square$] Recurso em 30 dias se indeferido
\end{itemize}

\textbf{FASE 5: VERIFICAÇÃO}
\begin{itemize}
    \item[$\square$] CNIS atualizado baixado após deferimento
    \item[$\square$] Conferência: acertos refletidos corretamente?
    \item[$\square$] Se erro: novo RAC para correção
\end{itemize}
\end{acaoImediata}

\section{Próximo Passo}

Com o CNIS corrigido, você está pronto para o \textbf{Pilar 5: O Cálculo Preciso}.

No \textbf{Módulo 5}, você aprenderá:
\begin{itemize}
    \item O PBC (Período Básico de Cálculo) correto
    \item As 5 Regras de Transição da EC 103/2019
    \item Como escolher a melhor regra para cada cliente
    \item Simulação de RMI com precisão
\end{itemize}

%% Continua na Parte 4C (Modelos RAC e Fluxograma)
