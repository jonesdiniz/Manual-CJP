\clearpage
\chapter*{Módulo 2: Auditoria do CNIS}
\addcontentsline{toc}{chapter}{Módulo 2: Auditoria do CNIS}
\markboth{Módulo 2: Auditoria do CNIS}{Módulo 2: Auditoria do CNIS}
\setcounter{chapter}{2}

\begin{center}
{\Large\textit{``O Diagnóstico Explícito: Decodificando os 15 Indicadores Prioritários''}}\\[0.5cm]
\textbf{Sistema CJP | Pilar 2 de 5 | Parte 1: Diagnóstico Explícito}
\end{center}

\begin{novidade}
\textbf{Validade Temporal --- Janeiro de 2026}

\begin{itemize}
    \item[\cmark] Portaria DIRBEN/INSS n.\textsuperscript{o} 1.316/2025 (Indicadores CNIS)
    \item[\cmark] Procuração Eletrônica Meu INSS (Portaria 10/2025)
    \item[\cmark] Tema 1124 STJ (interesse de agir e DIB)
    \item[\cmark] Tema 1090 STJ (EPI e ônus da prova)
    \item[\xmark] Tema 1300 STF (coeficiente AIP) --- 6×5 REJEITADO
    \item[\xmark] Revisão da Vida Toda --- CANCELADA
\end{itemize}
\end{novidade}

\begin{acaoImediata}
O CNIS do seu cliente NÃO é um ``extrato de direitos''. É um RELATÓRIO DE ERROS que você precisa decodificar, corrigir e transformar em tempo e dinheiro.

Este módulo te ensina a ler o que o INSS ``está dizendo'' em código.
\end{acaoImediata}

%% \tableofcontents removido - sumário único no master

\section{O CNIS como Campo de Batalha Previdenciário}

\subsection{A Diferença Entre ``Consultar'' e ``Auditar''}

Você já viveu este cenário?

Cliente traz o CNIS impresso. 15 páginas de vínculos, salários, códigos estranhos. Você abre a primeira página, passa os olhos rapidamente... tudo parece ``normal''. Vínculos CLT listados. Salários registrados. Contribuições pagas.

Você conclui: \textit{``O CNIS está ok. Vamos calcular.''}

\textbf{E aí você perde R\$ 50.000,00 do seu cliente.}

Porque aquele CNIS que parecia ``ok'' tinha \textbf{3 indicadores críticos} na página 7 que você não viu. Um \textbf{PEXT} (vínculo extemporâneo) de 2 anos que estava sendo ignorado. Um \textbf{PREC-MENOR-MIN} que reduzia o PBC em R\$ 400/mês. Um \textbf{PDESFAZ-AJ-EC103} que bloquearia a concessão do benefício.

\textbf{A verdade brutal:} Consultar o CNIS é abrir o documento. Auditar o CNIS é \textbf{caçar erros}.

\begin{conceitoChave}
O Pilar 2 se divide em duas partes:

\textbf{PARTE 1 (Este Módulo):} Diagnóstico EXPLÍCITO --- Os erros que o INSS TE MOSTRA através de indicadores

\textbf{PARTE 2 (Módulo 3):} Diagnóstico IMPLÍCITO --- Os erros que o INSS NÃO mostra (Armadilhas Ocultas)
\end{conceitoChave}

\subsection{Por Que o CNIS É o Documento Mais Importante}

O CNIS (Cadastro Nacional de Informações Sociais) não é apenas ``um documento''. Ele é \textbf{O} documento.

\textbf{1. Presunção de Veracidade (Art. 19, Decreto 3.048/99)}

O CNIS tem presunção de veracidade. Isso significa que, para o INSS, \textbf{se está no CNIS, é verdade}. E se \textbf{não está no CNIS, não existe}.

Não importa se você tem CTPS, contracheques, FGTS, testemunhas. Se o vínculo não consta no CNIS, o INSS vai \textbf{ignorar}.

\textbf{2. Base de Cálculo para TUDO}

100\% dos cálculos de benefícios previdenciários partem do CNIS:
\begin{itemize}
    \item Tempo de contribuição → CNIS
    \item Salários para PBC → CNIS
    \item Carência → CNIS
    \item Qualidade de segurado → CNIS
    \item Atividades concomitantes → CNIS
    \item Afastamentos → CNIS
\end{itemize}

Se o CNIS está errado, o benefício estará errado.

\textbf{3. O INSS Não Investiga, Ele Processa}

O sistema do INSS não é ``inteligente''. Ele não cruza informações. Ele não verifica contradições. Ele \textbf{processa dados}.

\textbf{O CNIS é uma fotografia do que FOI DECLARADO, não do que ACONTECEU.}

\subsection{As ``Arritmias'' do Sistema: Erros Mais Comuns}

\textbf{Os 5 Tipos de Erro Mais Comuns:}

\textbf{1. Erros de Omissão (70\% dos casos)}
\begin{itemize}
    \item Vínculos que existiram mas não foram declarados
    \item Salários reais maiores que os informados
    \item Tempo especial não reconhecido por falta de PPP
    \item Tempo rural não cadastrado
    \item Tempo militar não averbado
\end{itemize}

\textbf{2. Erros de Digitação (15\% dos casos)}
\begin{itemize}
    \item CNPJ errado (um dígito trocado)
    \item Salários com zeros a menos (R\$ 150 ao invés de R\$ 1.500)
    \item Datas invertidas
\end{itemize}

\textbf{3. Erros de Classificação (8\% dos casos)}
\begin{itemize}
    \item Tempo especial classificado como comum
    \item Afastamento sem remuneração não descontado
\end{itemize}

\textbf{4. Erros de Sistema (5\% dos casos)}
\begin{itemize}
    \item Falhas de integração GFIP-CNIS
    \item Perda de dados em migração de sistemas
\end{itemize}

\textbf{5. Fraudes Empresariais (2\% dos casos)}
\begin{itemize}
    \item Subdeclaração deliberada de salários
    \item Não declaração de vínculos
\end{itemize}

\textbf{E como o INSS comunica esses erros?} Através de \textbf{INDICADORES}.

\section{Anatomia do Indicador CNIS}

\subsection{O Que É um Indicador?}

Um indicador CNIS é um \textbf{código alfanumérico} que o sistema anexa a um vínculo, salário ou período específico para sinalizar que \textbf{algo precisa de atenção}.

Pense no indicador como um \textbf{alerta médico}:
\begin{itemize}
    \item O CNIS é o eletrocardiograma
    \item Os vínculos são os batimentos cardíacos
    \item Os indicadores são as arritmias detectadas
\end{itemize}

\subsection{Estrutura do Código: Prefixo + Tipo + Complemento}

Todo indicador segue um padrão: \textbf{[PREFIXO] - [TIPO] - [COMPLEMENTO]}

\subsubsection{PREFIXOS (O que o indicador FAZ)}

\begin{table}[H]
\centering
\caption{Prefixos dos Indicadores CNIS}
\begin{tabular}{|c|l|l|}
\hline
\textbf{Prefixo} & \textbf{Significado} & \textbf{Urgência} \\
\hline
\textbf{P} & Pendência - Bloqueia ou prejudica & CRÍTICA \\
\hline
\textbf{I} & Informativo - Alerta, não bloqueia & MÉDIA \\
\hline
\textbf{A} & Acerto - Correção em andamento & BAIXA \\
\hline
\end{tabular}
\end{table}

\subsubsection{TIPOS (O que está ERRADO)}

\begin{table}[H]
\centering
\caption{Tipos dos Indicadores CNIS}
\begin{tabular}{|l|l|}
\hline
\textbf{Tipo} & \textbf{Significado} \\
\hline
EXT & Extemporâneo (fora do prazo) \\
\hline
REC & Recolhimento (problema com contribuição) \\
\hline
MENOR-MIN & Abaixo do mínimo \\
\hline
VIN & Vínculo (problema de cadastro) \\
\hline
EMP & Empresa (problema cadastral) \\
\hline
DESFAZ & Desfazimento (ajuste manual) \\
\hline
\end{tabular}
\end{table}

\subsubsection{COMPLEMENTOS (Contexto)}

\begin{table}[H]
\centering
\caption{Complementos dos Indicadores CNIS}
\begin{tabular}{|l|l|}
\hline
\textbf{Complemento} & \textbf{Significado} \\
\hline
-EC103 & Relacionado à Reforma (pós-13/11/2019) \\
\hline
-AJ & Aguardando ajuste \\
\hline
-CAD & Problema de cadastro \\
\hline
-PROC-TRAB & Processo trabalhista \\
\hline
\end{tabular}
\end{table}

\subsection{Classificação Estratégica}

\begin{estrategiaCJP}
\textbf{GRUPO 1: PENDÊNCIAS CRÍTICAS}\\
Bloqueiam ou prejudicam diretamente o benefício.\\
Prioridade máxima. Resolução ANTES de pedir benefício.\\
\textit{Exemplos: PEXT, PSC-MEN-SM-EC103, PDESFAZ-AJ-EC103}

\textbf{GRUPO 2: INDICADORES INFORMATIVOS}\\
Não bloqueiam, mas sinalizam oportunidades ou riscos.\\
Prioridade média. Avaliar impacto no caso concreto.\\
\textit{Exemplos: IEAN (Especial Não Convertida), ICAR}

\textbf{GRUPO 3: ACERTOS E AJUSTES}\\
Correções em andamento. Monitorar progresso.\\
Prioridade baixa. Não exige ação imediata.\\
\textit{Exemplos: ACNIS (Acerto solicitado), AHOM (Homologado)}
\end{estrategiaCJP}

\subsection{A Linguagem Secreta do INSS Decodificada}

\begin{table}[H]
\centering
\caption{A Linguagem Secreta do INSS Decodificada}
\begin{tabular}{|l|p{8cm}|}
\hline
\textbf{Indicador} & \textbf{O Que o INSS Está Dizendo} \\
\hline
PEXT & ``Este vínculo pode ser falso. Prove que é verdadeiro.'' \\
\hline
PREC-MENOR-MIN & ``Este salário está errado. Era pra ser maior.'' \\
\hline
PSC-MEN-SM-EC103 & ``Esta contribuição não conta. Complemente agora.'' \\
\hline
PDESFAZ-AJ-EC103 & ``Esse caso é complexo demais. Preciso de um humano.'' \\
\hline
PEMP-CAD & ``Nunca ouvi falar dessa empresa. Ela existe mesmo?'' \\
\hline
PVIN-REC-PROC-TRAB & ``Tem processo trabalhista? Cadê a sentença?'' \\
\hline
IEAN & ``Vi que você trabalhou exposto a agente nocivo, mas não vou converter automaticamente.'' \\
\hline
\end{tabular}
\end{table}

\begin{conceitoChave}
\textbf{A grande sacada:} Quando você aprende a ``ler'' os indicadores, o CNIS deixa de ser um mistério e vira um \textbf{mapa do tesouro}.

Cada indicador é uma pista do que está faltando, do que está errado, e --- mais importante --- \textbf{de quanto dinheiro você pode recuperar para o cliente}.
\end{conceitoChave}

\section{OS 15 INDICADORES PRIORITÁRIOS}

\subsection{GRUPO 1: Pendências Críticas (7 Indicadores)}

\subsubsection{2.3.1 PEXT - Vínculo Extemporâneo}

\begin{armadilha}
O PEXT é o indicador mais comum (40\% dos CNIS têm pelo menos um) e mais perigoso. Ignorá-lo significa perder anos de tempo de contribuição.

Um único PEXT não resolvido pode atrasar aposentadoria em 2-5 anos.
\end{armadilha}

\textbf{O QUE É:}

PEXT significa ``Pendência de Vínculo Extemporâneo''. É o indicador que aparece quando uma empresa informou um vínculo de trabalho \textbf{fora do prazo legal} de cadastro no CNIS.

Traduzindo: O empregador disse ``fulano trabalhou aqui de 2010 a 2015'', mas só informou isso ao INSS em 2023. O sistema marca como ``pendente de confirmação''.

\textbf{POR QUE OCORRE:}
\begin{enumerate}
    \item Empresa faliu ou fechou sem transmitir GFIP completa
    \item Informação prestada tardiamente
    \item Divergências entre RAIS e CAGED
    \item Empresa ``esqueceu'' de declarar períodos antigos
    \item Requerimento administrativo do segurado anos depois
\end{enumerate}

\textbf{IMPACTO NO BENEFÍCIO:}
\begin{itemize}
    \item[\xmark] Tempo não conta → O período PEXT é ignorado no cálculo
    \item[\xmark] Salários não entram no PBC
    \item[\xmark] Carência não aumenta
    \item[\xmark] Pode inviabilizar o benefício
\end{itemize}

\textbf{Exemplo Prático:}

\textbf{Cliente:} José, 58 anos\\
\textbf{Tempo no CNIS:} 32 anos e 4 meses\\
\textbf{PEXT identificado:} 3 anos na Empresa ABC (2005-2008)

\textbf{CENÁRIO 1 (Se ignorar o PEXT):}
\begin{itemize}
    \item José não tem tempo suficiente para Aposentadoria por Pontos
    \item Precisaria esperar mais 3 anos
    \item RMI estimada: R\$ 3.500,00 (em 2029)
\end{itemize}

\textbf{CENÁRIO 2 (Se resolver o PEXT):}
\begin{itemize}
    \item José atinge 35 anos de contribuição AGORA
    \item Pode se aposentar IMEDIATAMENTE (103 pontos em 2026)
    \item RMI: R\$ 4.500,00 (hoje)
    \item \textbf{Ganho vitalício:} \textasciitilde R\$ 240.000,00
\end{itemize}

\textbf{ESTRATÉGIA DE RESOLUÇÃO:}

\textbf{Via Administrativa:}
\begin{enumerate}
    \item Acesse Meu INSS → Serviços → Acerto de Vínculo (RAC)
    \item Preencha o formulário com dados completos do vínculo
    \item Anexe documentação (ver lista abaixo)
    \item Protocole o requerimento
    \item Aguarde análise: 45 dias úteis
    \item Acompanhe o processo via Meu INSS
\end{enumerate}

\textbf{Documentos Necessários:}
\begin{itemize}
    \item[$\square$] CTPS (páginas de qualificação + contrato + baixa)
    \item[$\square$] Mínimo 3 contracheques por ano do período
    \item[$\square$] TRCT (Termo de Rescisão)
    \item[$\square$] Extrato Analítico do FGTS
    \item[$\square$] Declarações de IR (se tiver)
    \item[$\square$] Correspondências/crachás (prova complementar)
\end{itemize}

\begin{acaoImediata}
Se o PEXT for recusado administrativamente por ``documentação insuficiente'', NÃO desista. Vá direto para via judicial com TUDO o que tiver.

A jurisprudência é MUITO favorável a CTPS + Contracheques + FGTS.
\end{acaoImediata}

\begin{novidade}
\textbf{Tema 1124 STJ (Outubro/2025)}

Ao protocolar RAC para PEXT, certifique-se de apresentar DOCUMENTAÇÃO COMPLETA no requerimento administrativo.

\begin{itemize}
    \item DIB na DER se provas já estavam no PA
    \item DIB na citação se provas surgiram apenas em juízo
\end{itemize}

\textbf{ESTRATÉGIA CJP:} Instrua o PA completamente ANTES de eventual judicialização para garantir DIB na DER.
\end{novidade}

\textbf{Via Judicial:}
\begin{itemize}
    \item \textbf{Ação:} Ação Declaratória de Tempo de Contribuição
    \item \textbf{Competência:} JEF (até 60 SM) ou Vara Federal Previdenciária
    \item \textbf{Fundamento:} Art. 55, §3º da Lei 8.213/91
    \item \textbf{Provas:} CTPS + Contracheques + FGTS + Testemunhas
    \item \textbf{Jurisprudência:} Súmula 75 TNU
\end{itemize}

\begin{teseRevisional}
Se o cliente JÁ se aposentou com PEXT não resolvido, você tem uma REVISÃO DE BENEFÍCIO garantida.

Prazo decadencial: 10 anos da concessão (Tema 1.004 STJ)
\end{teseRevisional}

\subsubsection{2.3.2 PREC-MENOR-MIN - Remuneração Abaixo do Mínimo}

\begin{teseRevisional}
Salários abaixo do mínimo constitucional são ILEGAIS (CF/88, Art. 7º, IV).

Correção pode aumentar PBC em 15-30\% e RMI em R\$ 200-500/mês.
\end{teseRevisional}

\textbf{O QUE É:}

PREC-MENOR-MIN significa ``Pendência de Recolhimento - Menor que o Salário Mínimo''. Aparece quando uma remuneração cadastrada está \textbf{abaixo do salário mínimo} vigente naquele mês/ano.

\textbf{POR QUE OCORRE:}
\begin{enumerate}
    \item Erro de digitação na GFIP
    \item Trabalho parcial declarado incorretamente
    \item Fraude empresarial
    \item Sistema não corrigiu valores antigos
    \item Salários proporcionais (admissão/demissão no meio do mês)
\end{enumerate}

\textbf{IMPACTO NO BENEFÍCIO:}
\begin{itemize}
    \item[\xmark] Reduz artificialmente o PBC
    \item[\xmark] Pode impedir carência
    \item[\xmark] Prejudica RMI diretamente
\end{itemize}

\textbf{ESTRATÉGIA DE RESOLUÇÃO:}

\textbf{Via Administrativa:}
\begin{enumerate}
    \item Acesse Meu INSS → Serviços → Revisão de Remunerações
    \item Selecione o período com PREC-MENOR-MIN
    \item Anexe provas do salário real
    \item Protocole e aguarde: 30-45 dias úteis
\end{enumerate}

\textbf{Documentos:}
\begin{itemize}
    \item[$\square$] Contracheques originais do período
    \item[$\square$] Acordo/Convenção Coletiva da categoria
    \item[$\square$] Extratos bancários comprovando depósitos
    \item[$\square$] Declarações de IR
\end{itemize}

\begin{acaoImediata}
Se o PREC-MENOR-MIN for de período anterior a 1994 (antes do Real), use tabelas oficiais de conversão de moedas antigas.

Muitos indicadores são FALSOS por erro de conversão monetária.
\end{acaoImediata}

\subsubsection{2.3.3 PSC-MEN-SM-EC103 - Pós-Reforma Abaixo do Mínimo}

\begin{acaoImediata}
Este indicador BLOQUEIA benefícios após 13/11/2019. Correção URGENTE.

Contribuições pós-reforma \textless\ mínimo NÃO contam para NADA (nem carência, nem tempo, nem PBC).
\end{acaoImediata}

\textbf{O QUE É:}

PSC-MEN-SM-EC103 é o indicador específico para contribuições \textbf{após 13/11/2019} que estão \textbf{abaixo do salário mínimo}.

É similar ao PREC-MENOR-MIN, mas \textbf{muito mais grave} porque a EC 103/2019 criou regras mais rígidas.

\begin{conceitoChave}
\textbf{Antes da EC 103/2019:}
Contribuição mínima: qualquer valor acima de 5\% do SM

\textbf{Depois da EC 103/2019:}
Contribuição mínima: 7,5\% do SM (empregado) ou 20\% (autônomo)
Valor mínimo 2026: R\$ 1.621,00 (confirmado)

\textbf{EXCEÇÕES:}
\begin{itemize}
    \item MEI: 5\% do SM (R\$ 81,05/2026)
    \item Facultativo Baixa Renda: 5\% do SM (com restrições)
\end{itemize}
\end{conceitoChave}

\textbf{ESTRATÉGIA DE RESOLUÇÃO:}

\textbf{Via Administrativa (OBRIGATÓRIA):}
\begin{enumerate}
    \item Calcule a diferença de cada mês
    \item Gere GPS de complementação via Meu INSS
    \item Pague as diferenças com juros e multa (SELIC)
    \item Protocole comprovantes no Meu INSS
    \item Aguarde atualização do CNIS: 15-30 dias
\end{enumerate}

\begin{armadilha}
NÃO tente ``discutir'' a regra dos 7,5\%/20\% mínimos. A jurisprudência é pacífica: É constitucional (STF).

A ÚNICA discussão cabível é o prazo para regularização e a forma de cobrança (juros/multa).
\end{armadilha}

\subsubsection{2.3.4 PDESFAZ-AJ-EC103 - O Bloqueador Master}

\begin{armadilha}
Este é o indicador mais TEMIDO. Ele CONGELA o benefício até correção MANUAL por servidor do INSS.

Casos com PDESFAZ podem levar 6-18 meses para serem resolvidos administrativamente.

\textbf{ESTRATÉGIA:} Ir direto para judicial com cálculos PRONTOS.
\end{armadilha}

\textbf{O QUE É:}

PDESFAZ-AJ-EC103 significa ``Pendência de Desfazimento - Aguardando Ajuste - EC 103/2019''. Aparece quando o sistema do INSS \textbf{não consegue} processar o caso automaticamente.

Traduzindo: O computador desistiu. Precisa de um humano.

\textbf{QUANDO APARECE:}
\begin{enumerate}
    \item Múltiplas regras de transição aplicáveis
    \item Atividades concomitantes em períodos críticos
    \item Conversões de tempo especial pós-13/11/2019
    \item Divergências entre sistemas
    \item Casos híbridos (parte pré, parte pós-reforma)
    \item Contribuinte que foi empregado, autônomo, facultativo, MEI em sequência
\end{enumerate}

\textbf{POR QUE É PERIGOSO:}
\begin{itemize}
    \item Paralisa a concessão completamente
    \item Fila de espera para análise manual: 4-12 meses
    \item Pode gerar indeferimento por ``impossibilidade técnica''
\end{itemize}

\begin{estrategiaCJP}
\textbf{ESTRATÉGIA CJP PARA PDESFAZ:}

\begin{enumerate}
    \item Faça os cálculos VOCÊ MESMO (todas as regras possíveis)
    \item Apresente ao INSS com memorial de cálculo detalhado
    \item Indique qual regra é mais vantajosa
    \item Se negar: via judicial com tutela de urgência
\end{enumerate}

O INSS não consegue calcular? Calcule você. E cobre por isso.
\end{estrategiaCJP}

%% Continua na Parte 2B
