
%% Continuação do Módulo 7: Benefícios Não Programáveis

\begin{center}
{\Large\textit{Benefícios Não Programáveis: Auxílio Temporário, AIP e Checklists}}\\[0.5cm]
\textbf{Sistema CJP | Módulo 7}
\end{center}

\section{Auxílio por Incapacidade Temporária}

\subsection{O Que É o Auxílio Temporário?}

O \textbf{Auxílio por Incapacidade Temporária} (antigo ``auxílio-doença'') é o benefício pago ao segurado que fica \textbf{temporariamente incapaz} de trabalhar por doença ou acidente.

\textbf{Características principais:}
\begin{itemize}
    \item Benefício \textbf{temporário} (até recuperação)
    \item Carência: \textbf{12 meses} (regra geral)
    \item Empresa paga os \textbf{primeiros 15 dias}
    \item INSS paga a partir do \textbf{16\textsuperscript{o} dia}
    \item Exige \textbf{perícia médica} para concessão e prorrogação
\end{itemize}

\subsection{Cálculo do SB com ``Subteto''}

O cálculo do Auxílio por Incapacidade Temporária usa uma regra especial: o \textbf{``Subteto''}.

\begin{conceitoChave}
\textbf{REGRA DO SUBTETO}

O Salário de Benefício (SB) do Auxílio é o \textbf{MENOR} entre:
\begin{itemize}
    \item \textbf{Média A:} Media de TODOS os salários do PBC (Jul/1994 $\rightarrow$ DER)
    \item \textbf{Média B:} Média dos 12 ÚLTIMOS salários de contribuição
\end{itemize}

$$SB = MIN(M\acute{e}dia\ A,\ M\acute{e}dia\ B)$$
\end{conceitoChave}

\textbf{Por que isso existe?}

Para evitar que segurados que contribuíram pouco ao longo da vida façam ``contribuições de última hora'' altas para aumentar o benefício.

\subsection{Cálculo da RMI}

$$RMI = SB \times 91\%$$

\textbf{Por que 91\%?}

\begin{itemize}
    \item Para desincentivar fraudes (se fosse 100\%, haveria incentivo para ``ficar doente'')
    \item Para equilibrar custo previdenciário
    \item Para manter diferença vs. aposentadoria
\end{itemize}

É uma ``punição simbólica'' que na prática representa apenas -9\%.

\subsection{Exemplos Práticos}

\subsubsection{Caso A: Afastamento Simples}

\begin{estrategiaCJP}
\textbf{SEGURADA: Marta (Professora CLT)}

\textbf{SITUAÇÃO:}
\begin{itemize}
    \item Fraturou a perna em acidente
    \item Ficará afastada 90 dias (3 meses)
    \item Salário atual: R\$ 4.500/mês
\end{itemize}

\textbf{CÁLCULO:}
\begin{verbatim}
Média histórica (15 anos): R\$ 3.800,00
Média 12 últimos: R\$ 4.400,00
MENOR: R\$ 3.800,00
SB: R\$ 3.800,00
RMI: R\$ 3.800,00 x 91% = R\$ 3.458,00
\end{verbatim}

\textbf{PERÍODO:}
\begin{itemize}
    \item Primeiros 15 dias: Empresa paga salário integral
    \item A partir do 16\textsuperscript{o} dia: INSS paga R\$ 3.458,00
    \item Total INSS: 75 dias
\end{itemize}
\end{estrategiaCJP}

\subsubsection{Caso B: Afastamento Longo com Prorrogações}

\begin{verbatim}
SEGURADO: João (Metalúrgico)

SITUAÇÃO:
• Hérnia de disco (cirurgia + fisioterapia)
• Afastamento inicial: 60 dias
• Prorrogação 1: +60 dias
• Prorrogação 2: +60 dias
• Total: 180 dias (6 meses)

CÁLCULO:
• Média histórica (20 anos): R\$ 5.200,00
• Média 12 últimos: R\$ 5.800,00
• MENOR: R\$ 5.200,00
• SB: R\$ 5.200,00
• RMI: R\$ 5.200,00 x 91% = R\$ 4.732,00

PERÍCIA:
• Inicial (30 dias): Concedido 60 dias
• Perícia 1 (60 dias): Prorrogado +60
• Perícia 2 (120 dias): Prorrogado +60
• Perícia 3 (180 dias): ALTA MÉDICA (apto)

VALOR TOTAL RECEBIDO:
• R\$ 4.732,00 x 6 = R\$ 28.392,00
\end{verbatim}

\subsection{Armadilhas Comuns}

\begin{armadilha}
\textbf{ARMADILHA \#1: Calcular SB Errado (Não Usar o Menor)}

\textbf{Problema:} INSS às vezes erra e usa só a média dos 12 últimos (favorecendo quem contribuiu alto recentemente)

\textbf{Impacto:} SB maior do que deveria

\textbf{Solução:} Se for favorável ao segurado, não corrigir! Se prejudicar, recalcular e revisar.
\end{armadilha}

\begin{armadilha}
\textbf{ARMADILHA \#2: Confundir com Aposentadoria por Invalidez}

\textbf{Problema:} Cliente acha que vai receber para sempre

\textbf{Impacto:} Frustração quando recebe alta médica

\textbf{Solução:} Explicar CLARAMENTE que é benefício TEMPORÁRIO
\end{armadilha}

\begin{armadilha}
\textbf{ARMADILHA \#3: Empresa Não Pagar os 15 Primeiros Dias}

\textbf{Problema:} Empresa alega que é responsabilidade do INSS

\textbf{Impacto:} Trabalhador fica 15 dias sem receber nada

\textbf{Solução:} Art. 60, \S 3\textsuperscript{o}, Lei 8.213/91 $\rightarrow$ Empresa DEVE pagar (se não pagar, ação trabalhista)
\end{armadilha}

\subsection{Checklist de Auditoria Auxílio Temporário}

\begin{acaoImediata}
\textbf{CHECKLIST --- AUXÍLIO TEMPORÁRIO}

\textbf{REQUISITOS BÁSICOS}
\begin{itemize}
    \item[$\square$] Tem 12 meses de carência?
    \item[$\square$] Está incapaz temporariamente?
    \item[$\square$] Passou por perícia médica?
\end{itemize}

\textbf{CÁLCULO DO SB}
\begin{itemize}
    \item[$\square$] Média histórica (PBC) calculada?
    \item[$\square$] Média dos 12 últimos calculada?
    \item[$\square$] Usou o MENOR entre as duas?
\end{itemize}

\textbf{CÁLCULO DA RMI}
\begin{itemize}
    \item[$\square$] RMI = SB $\times$ 91\%?
    \item[$\square$] Valor confere?
\end{itemize}

\textbf{PERÍODO}
\begin{itemize}
    \item[$\square$] Primeiros 15 dias pagos pela empresa?
    \item[$\square$] INSS paga a partir do 16\textsuperscript{o} dia?
    \item[$\square$] Perícia está agendada?
\end{itemize}

\textbf{DOCUMENTAÇÃO}
\begin{itemize}
    \item[$\square$] Atestado médico
    \item[$\square$] Exames comprobatórios
    \item[$\square$] Carta de concessão (se já pediu)
    \item[$\square$] Agendamento de perícia
\end{itemize}
\end{acaoImediata}

\section{Tema 1300 STF --- Tese REJEITADA (Registro Histórico)}

\begin{novidade}
\textbf{RESULTADO FINAL DO JULGAMENTO (Dez/2025)}

\textbf{PLACAR: 6 $\times$ 5 pela CONSTITUCIONALIDADE (Tese REJEITADA)}

\textbf{PELA CONSTITUCIONALIDADE (vencedora):}
\begin{itemize}
    \item Luís Roberto Barroso (relator)
    \item Cristiano Zanin
    \item André Mendonça
    \item Nunes Marques
    \item Gilmar Mendes
    \item Luiz Fux
\end{itemize}

\textbf{PELA INCONSTITUCIONALIDADE (vencida):}
\begin{itemize}
    \item Flávio Dino
    \item Edson Fachin
    \item Alexandre de Moraes
    \item Dias Toffoli
    \item Cármen Lúcia
\end{itemize}

\textbf{RESULTADO:} Regra de 60\%+2\% PERMANECE VIGENTE
\end{novidade}

\subsection{O ``Paradoxo da Incapacidade'' --- Mantido pela Decisão}

O Min. Flávio Dino, em voto vencido, destacou o problema que permanece:

\begin{armadilha}
\textbf{REGRA MANTIDA (EC 103/2019):}

\begin{itemize}
    \item Auxílio por Incapacidade Temporária: 91\% da média
    \item Aposentadoria por Incap. Permanente (comum): 60\% + 2\%
\end{itemize}

\textbf{O ``PARADOXO'' PERMANECE:}

A piora clínica ainda pode gerar REDUÇÃO de renda. Um segurado pode ter AIT de R\$ 4.550,00 e, ao piorar para incapacidade permanente, receber AIP de apenas R\$ 3.000,00.

\textbf{ESTRATÉGIA ÚNICA PARA 100\%:} Demonstrar NEXO CAUSAL (converter para B92 --- AIP Acidentária)
\end{armadilha}

\subsection{Tabela de Coeficientes Vigente}

\begin{center}
\begin{tabular}{|l|l|}
\hline
\textbf{Tipo de AIP} & \textbf{Coeficiente VIGENTE} \\
\hline
AIP Acidentária (B92) & 100\% \\
\hline
AIP Comum --- não acidentária (B32) & 60\% + 2\% (mín. 60\%) \\
\hline
AIP com DII antes 13/11/2019 & 100\% (direito adquirido) \\
\hline
\end{tabular}
\end{center}

\textbf{ATENÇÃO:} A tese de 100\% para AIP comum foi REJEITADA. NÃO há revisão disponível com base no Tema 1300.

\subsection{Estratégia Alternativa: Nexo Causal (B32 $\rightarrow$ B92)}

Uma das \textbf{vias para obter 100\%} é demonstrar origem ocupacional da doença (a outra é a tese da DII Única, tratada a seguir).

\begin{teseRevisional}
\textbf{CONVERSÃO DE ESPÉCIE}

\textbf{OBJETIVO:}
Converter AIP comum (B32) em AIP acidentária (B92) = 100\%

\textbf{FERRAMENTAS:}
\begin{itemize}
    \item NTEP (presunção legal CID + CNAE)
    \item CAT emitida
    \item Prova pericial técnica
\end{itemize}

\textbf{DOENÇAS COM POTENCIAL:}
\begin{itemize}
    \item Dorsalgias (M54) --- Construção
    \item LER/DORT (M65) --- Bancos
    \item Depressão (F32) --- Telemarketing
\end{itemize}
\end{teseRevisional}

\begin{acaoImediata}
\textbf{CHECKLIST --- NEXO CAUSAL}

\begin{itemize}
    \item[$\square$] Cliente tem AIP comum (B32)?
    \item[$\square$] Coeficiente $<$ 100\%?
    \item[$\square$] Doença pode ter relação laboral?
    \item[$\square$] Verificar NTEP (CID + CNAE)
    \item[$\square$] Existe CAT?
    \item[$\square$] PPP indica riscos?
\end{itemize}

\textbf{SE NTEP:} Requerer conversão administrativamente\\
\textbf{SE NÃO:} Ação judicial com perícia
\end{acaoImediata}

\subsection{Estratégia Alternativa: DII Única (Conversão com Direito Adquirido)}

Além do nexo causal, existe \textbf{outra via} para manter o coeficiente de 100\%: a tese da \textbf{DII Única}.

\begin{teseRevisional}
\textbf{TESE DA DII ÚNICA --- Conversão Auxílio $\rightarrow$ AIP}

\textbf{O que é:} Se o segurado recebeu Auxílio por Incapacidade Temporária com DII \textbf{anterior a 13/11/2019} e posteriormente foi convertido em Aposentadoria por Incapacidade Permanente, o cálculo deve seguir as \textbf{regras anteriores à Reforma} (100\% da média dos 80\% maiores salários).

\textbf{Fundamento Jurídico:}
\begin{itemize}
    \item Princípio \textit{tempus regit actum} (a lei do fato gerador se aplica)
    \item O fato gerador é a DII original, não a data da conversão
    \item Direito adquirido ao cálculo vigente na DII
\end{itemize}

\textbf{Base Normativa:}
\begin{itemize}
    \item \textbf{ACP n\textsuperscript{o} 5020446-70.2023.4.02.5001/ES} (DPU/ES)
    \item \textbf{Portaria Conjunta DIRBEN/PFE/INSS n\textsuperscript{o} 87/2023}
\end{itemize}

\textbf{Impacto:} +30\% a +40\% na RMI (de 60\% para 100\%)

\textbf{Aplicável a:} Conversões com DII anterior a 13/11/2019
\end{teseRevisional}

\begin{armadilha}
\textbf{ATENÇÃO: DECISÃO LIMINAR (NÃO DEFINITIVA)}

A ACP 5020446-70 possui \textbf{tutela de urgência deferida e vigente}, mas ainda não há sentença de mérito transitada em julgado.

\textbf{Estado atual (Jan/2026):}
\begin{itemize}
    \item Liminar concedida pela 2\textsuperscript{a} Vara Federal de Vitória/ES
    \item INSS interpôs Agravo de Instrumento no TRF-2
    \item \textbf{Efeito suspensivo NEGADO} $\rightarrow$ Liminar permanece vigente
    \item Processo em trâmite regular rumo à sentença
\end{itemize}

\textbf{A Portaria 87/2023:}
\begin{itemize}
    \item Está \textbf{VIGENTE} e produzindo efeitos
    \item Suspende cobranças de diferenças (Art. 2\textsuperscript{o})
    \item Fundamenta pedidos de revisão administrativos
\end{itemize}

\textbf{RECOMENDAÇÃO:} Use a tese com segurança para casos enquadrados (DII $<$ 13/11/2019), mas informe o cliente sobre a natureza provisória da decisão judicial.
\end{armadilha}

\begin{acaoImediata}
\textbf{CHECKLIST --- DII ÚNICA}

\begin{itemize}
    \item[$\square$] Cliente recebe AIP convertida de Auxílio?
    \item[$\square$] DII original anterior a 13/11/2019?
    \item[$\square$] Coeficiente atual $<$ 100\%?
    \item[$\square$] INSS aplicou regra pós-reforma?
    \item[$\square$] Há descontos/cobrança de diferenças?
\end{itemize}

\textbf{SE TODOS SIM:}
\begin{enumerate}
    \item Protocolar Revisão Administrativa citando:
    \begin{itemize}
        \item ACP 5020446-70.2023.4.02.5001/ES
        \item Portaria Conjunta 87/2023
        \item Tema 1300 STF (\textit{a contrario sensu})
    \end{itemize}
    \item Se indeferido ou demorado: Ação individual
\end{enumerate}
\end{acaoImediata}

\section{Diferenças Estruturais: Aposentadorias vs. Não Programáveis}

\begin{center}
\begin{tabular}{|p{2.5cm}|p{3cm}|p{5.5cm}|}
\hline
\textbf{Aspecto} & \textbf{Aposentadoria} & \textbf{Pensão/Sal-Mat/Auxílio} \\
\hline
Planejamento & POSSÍVEL (escolhe quando) & IMPOSSÍVEL (evento externo) \\
\hline
Cálculo SB & PBC 80\% ou 100\% + Divisor Mínimo & Varia por tipo \\
\hline
Coeficiente & 60\% + 2\% por ano & Varia (50\%+10\%/91\%/100\%) \\
\hline
Duração & VITALÍCIO & Varia (vitalício/dias/até recuperação) \\
\hline
Carência & 180 meses & Zero a 12 meses \\
\hline
Fator Prev. & Só Pedágio 50\%/Dir. Adq. & NUNCA \\
\hline
Divisor Mín. & 108 meses (pós-05/2022) & NÃO se aplica \\
\hline
\end{tabular}
\end{center}

\begin{armadilha}
\textbf{ERRO \#1: Aplicar Divisor Mínimo em Pensão}

\textbf{ERRADO:} ``A pensão foi concedida em 2023, então o divisor mínimo de 108 se aplica.''

\textbf{CORRETO:} Divisor mínimo é EXCLUSIVO de aposentadorias. Pensão usa a base da aposentadoria do falecido (se ele era aposentado) ou 100\% da média (se não).
\end{armadilha}

\begin{armadilha}
\textbf{ERRO \#2: Aplicar Coeficiente 60\%+2\% em Pensão}

\textbf{ERRADO:} ``A viúva tem 35 anos de contribuição, então o coeficiente da pensão é 60\% + 30\% = 90\%.''

\textbf{CORRETO:} O coeficiente da pensão depende do NÚMERO DE DEPENDENTES, não do tempo de contribuição. Fórmula: 50\% + 10\% por dependente.
\end{armadilha}

\section{Fluxograma de Decisão: Qual Benefício?}

Use este fluxograma para identificar rapidamente qual benefício não programável se aplica em cada situação:

\begin{estrategiaCJP}
\textbf{ÁRVORE DE DECISÃO INICIAL}

\textbf{SITUAÇÃO DO SEGURADO:}
\begin{itemize}
    \item Segurado \textbf{FALECEU}? $\rightarrow$ \textbf{PENSÃO POR MORTE}
    \item Segurada \textbf{DEU À LUZ / ADOTOU}? $\rightarrow$ \textbf{SALÁRIO-MATERNIDADE}
    \item Segurado \textbf{INCAPACITADO}? $\rightarrow$ Verificar tipo de incapacidade
    \begin{itemize}
        \item Incapacidade \textbf{TEMPORÁRIA} $\rightarrow$ \textbf{Auxílio por Incapacidade Temporária}
        \item Incapacidade \textbf{PERMANENTE} $\rightarrow$ \textbf{Aposentadoria por Invalidez}
    \end{itemize}
\end{itemize}
\end{estrategiaCJP}

\begin{conceitoChave}
\textbf{DECISÃO: PENSÃO POR MORTE}

\textbf{Data do óbito $<$ 13/11/2019?}
\begin{itemize}
    \item \textbf{SIM} $\rightarrow$ Regra antiga (100\% rateado entre dependentes)
    \item \textbf{NÃO} $\rightarrow$ Regra EC 103/2019:
    \begin{itemize}
        \item Há dependente inválido/deficiente grave?
        \begin{itemize}
            \item SIM $\rightarrow$ Coeficiente = \textbf{100\%}
            \item NÃO $\rightarrow$ Coeficiente = \textbf{50\% + 10\% por dependente}
        \end{itemize}
        \item Óbito após 13/03/2025?
        \begin{itemize}
            \item Menor sob guarda judicial $\rightarrow$ \textbf{CONTA como dependente} (Lei 15.108)
        \end{itemize}
    \end{itemize}
\end{itemize}
\end{conceitoChave}

\begin{conceitoChave}
\textbf{DECISÃO: SALÁRIO-MATERNIDADE}

\textbf{Categoria da segurada?}
\begin{itemize}
    \item \textbf{CLT} $\rightarrow$ RMI = Salário integral (ou média 6 meses se variável)
    \item \textbf{Doméstica} $\rightarrow$ RMI = Último salário-de-contribuição
    \item \textbf{Autônoma/MEI} $\rightarrow$ RMI = Média 12 últimas contribuições $\div$ 12
    \item \textbf{Especial Rural} $\rightarrow$ RMI = 1 salário mínimo (R\$ 1.518)
\end{itemize}

\textbf{Pedido após 05/04/2024?}
\begin{itemize}
    \item Carência = \textbf{ZERO} (IN 188/2025)
\end{itemize}

\textbf{Internação $>$ 14 dias?}
\begin{itemize}
    \item Prorrogação: Internação + 120 dias (Lei 15.222/2025)
\end{itemize}
\end{conceitoChave}

\begin{conceitoChave}
\textbf{DECISÃO: INCAPACIDADE}

\textbf{Perícia médica atesta:}

\textbf{Incapacidade TEMPORÁRIA:}
\begin{itemize}
    \item $\rightarrow$ Auxílio por Incapacidade Temporária
    \item SB = MENOR entre (Média 100\% PBC, Média 12 últimos)
    \item RMI = SB $\times$ 91\%
    \item Duração: Até alta médica
\end{itemize}

\textbf{Incapacidade PERMANENTE/TOTAL:}
\begin{itemize}
    \item $\rightarrow$ Aposentadoria por Invalidez
    \item Cálculo igual a aposentadoria comum
    \item Exceção: Se acidente de trabalho ou doença grave da lista $\rightarrow$ \textbf{100\%}
\end{itemize}
\end{conceitoChave}

\section{Checklist Executivo Unificado}

\begin{acaoImediata}
\textbf{CHECKLIST MASTER --- TODOS OS BENEFÍCIOS NÃO PROGRAMÁVEIS}

\textbf{1. IDENTIFICAÇÃO DO BENEFÍCIO}
\begin{itemize}
    \item[$\square$] Qual benefício? (Pensão / Sal-Mat / Auxílio)
    \item[$\square$] Data do evento (óbito / parto / afastamento)
    \item[$\square$] DER (Data de Entrada do Requerimento)
    \item[$\square$] DIB (Data de Início do Benefício)
\end{itemize}

\textbf{2. PENSÃO POR MORTE}
\begin{itemize}
    \item[$\square$] Data do óbito: \_\_\_\_\_\_
    \item[$\square$] Regra aplicável: ( ) Pré-Reforma ( ) Pós-Reforma
    \item[$\square$] Falecido era aposentado? ( ) Sim ( ) Não
    \item[$\square$] Número de dependentes: \_\_\_
    \item[$\square$] Menor sob guarda? (Lei 15.108) ( ) Sim ( ) Não
    \item[$\square$] Dependente deficiente? ( ) Sim ( ) Não
    \item[$\square$] Coeficiente calculado: \_\_\_\_\%
    \item[$\square$] RMI final: R\$ \_\_\_\_\_\_
\end{itemize}

\textbf{3. SALÁRIO-MATERNIDADE}
\begin{itemize}
    \item[$\square$] Data do parto/adoção: \_\_\_\_\_\_
    \item[$\square$] Categoria: ( ) CLT ( ) Doméstica ( ) Autônoma ( ) Especial
    \item[$\square$] RMI calculada: R\$ \_\_\_\_\_\_
    \item[$\square$] Carência isenta? (IN 188) ( ) Sim ( ) Não
    \item[$\square$] Internação $>$ 14 dias? ( ) Sim ( ) Não
    \item[$\square$] Período prorrogado? ( ) Sim ( ) Não
\end{itemize}

\textbf{4. AUXÍLIO INCAPACIDADE TEMPORÁRIA}
\begin{itemize}
    \item[$\square$] Data do afastamento: \_\_\_\_\_\_
    \item[$\square$] Média histórica (PBC): R\$ \_\_\_\_\_\_
    \item[$\square$] Média 12 últimos: R\$ \_\_\_\_\_\_
    \item[$\square$] SB (MENOR): R\$ \_\_\_\_\_\_
    \item[$\square$] RMI (91\% do SB): R\$ \_\_\_\_\_\_
    \item[$\square$] Empresa pagou 15 dias? ( ) Sim ( ) Não
    \item[$\square$] Perícia agendada para: \_\_\_\_\_\_
\end{itemize}

\textbf{5. TESES REVISIONAIS APLICÁVEIS (2025-2026)}
\begin{itemize}
    \item[$\square$] Pensão: Inclusão menor sob guarda (Lei 15.108)
    \item[$\square$] Pensão: Deficiente = 100\% (EC 103, Art. 23, \S 2\textsuperscript{o})
    \item[$\square$] Sal-Mat: Isenção carência (IN 188/2025)
    \item[$\square$] Sal-Mat: Prorrogação internação (Lei 15.222)
    \item[$\square$] AIP: Conversão B32 $\rightarrow$ B92 (Nexo Causal)
\end{itemize}
\end{acaoImediata}

\section{Resumo do Módulo 7}

\textbf{O Que Você Aprendeu:}

\begin{itemize}
    \item[\cmark] \textbf{Pensão por Morte:} Regras pré e pós-Reforma, coeficiente por dependentes, Lei 15.108/2025, exceção deficiente = 100\%
    \item[\cmark] \textbf{Salário-Maternidade:} Cálculo por categoria, IN 188/2025 (zero carência), Lei 15.222/2025 (prorrogação)
    \item[\cmark] \textbf{Auxílio Temporário:} Subteto, 91\%, armadilhas comuns
    \item[\cmark] \textbf{Tema 1300 STF:} Tese REJEITADA, estratégia alternativa (nexo causal)
    \item[\cmark] \textbf{Visão Sistêmica:} Diferenças estruturais, checklist unificado
\end{itemize}

%% Continua no Módulo 8

%% ============================================================================
%% INFOGRÁFICO DO MÓDULO 7
%% ============================================================================
\clearpage
\backtotoc

\section*{\faImage\ Infográfico de Consolidação}

\begin{figure}[H]
    \centering
    \begin{tcolorbox}[colback=white, colframe=cjpAzulEscuro, title={\textbf{\faBookOpen\ Infográfico: Módulo 7 --- Benefícios Não Programáveis}}, fonttitle=\bfseries\color{white}, sharp corners=downhill, boxrule=2pt]
        \centering
        \includegraphics[width=0.95\textwidth, keepaspectratio]{modulo7}
    \end{tcolorbox}
    \caption{Resumo Visual do Módulo 7: Benefícios Não Programáveis}
    \label{fig:modulo7}
\end{figure}
