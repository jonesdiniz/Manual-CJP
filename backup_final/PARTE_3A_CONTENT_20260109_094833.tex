
\clearpage
\chapter*{Módulo 3: Armadilhas Ocultas}
\addcontentsline{toc}{chapter}{Módulo 3: Armadilhas Ocultas}
\markboth{Módulo 3: Armadilhas Ocultas}{Módulo 3: Armadilhas Ocultas}
\setcounter{chapter}{3}

\begin{center}
{\Large\textit{``O Diagnóstico Implícito: Os Erros Que o CNIS NÃO Mostra''}}\\[0.5cm]
\textbf{Sistema CJP | Pilar 2 de 5 | Parte 2: Diagnóstico Implícito}
\end{center}

\begin{novidade}
\textbf{Validade Temporal --- Janeiro de 2026}

\begin{itemize}
    \item[\cmark] Portaria DIRBEN/INSS n.\textsuperscript{o} 1.316/2025 (Indicadores CNIS)
    \item[\cmark] Tema 1090 STJ (EPI e ônus da prova - Abril/2025)
    \item[\cmark] Tema 1070 STJ (Atividades Concomitantes)
    \item[\cmark] Lei 14.331/2022 (Divisor Mínimo 108)
    \item[\cmark] PPP Eletrônico obrigatório desde 01/01/2023
\end{itemize}
\end{novidade}

\begin{acaoImediata}
``O INSS errou. Seu software não viu. Mas VOCÊ vai descobrir.''

Este módulo revela as 8 armadilhas mais comuns que passam despercebidas em 70-80\% dos cálculos previdenciários.

Cada armadilha pode representar \textbf{R\$ 50.000 - R\$ 300.000} de ganho vitalício para o cliente.
\end{acaoImediata}

%% \tableofcontents removido - sumário único no master

\section{A Diferença Entre Erro Explícito e Erro Implícito}

\subsection{Recapitulando o Pilar 2}

No \textbf{Módulo 2}, você aprendeu a auditar o \textbf{Diagnóstico Explícito} --- os erros que o INSS te mostra através de indicadores no CNIS.

Agora, no \textbf{Módulo 3}, você vai dominar o \textbf{Diagnóstico Implícito} --- os erros que o INSS \textbf{não te mostra}, mas que existem e custam caro.

\begin{conceitoChave}
\textbf{DIAGNÓSTICO EXPLÍCITO (Módulo 2)}

\begin{itemize}
    \item[\cmark] Erros sinalizados por INDICADORES no CNIS
    \item[\cmark] INSS ``sabe'' que há problema
    \item[\cmark] Há protocolo/código alfanumérico
    \item[\cmark] Resolução: Seguir procedimento específico
\end{itemize}

\textbf{DIAGNÓSTICO IMPLÍCITO (Módulo 3)}

\begin{itemize}
    \item[\xmark] Erros SEM sinalização no CNIS
    \item[\xmark] INSS ``não sabe'' que há problema
    \item[\xmark] Não há indicador/código de alerta
    \item[\xmark] Resolução: INVESTIGAÇÃO ATIVA pelo advogado
\end{itemize}
\end{conceitoChave}

\subsection{Por Que as Armadilhas Existem?}

\textbf{3 Causas Raiz:}

\textbf{1. Informações que nunca foram cadastradas}\\
Exemplo: Tempo rural da infância, tempo militar, atividade especial sem PPP

\textbf{2. Regras complexas que o sistema não aplica automaticamente}\\
Exemplo: Soma de atividades concomitantes (Tema 1070 STJ)

\textbf{3. Períodos com natureza ambígua}\\
Exemplo: Pecúlios classificados como ``indenização''

\begin{conceitoChave}
\textbf{ARMADILHA} = Informação relevante que:

\begin{itemize}
    \item NÃO aparece no CNIS automaticamente
    \item NÃO é detectada por softwares padrão
    \item REQUER análise humana especializada
    \item PODE aumentar o benefício em 15-40\%
\end{itemize}

A \textbf{Validação Estratégica Humana} é o diferencial competitivo do Método CJP.
\end{conceitoChave}

\subsection{O Custo do Erro Invisível}

\begin{table}[H]
\centering
\caption{Impacto Financeiro das 8 Armadilhas Ocultas}
\begin{tabular}{|l|l|}
\hline
\textbf{ARMADILHA} & \textbf{IMPACTO FINANCEIRO MÉDIO} \\
\hline
Atividades concomitantes & R\$ 240-800/mês \\
\hline
Pecúlios esquecidos & R\$ 4.500-12.000 \\
\hline
Tempo militar & +6-12 meses TC \\
\hline
Especial oculta & R\$ 400-1.200/mês \\
\hline
Salário-maternidade no PBC & R\$ 80-300/mês \\
\hline
Vínculos rurais & +12-36 meses TC \\
\hline
Contribuições em atraso & +3-18 meses TC \\
\hline
CTC mal aplicado & +12-60 meses TC \\
\hline
\end{tabular}
\end{table}

\begin{armadilha}
A estatística mais perigosa:

\begin{itemize}
    \item 73\% dos advogados confiam 100\% no CNIS sem auditoria profunda
    \item 89\% dos CNIS contêm pelo menos 1 erro ou omissão
\end{itemize}

\textit{(Pesquisa Método CJP, 2025)}

Não seja parte dessa estatística.
\end{armadilha}

\section{AS 8 ARMADILHAS OCULTAS}

\subsection{ARMADILHA \#1: Atividades Concomitantes Não Somadas}

\textbf{Tema 1070 STJ | Lei 8.213/91, arts. 29 e 32 | EC 103/2019}

\subsubsection{O que é}

Segurado que exerceu \textbf{duas ou mais atividades remuneradas no mesmo período} e não teve os salários de contribuição somados para cálculo da média.

\textbf{Exemplo típico:} Professor que dava aulas em escola pública pela manhã e escola particular à tarde.

\subsubsection{Por que acontece}

\textbf{Regra antiga:} O INSS aplicava cálculo com ``escala de salário-base'' para a atividade secundária.

\textbf{Regra atual (Tema 1070 STJ):} Para benefícios com DIB entre 01/04/2003 e 13/11/2019, os salários \textbf{devem ser somados} integralmente.

\subsubsection{Sinais de alerta}

\begin{itemize}
    \item[$\square$] CNIS mostra vínculos SIMULTÂNEOS (mesmo período, CNPJs diferentes)
    \item[$\square$] Cliente menciona ``dois empregos'' na entrevista
    \item[$\square$] Há ``contribuinte individual'' + ``empregado'' no mesmo mês
    \item[$\square$] DIB entre 01/04/2003 e 13/11/2019
\end{itemize}

\subsubsection{Estratégia de correção}

\begin{estrategiaCJP}
\textbf{REVISÃO ADMINISTRATIVA - TEMA 1070 STJ}

\begin{enumerate}
    \item Protocolar Pedido de Revisão no Meu INSS
    \item Fundamentar: Art. 29, I da Lei 8.213/91 + Tema 1070 STJ
    \item Anexar: CNIS, simulação e comprovantes das atividades
    \item Prazo INSS: 30 dias
    \item Se indeferido $\rightarrow$ Recurso à CRPS ou Ação Judicial
\end{enumerate}

\textbf{ATENÇÃO:} Respeitar prazo decadencial de 10 anos da DIB
\end{estrategiaCJP}

\textbf{Impacto financeiro:}\\
Ganho mensal típico: R\$ 240 - R\$ 800\\
Ganho total com retroativos: R\$ 15.000 - R\$ 50.000

\subsection{ARMADILHA \#2: Pecúlios e Abonos Esquecidos}

\textbf{Lei 8.870/1994 | Lei 8.212/91}

\subsubsection{O que é}

Valores recebidos como \textbf{pecúlio} (devolução de contribuições) ou \textbf{abonos} que foram incorretamente classificados como ``não contributivos''.

\textbf{Histórico:}
\begin{itemize}
    \item \textbf{Pecúlio:} Devolução de contribuições ao segurado (extinto em 1994)
    \item \textbf{Abono de Permanência:} Pago a quem continuava trabalhando após completar tempo (extinto em 1994)
\end{itemize}

\subsubsection{Sinais de alerta}

\begin{itemize}
    \item[$\square$] Cliente menciona ``recebi dinheiro do INSS'' antes de aposentar
    \item[$\square$] Há ``lacuna'' no CNIS entre períodos trabalhados (pré-1994)
    \item[$\square$] Houve ``mudança de regime'' (CLT $\rightarrow$ estatutário) com saque
\end{itemize}

\subsubsection{Estratégia de correção}

\begin{estrategiaCJP}
\textbf{INVESTIGAÇÃO DE PECÚLIO}

\begin{enumerate}
    \item Perguntar diretamente ao cliente na entrevista: ``Você já recebeu algum valor do INSS antes de aposentar?''
    \item Verificar lacunas no CNIS pré-1994
    \item Solicitar certidão de tempo de contribuição do período
    \item Requerer inclusão via RAC
\end{enumerate}
\end{estrategiaCJP}

\textbf{Impacto:} Pode significar 12-36 meses a mais de TC

\subsection{ARMADILHA \#3: Períodos de Afastamento Ignorados}

\textbf{Lei 8.213/91, art. 60 | CLT, art. 476}

\subsubsection{O que é}

Períodos de \textbf{afastamento do trabalho} (auxílio-doença, licença-maternidade, acidente de trabalho, serviço militar) não computados como tempo de contribuição.

\textbf{Tipos de afastamento computáveis:}
\begin{itemize}
    \item Auxílio-doença/Auxílio por incapacidade temporária
    \item Licença-maternidade
    \item Acidente de trabalho
    \item Serviço militar obrigatório
    \item Licença remunerada (com manutenção do vínculo)
\end{itemize}

\subsubsection{Regra importante}

O afastamento só conta como TC se \textbf{INTERCALADO} com períodos de atividade (antes e depois).

\subsubsection{Sinais de alerta}

\begin{itemize}
    \item[$\square$] CNIS mostra ``lacunas'' durante vínculo contínuo na CTPS
    \item[$\square$] Cliente menciona ``fiquei doente e recebi do INSS''
    \item[$\square$] Há registro de auxílio-doença mas período não conta como TC
    \item[$\square$] Cliente fez serviço militar e período não aparece
\end{itemize}

\subsubsection{Estratégia de correção}

\begin{estrategiaCJP}
\textbf{MAPEAMENTO DE AFASTAMENTOS}

\begin{enumerate}
    \item Comparar CNIS com CTPS (verificar lacunas)
    \item Solicitar ``Histórico de Benefícios'' no Meu INSS
    \item Verificar se afastamentos estão intercalados com atividade
    \item Confirmar que há vínculo ANTES e DEPOIS do afastamento
    \item Requerer cômputo via RAC
\end{enumerate}
\end{estrategiaCJP}

\textbf{Impacto:} +3 a +24 meses de TC

\subsection{ARMADILHA \#4: Tempo Militar e CTC Mal Aplicados}

\textbf{Lei 8.213/91, art. 55 | CF/88, art. 40, \S 9\textsuperscript{o}}

\subsubsection{O que é}

Tempo de \textbf{serviço militar obrigatório} ou tempo de \textbf{serviço público} (via CTC --- Certidão de Tempo de Contribuição) não averbado ou mal averbado no RGPS.

\subsubsection{Tipos de tempo afetados}

\begin{itemize}
    \item \textbf{Serviço militar obrigatório:} Conta como TC (não precisa comprovar contribuições)
    \item \textbf{Tempo público (estatutário):} Precisa de CTC para averbar no RGPS
    \item \textbf{Tempo como aluno-aprendiz:} Conta se recebia retribuição pecuniária
\end{itemize}

\subsubsection{Sinais de alerta}

\begin{itemize}
    \item[$\square$] Cliente menciona ``servi o exército''
    \item[$\square$] Cliente foi servidor público e pediu exoneração
    \item[$\square$] Há período sem contribuições correspondente a serviço militar
    \item[$\square$] CNIS não mostra indicador IPREVI mas cliente foi servidor
\end{itemize}

\subsubsection{Documentação necessária}

\begin{itemize}
    \item[$\square$] Certificado de Reservista (serviço militar)
    \item[$\square$] CTC do órgão público de origem
    \item[$\square$] Portaria de nomeação/exoneração
    \item[$\square$] Declaração do órgão militar
\end{itemize}

\subsubsection{Estratégia de correção}

\begin{itemize}
    \item \textbf{Serviço militar:} Solicitar averbação via Meu INSS + Certificado de Reservista
    \item \textbf{Tempo público:} Solicitar CTC ao órgão de origem $\rightarrow$ Averbar no RGPS
\end{itemize}

\textbf{Impacto:} +12 a +60 meses de TC

\subsection{ARMADILHA \#5: Aposentadoria Especial ``Oculta''}

\textbf{Lei 8.213/91, art. 57 | Tema 1090 STJ (Abril/2025)}

\begin{novidade}
Esta armadilha foi ATUALIZADA com o Tema 1090 STJ (Abril/2025) que redefiniu as regras sobre EPI e ônus da prova.
\end{novidade}

\subsubsection{O que é}

Tempo de trabalho em \textbf{atividade especial} (insalubre, perigosa ou penosa) que não foi reconhecido como tal, sendo computado apenas como tempo comum.

\subsubsection{Por que acontece}

\begin{enumerate}
    \item \textbf{PPP não foi anexado} ao requerimento
    \item \textbf{Empresa informou EPI eficaz} no PPP
    \item \textbf{LTCAT não foi elaborado} corretamente
    \item \textbf{Código de atividade} no CNIS é genérico
    \item \textbf{Empresa fechou} sem entregar documentação
\end{enumerate}

\begin{novidade}
\textbf{TEMA 1090 STJ: EPI E ÔNUS DA PROVA (Abril/2025)}

\textbf{Tese Fixada em 3 Partes:}

\textbf{I.} Informação de EPI eficaz no PPP DESCARACTERIZA tempo especial, \textbf{EXCETO} para:
\begin{itemize}
    \item Ruído (Tema 555 STF)
    \item Agentes biológicos
    \item Agentes cancerígenos
    \item Periculosidade
\end{itemize}

\textbf{II.} ÔNUS DO SEGURADO comprovar ineficácia do EPI:
\begin{itemize}
    \item Inadequação ao risco específico
    \item Irregularidade no CA (Certificado de Aprovação)
    \item Descumprimento de normas de manutenção/substituição
    \item Ausência de treinamento adequado
\end{itemize}

\textbf{III.} Em caso de DÚVIDA sobre eficácia $\rightarrow$ Conclusão FAVORÁVEL ao segurado
\end{novidade}

\subsubsection{Sinais de alerta}

\begin{itemize}
    \item[$\square$] Cliente trabalhou em atividades insalubres/perigosas
    \item[$\square$] CNIS mostra vínculo como ``comum'' mas CTPS indica função especial
    \item[$\square$] Cliente usava EPI (pode ser ineficaz --- verificar!)
    \item[$\square$] Há indicador IEAN no CNIS (exposição detectada mas não convertida)
\end{itemize}

\subsubsection{Estratégia de correção}

\begin{estrategiaCJP}
\textbf{FLUXO DE ANÁLISE PÓS-TEMA 1090}

\begin{enumerate}
    \item Obter PPP (ou PPP Eletrônico para períodos após 01/01/2023)
    \item Verificar se há informação de ``EPI eficaz''
    \item Se agente é ruído, biológico, cancerígeno ou periculosidade: \textbf{Tempo especial mantido} (EPI não elide)
    \item Se outro agente: Analisar pontos de impugnação do PPP
    \item Compilar provas de ineficácia do EPI
    \item Requerer conversão via RAC ou ação judicial
\end{enumerate}
\end{estrategiaCJP}

\textbf{Conversão (antes da Reforma):}\\
1 ano especial = 1,4 anos comuns (homem) ou 1,2 anos (mulher)

\textbf{Impacto:} +25-40\% no tempo de contribuição + aumento da RMI

%% Continua na Parte 3B
