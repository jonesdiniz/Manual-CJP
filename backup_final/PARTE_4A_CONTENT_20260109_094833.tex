\clearpage
\chapter*{Módulo 4: Acertos e Documentação}
\addcontentsline{toc}{chapter}{Módulo 4: Acertos e Documentação}
\markboth{Módulo 4: Acertos e Documentação}{Módulo 4: Acertos e Documentação}
\setcounter{chapter}{4}

\begin{center}
{\Large\textit{``Da Descoberta à Correção: Como Transformar Diagnóstico em Resultado''}}\\[0.5cm]
\textbf{Sistema CJP | Pilar 3 de 5 | Acertos de Vínculos}
\end{center}

\begin{acaoImediata}
``Descobrir o erro é 30\% do trabalho. PROVAR e CORRIGIR o erro é 70\%.''

Este módulo é o manual operacional que transforma diagnóstico em resultado.

Você vai aprender exatamente COMO protocolar, QUAIS documentos anexar e COMO acompanhar cada tipo de acerto.
\end{acaoImediata}

%% \tableofcontents removido - sumário único no master

\section{O Pilar 4: Por Que Acertos São Estratégicos}

\subsection{O Problema da Maioria dos Advogados}

Você fez o diagnóstico perfeito. Descobriu as 8 armadilhas. Calculou o impacto: \textbf{R\$ 320.000 de ganho vitalício} se o CNIS for corrigido.

Mas então você faz o seguinte:
\begin{itemize}
    \item[\xmark] Protocola o pedido de aposentadoria \textbf{sem} corrigir o CNIS antes
    \item[\xmark] Conta com o servidor do INSS para ``descobrir e corrigir'' os erros
    \item[\xmark] Junta provas desorganizadas, sem parecer técnico
\end{itemize}

\textbf{Resultado?} Benefício concedido com valor errado.

\begin{armadilha}
\textbf{ARMADILHA FATAL: Aposentar Sem Acertos Prévios}

Um CNIS incorreto gera um benefício incorreto. E corrigir um benefício CONCEDIDO (revisão) é:
\begin{itemize}
    \item Mais complexo (precisa provar erro de cálculo)
    \item Mais demorado (análise de revisão = 60-120 dias)
    \item Mais arriscado (decadência de 10 anos)
\end{itemize}

\textbf{ESTRATÉGIA CJP:} Corrigir o CNIS ANTES da concessão.
\end{armadilha}

\subsection{Diferença Entre RAC, Pedido de Revisão e Revisão Judicial}

\begin{table}[H]
    \centering
    \caption{Diferença Entre RAC, Pedido de Revisão e Revisão Judicial}
    \begin{tabular}{|l|p{3cm}|p{2.5cm}|l|}
    \hline
    \textbf{Instrumento} & \textbf{Objetivo} & \textbf{Timing} & \textbf{Prazo INSS} \\
    \hline
    RAC & Corrigir CNIS ANTES da concessão & ANTES do benefício & 45 dias \\
    \hline
    Pedido de Revisão & Corrigir benefício APÓS concessão & APÓS concessão & 45-60 dias \\
    \hline
    Revisão Judicial & Corrigir via Justiça Federal & Após via admin. & Variável \\
    \hline
    \end{tabular}
\end{table}

\begin{estrategiaCJP}
\textbf{TIMING DOS ACERTOS}

\textbf{CENÁRIO 1 (IDEAL):}\\
Diagnóstico $\rightarrow$ RAC $\rightarrow$ CNIS Perfeito $\rightarrow$ Pedido Benefício

\textbf{CENÁRIO 2 (REATIVO):}\\
Benefício Concedido $\rightarrow$ Diagnóstico $\rightarrow$ Revisão

O Método CJP prioriza o CENÁRIO 1. Evite o Cenário 2.
\end{estrategiaCJP}

\section{O RAC (Requerimento de Acerto de CNIS)}

\subsection{O Que É o RAC?}

O \textbf{RAC} é o instrumento administrativo previsto na Portaria DIRBEN/INSS n.\textsuperscript{o} 1.297/2025 para \textbf{retificar dados incorretos} no CNIS \textbf{antes} da concessão de benefícios.

\textbf{Base Legal:}
\begin{itemize}
    \item Portaria DIRBEN/INSS n.\textsuperscript{o} 1.297/2025
    \item Portaria DIRBEN/INSS n.\textsuperscript{o} 1.316/2025 (Indicadores CNIS)
    \item Lei 8.213/91, Art. 29-A
    \item Decreto 3.048/99, Art. 19
\end{itemize}

\subsection{Acesso via Procuração Eletrônica (Novidade 2025)}

\begin{novidade}
\textbf{REVOLUÇÃO NO ACESSO AO CNIS}

A Portaria Conjunta DIT/DIRBEN/INSS n.\textsuperscript{o} 10/2025 (vigente desde 13/11/2025) instituiu a PROCURAÇÃO ELETRÔNICA no Meu INSS.

\textbf{ANTES:} ``Cliente, preciso da sua senha...''\\
\textbf{DEPOIS:} ``Vou acessar via procuração eletrônica.''

Isso FACILITA o diagnóstico e a instrução do RAC!
\end{novidade}

\textbf{Requisitos:}
\begin{itemize}
    \item Cliente (Representado): Conta gov.br nível PRATA ou OURO
    \item Advogado (Representante): Conta gov.br nível PRATA ou OURO
\end{itemize}

\textbf{Serviços Acessíveis:}
\begin{itemize}
    \item[\cmark] CNIS completo (essencial para diagnóstico)
    \item[\cmark] Consultas de documentos e serviços
    \item[\cmark] Extrato de pagamentos
    \item[\cmark] Carta de concessão
    \item[\xmark] NÃO permite protocolar requerimentos
    \item[\xmark] NÃO permite recurso ou manifestação
\end{itemize}

\subsection{Tema 1124 STJ: A Importância da Instrução Completa}

\begin{novidade}
\textbf{TEMA 1124 STJ (Outubro/2025)}

O STJ estabeleceu parâmetros sobre interesse de agir previdenciário que IMPACTAM o RAC e futuras ações judiciais.

\textbf{REGRA:} O segurado DEVE apresentar requerimento administrativo APTO com documentação SUFICIENTE.

\textbf{CONSEQUÊNCIA PRÁTICA:}
\begin{itemize}
    \item Se provas já estavam no PA $\rightarrow$ DIB na DER
    \item Se provas não foram levadas $\rightarrow$ DIB na citação
\end{itemize}

\textbf{ESTRATÉGIA CJP:} Instruir o RAC COMPLETAMENTE, com TODAS as provas disponíveis, para garantir DIB mais favorável.
\end{novidade}

\subsection{Quando Usar o RAC?}

Use o RAC quando identificar:
\begin{itemize}
    \item[\cmark] Vínculos ausentes (PEXT)
    \item[\cmark] Salários incorretos
    \item[\cmark] Contribuições abaixo do mínimo (PSC-MEN-SM-EC103)
    \item[\cmark] Tempo especial não reconhecido
    \item[\cmark] Atividades concomitantes não somadas
    \item[\cmark] Vínculos rurais ausentes
    \item[\cmark] Tempo militar não averbado
    \item[\cmark] CTC não incluídas
    \item[\cmark] Salário-maternidade excluído do PBC
    \item[\cmark] Pecúlios e abonos não computados
\end{itemize}

\begin{conceitoChave}
\textbf{HIERARQUIA PROBATÓRIA}

O INSS aceita o RAC se você apresentar provas SUFICIENTES:

\textbf{1\textsuperscript{o} - DOCUMENTOS ORIGINAIS OFICIAIS}\\
(CTPS, PPP, Certidões, Holerites originais)

\textbf{2\textsuperscript{o} - DOCUMENTOS COMPLEMENTARES}\\
(Extrato FGTS, Ficha de Registro, Contracheque)

\textbf{3\textsuperscript{o} - PROVAS TESTEMUNHAIS/INDIRETAS}\\
(Declarações de terceiros, fotos --- aceitos apenas em casos especiais)

\textbf{NUNCA} protocole RAC sem pelo menos 1 prova do nível 1.
\end{conceitoChave}

\subsection{Procedimento Completo do RAC}

\textbf{PASSO 1: COLETA E ORGANIZAÇÃO DAS PROVAS}

Regra CJP: 1 acerto = 1 prova de nível 1 (mínimo)

\textbf{PASSO 2: ELABORAÇÃO DO REQUERIMENTO}

\textbf{Estrutura Mínima do RAC:}
\begin{enumerate}
    \item \textbf{Identificação:} Nome, CPF, NIT, endereço, contato
    \item \textbf{Objeto:} ``Requer a retificação do CNIS para incluir/corrigir...''
    \item \textbf{Fundamentação:} Portarias, Lei 8.213/91, Decreto 3.048/99
    \item \textbf{Fatos:} Detalhamento de cada acerto com período exato, empresa, CNPJ, prova
    \item \textbf{Direito:} Por que o acerto é devido (citar jurisprudência se relevante)
    \item \textbf{Pedidos:} Lista clara de cada correção
    \item \textbf{Documentos:} Relação numerada de anexos
    \item \textbf{Data e Assinatura}
\end{enumerate}

\textbf{PASSO 3: PROTOCOLO NO MEU INSS}

\textbf{Opção Online (Recomendado):}
\begin{enumerate}
    \item Acesse meu.inss.gov.br
    \item Login com Gov.br do segurado
    \item Selecione: Novo Pedido $\rightarrow$ Atualizar Meus Dados $\rightarrow$ Acerto de CNIS
    \item Anexe o Requerimento (PDF) e todos os documentos
    \item Protocolo gerado automaticamente
\end{enumerate}

\begin{acaoImediata}
\textbf{DICA: Qualidade da Digitalização}

Digitalize documentos em ALTA RESOLUÇÃO (mínimo 300 DPI).

PDFs ilegíveis geram indeferimento automático.
\end{acaoImediata}

\textbf{PASSO 4: ACOMPANHAMENTO}

Prazo: 45 dias (Portaria 1.297/2025)

\textbf{Status possíveis:}
\begin{itemize}
    \item ``Em análise'' $\rightarrow$ Aguardando
    \item ``Exigência'' $\rightarrow$ INSS solicitou documentos adicionais (responda em 30 dias!)
    \item ``Deferido'' $\rightarrow$ Acerto realizado
    \item ``Indeferido'' $\rightarrow$ Cabe recurso em 30 dias
\end{itemize}

\begin{armadilha}
\textbf{EXIGÊNCIA NÃO RESPONDIDA}

Se o INSS publicar uma EXIGÊNCIA (pedido de mais documentos), você tem 30 DIAS para responder.

\textbf{NÃO RESPONDER = INDEFERIMENTO AUTOMÁTICO.}

Configure ALERTA no MEU INSS para não perder prazo!
\end{armadilha}

\textbf{PASSO 5: RECURSO (Se Indeferido)}

Prazo: 30 dias para Recurso Ordinário ao CRPS

\textbf{Estrutura:}
\begin{enumerate}
    \item Identificação do Ato Recorrido
    \item Fundamentação da Discordância
    \item Reforço Probatório (novos documentos, se houver)
    \item Pedido de Reforma da Decisão
\end{enumerate}

\section{Dossiê Probatório: A Arte da Documentação Estratégica}

\subsection{A Pirâmide de Hierarquia Probatória}

\begin{estrategiaCJP}
\textbf{NÍVEL 1 (Força Máxima):}
\begin{itemize}
    \item CTPS (páginas de contrato)
    \item PPP/LTCAT assinados
    \item Certidões (Casamento, Nascimento)
    \item Certificado de Reservista
    \item CTC (Certidão Tempo Contribuição)
\end{itemize}

\textbf{NÍVEL 2 (Força Média):}
\begin{itemize}
    \item Extrato FGTS (CEF)
    \item Holerites/Contracheques
    \item Ficha de Registro de Empregado
    \item Histórico Escolar (zona rural)
    \item Declaração do Empregador
\end{itemize}

\textbf{NÍVEL 3 (Força Baixa):}
\begin{itemize}
    \item Declarações de terceiros
    \item Fotos/Vídeos
    \item Documentos parciais
    \item Testemunhas
\end{itemize}
\end{estrategiaCJP}

\textbf{Regra de Ouro:} Para cada acerto, anexe pelo menos 1 documento do Nível 1 OU 2 documentos do Nível 2 combinados.

\subsection{Documentos por Tipo de Acerto}

\subsubsection{ACERTO 1: Vínculos PEXT (Período Externo)}

\textbf{Documentos Necessários:}
\begin{table}[H]
\centering
\caption{Documentos para Acerto de Vínculos PEXT}
\begin{tabular}{|c|l|p{6cm}|}
\hline
\textbf{Nível} & \textbf{Documento} & \textbf{Observação} \\
\hline
1 & CTPS (cópia autenticada) & ESSENCIAL - admissão, demissão, assinatura \\
\hline
2 & Extrato Analítico FGTS & Confirma depósitos da empresa \\
\hline
2 & Ficha de Registro & Se disponível no arquivo da empresa \\
\hline
2 & Holerites do Período & Primeiro, último e um do meio \\
\hline
2 & TRCT & Comprova fim do vínculo \\
\hline
\end{tabular}
\end{table}

\begin{acaoImediata}
\textbf{EMPRESAS EXTINTAS/FALIDAS}

Se a empresa não existe mais, a CTPS ganha força MÁXIMA.

Nestes casos, adicione:
\begin{itemize}
    \item Certidão de falência/encerramento (Junta Comercial)
    \item Qualquer documento que prove que a empresa existiu
\end{itemize}
\end{acaoImediata}

\subsubsection{ACERTO 2: Contribuições PSC-MEN-SM-EC103}

Contribuições de valor inferior ao salário mínimo após a Reforma (EC 103/2019).

\textbf{Estratégia:} Requerer complementação das contribuições junto à Receita Federal ou apresentar comprovantes de pagamento correto.

\subsubsection{ACERTO 3: Tempo Especial Não Reconhecido}

\textbf{Documentos Necessários:}
\begin{itemize}
    \item PPP (Perfil Profissiográfico Previdenciário)
    \item LTCAT (Laudo Técnico das Condições Ambientais)
    \item Desde 01/01/2023: PPP Eletrônico obrigatório via eSocial
    \item Provas de ineficácia do EPI (se aplicável - Tema 1090)
\end{itemize}

\subsubsection{ACERTO 4: Atividades Concomitantes (Tema 1070)}

\textbf{Documentos Necessários:}
\begin{itemize}
    \item CNIS detalhado mostrando vínculos simultâneos
    \item CTPS de ambos os vínculos
    \item Contracheques de ambas as atividades
    \item Planilha de soma dos salários
\end{itemize}

\subsubsection{ACERTO 5: Vínculos Rurais}

\textbf{Documentos (Início de Prova Material):}
\begin{itemize}
    \item Certidão de nascimento (local rural)
    \item Certidão de casamento (profissão lavrador)
    \item Histórico escolar de escola rural
    \item Ficha de alistamento militar
    \item Documentos de terra em nome da família
    \item ITR, cadastro INCRA
\end{itemize}

\subsubsection{ACERTO 6: Tempo Militar e CTC}

\textbf{Documentos:}
\begin{itemize}
    \item Certificado de Reservista
    \item Certidão de tempo de serviço militar
    \item CTC do órgão público de origem (para RPPS)
\end{itemize}

%% Continua na Parte 4B
