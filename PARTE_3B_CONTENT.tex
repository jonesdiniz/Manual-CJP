%% Continuação do Módulo 3: Armadilhas Ocultas

\section{ARMADILHA \#6: Salário-Maternidade Excluído do PBC}

\textbf{Art. 28, §2º, Lei 8.212/91}

\textbf{Impacto financeiro:} R\$ 80 - R\$ 300/mês | Ganho vitalício: R\$ 19.200 - R\$ 72.000

\subsection{O Que É}

O \textbf{salário-maternidade} integra o salário de contribuição e deve ser incluído no cálculo do PBC.

\textbf{O problema:} Muitas vezes o INSS:
\begin{itemize}
    \item Exclui o período de salário-maternidade do PBC
    \item Considera como ``lacuna'' contributiva
    \item Não inclui o valor na média de salários
\end{itemize}

\subsection{Sinais de Alerta}

\begin{itemize}
    \item[$\square$] Cliente do sexo feminino com filhos?
    \item[$\square$] Estava empregada quando engravidou?
    \item[$\square$] Há períodos de 4-6 meses ``sem contribuição'' que coincidem com nascimento de filhos?
    \item[$\square$] Cálculo do INSS excluiu meses de licença-maternidade?
\end{itemize}

\subsection{Solução CJP}

\begin{enumerate}
    \item \textbf{Identificar todos os períodos de salário-maternidade} --- Cruzar datas de nascimento dos filhos com vínculos
    \item \textbf{Verificar se estão no PBC} --- Analisar carta de concessão/memória de cálculo
    \item \textbf{Se excluídos, solicitar revisão} --- Fundamento: Art. 28, §2º, Lei 8.212/91
    \item \textbf{Avaliar impacto} --- Se salário-maternidade era MAIOR que a média → aumenta benefício
\end{enumerate}

\begin{conceitoChave}
O salário-maternidade é considerado \textbf{SALÁRIO DE CONTRIBUIÇÃO}.

Deve ser incluído no PBC como se fosse remuneração normal do vínculo.

\textbf{INSS costuma ``esquecer'' isso!}
\end{conceitoChave}

\section{ARMADILHA \#7: Vínculos Rurais Ignorados}

\textbf{Art. 55, §2º, Lei 8.213/91}

\textbf{Impacto financeiro:} +12 a +36 meses TC | Pode ser DECISIVO para cumprir carência

\subsection{O Que É}

\textbf{Tempo rural} é o período trabalhado em atividade agrícola, pecuária, extrativista ou pesqueira.

\textbf{Características especiais:}
\begin{itemize}
    \item Pode ser comprovado sem registro em carteira (início de prova material + testemunhas)
    \item Conta para tempo de contribuição e carência
    \item Pode ser somado com tempo urbano para aposentadoria híbrida
\end{itemize}

\subsection{Por Que Não Aparece no CNIS}

\begin{itemize}
    \item Trabalho rural familiar raramente tinha registro formal
    \item Períodos anteriores a 1991 podem não ter migrado para o CNIS
    \item CNIS não mostra ``tempo rural presumido''
\end{itemize}

\subsection{Sinais de Alerta}

\begin{itemize}
    \item[$\square$] Cliente nasceu ou cresceu em zona rural?
    \item[$\square$] Pais eram agricultores, pescadores, extrativistas?
    \item[$\square$] Trabalhou na roça na infância/adolescência?
    \item[$\square$] Estudou em escola rural?
    \item[$\square$] Mudou-se para cidade após os 18 anos?
    \item[$\square$] Há lacunas no CNIS antes do primeiro emprego urbano?
\end{itemize}

\subsection{Documentação para Início de Prova Material}

\begin{itemize}
    \item Certidão de nascimento (local rural)
    \item Certidão de casamento (profissão: lavrador/agricultor)
    \item Histórico escolar de escola rural
    \item Ficha de alistamento militar
    \item Documentos de terra em nome da família
    \item Comprovantes de ITR, cadastro INCRA
    \item Notas fiscais de produção rural
\end{itemize}

\begin{novidade}
\textbf{IN 188/2025 --- Aposentadoria Híbrida Flexibilizada}

Agora é mais fácil somar tempo rural + urbano para cumprir requisitos.

Verifique se seu cliente se beneficia!
\end{novidade}

\section{ARMADILHA \#8: Contribuições em Atraso Não Computadas}

\textbf{Art. 45-A, Lei 8.212/91}

\textbf{Impacto financeiro:} +3 a +18 meses TC | Pode completar carência/tempo mínimo

\subsection{O Que É}

\textbf{Contribuições em atraso} são pagamentos de GPS referentes a períodos em que o segurado:
\begin{itemize}
    \item Era contribuinte individual mas não pagou na época
    \item Exerceu atividade sem vínculo formal
    \item Perdeu prazo de pagamento
\end{itemize}

\textbf{Tipos:}
\begin{itemize}
    \item \textbf{Indenização (art. 45-A):} Para períodos de CI sem comprovação de atividade
    \item \textbf{Retroação de DIC:} Reconhecimento de atividade com recolhimento posterior
    \item \textbf{Débito em atraso:} GPS vencidas não pagas
\end{itemize}

\subsection{Sinais de Alerta}

\begin{itemize}
    \item[$\square$] Cliente trabalhou como autônomo/MEI em algum período?
    \item[$\square$] Há lacunas no CNIS em que o cliente tinha alguma atividade?
    \item[$\square$] Faltam poucos meses para completar tempo/carência?
    \item[$\square$] Cliente tem comprovantes de atividade (recibos, contratos)?
\end{itemize}

\subsection{Solução CJP}

\begin{enumerate}
    \item \textbf{Mapear lacunas no CNIS}
    \item \textbf{Classificar o tipo de lacuna:}
    \begin{itemize}
        \item Com atividade comprovada → Retroação de DIC
        \item Sem comprovação → Indenização (art. 45-A)
        \item GPS vencida → Débito em atraso
    \end{itemize}
    \item \textbf{Calcular custo-benefício}
    \item \textbf{Formalizar o recolhimento}
\end{enumerate}

\begin{armadilha}
\textbf{CÁLCULO OBRIGATÓRIO}

Antes de pagar contribuições em atraso, CALCULE se o investimento vale a pena!

\textbf{Fórmula:}\\
GANHO = (Nova RMI - RMI atual) × Expectativa de vida em meses

Se GANHO \textgreater\ CUSTO → Vale pagar\\
Se GANHO \textless\ CUSTO → Não vale
\end{armadilha}

\section{TEMA 1090 STJ: EPI E ÔNUS DA PROVA (Detalhamento Completo)}

\subsection{A Mudança Paradigmática de Abril/2025}

O julgamento do \textbf{Tema 1090 STJ} (REsp repetitivo), concluído em \textbf{09 de abril de 2025}, representa uma das mais importantes definições jurisprudenciais sobre aposentadoria especial.

\begin{novidade}
\textbf{TESE FIXADA --- TEMA 1090 STJ (ABRIL/2025)}

\textbf{I.} A informação no PPP sobre existência de EPI eficaz DESCARACTERIZA, em princípio, o tempo especial, EXCETO nas hipóteses excepcionais (ex: ruído - Tema 555 STF).

\textbf{II.} Cabe ao SEGURADO o ônus de comprovar a ineficácia do EPI, demonstrando:
\begin{itemize}
    \item Inadequação ao risco
    \item Irregularidade no certificado de conformidade
    \item Descumprimento de normas de manutenção/substituição
    \item Ausência de treinamento adequado
\end{itemize}

\textbf{III.} Em caso de DÚVIDA ou DIVERGÊNCIA sobre a real eficácia do EPI, a conclusão será FAVORÁVEL ao segurado.

\textit{Acórdão publicado: 22/04/2025}
\end{novidade}

\subsection{Tabela: Agentes Nocivos × Impacto do EPI Eficaz}

\begin{table}[H]
\centering
\caption{Agentes Nocivos × Impacto do EPI Eficaz (Tema 1090 STJ)}
\begin{tabular}{|l|c|l|}
\hline
\textbf{Agente Nocivo} & \textbf{EPI Eficaz Descaracteriza?} & \textbf{Fundamento} \\
\hline
Ruído & NÃO & Tema 555 STF \\
\hline
Agentes biológicos & NÃO & Exceção Tema 1090 \\
\hline
Agentes cancerígenos & NÃO & Exceção Tema 1090 \\
\hline
Periculosidade & NÃO & Exceção Tema 1090 \\
\hline
Calor & SIM & Regra geral \\
\hline
Frio & SIM & Regra geral \\
\hline
Vibração & SIM & Regra geral \\
\hline
Agentes químicos* & SIM & Regra geral \\
\hline
\end{tabular}
\end{table}

\textit{*Alguns químicos são cancerígenos e entram na exceção}

\subsection{Checklist de Pontos Atacáveis no PPP}

\begin{estrategiaCJP}
\textbf{IRREGULARIDADES FORMAIS:}
\begin{itemize}
    \item[$\square$] EPI sem Certificado de Aprovação (CA) válido
    \item[$\square$] CA vencido durante o período de uso
    \item[$\square$] EPI inadequado para o agente específico
    \item[$\square$] Ausência de registro de entrega
\end{itemize}

\textbf{IRREGULARIDADES DE MANUTENÇÃO:}
\begin{itemize}
    \item[$\square$] Sem programa de substituição periódica
    \item[$\square$] Sem higienização regular
    \item[$\square$] Prazo de validade excedido
\end{itemize}

\textbf{IRREGULARIDADES DE USO:}
\begin{itemize}
    \item[$\square$] Sem treinamento documentado
    \item[$\square$] Sem fiscalização do uso
    \item[$\square$] Condições de trabalho incompatíveis com uso do EPI
\end{itemize}

\textbf{CONTRADIÇÕES NO PPP:}
\begin{itemize}
    \item[$\square$] PPP indica EPI eficaz mas há adicional de insalubridade
    \item[$\square$] LTCAT da empresa contradiz o PPP
    \item[$\square$] Perícia trabalhista reconheceu insalubridade no mesmo período
    \item[$\square$] CAT emitida no período
\end{itemize}
\end{estrategiaCJP}

\subsection{Estratégia: In Dubio Pro Segurado}

\begin{conceitoChave}
\textbf{PARTE III DA TESE:}

Se houver dúvida ou divergência sobre a eficácia real do EPI, a conclusão deve ser \textbf{favorável ao segurado}.

\textbf{ESTRATÉGIA CJP:}

Se não conseguir PROVAR ineficácia, tente criar DÚVIDA sobre a eficácia.

Dúvida = Resultado favorável ao segurado
\end{conceitoChave}

\section{Matriz de Detecção: Sinais de Alerta por Armadilha}

\begin{estrategiaCJP}
\textbf{ARMADILHA \#1: CONCOMITANTES}
\begin{itemize}
    \item[\cmark] Múltiplos vínculos no mesmo período no CNIS
    \item[\cmark] Professor, médico, enfermeiro, advogado
    \item[\cmark] DER entre 1999-2019
\end{itemize}

\textbf{ARMADILHA \#2: PECÚLIOS}
\begin{itemize}
    \item[\cmark] Aposentadoria antes de 1994
    \item[\cmark] Continuou trabalhando após aposentar
    \item[\cmark] Cliente idoso (70+ anos)
\end{itemize}

\textbf{ARMADILHA \#3: AFASTAMENTOS}
\begin{itemize}
    \item[\cmark] Lacunas no CNIS que coincidem com vínculos na CTPS
    \item[\cmark] Histórico de acidentes/doenças
    \item[\cmark] Benefícios por incapacidade no histórico
\end{itemize}

\textbf{ARMADILHA \#4: MILITAR/CTC}
\begin{itemize}
    \item[\cmark] Homem nascido antes de 1980
    \item[\cmark] Ex-servidor público
    \item[\cmark] Lacuna entre 18-19 anos
\end{itemize}

\textbf{ARMADILHA \#5: ESPECIAL OCULTA}
\begin{itemize}
    \item[\cmark] Profissões de risco
    \item[\cmark] Nunca ouviu falar em PPP
    \item[\cmark] PPP indica EPI eficaz (verificar exceções Tema 1090)
\end{itemize}

\textbf{ARMADILHA \#6: SALÁRIO-MATERNIDADE}
\begin{itemize}
    \item[\cmark] Mulher com filhos
    \item[\cmark] Estava empregada durante gravidez
    \item[\cmark] Meses ``sem contribuição'' coincidem com licença
\end{itemize}

\textbf{ARMADILHA \#7: RURAL}
\begin{itemize}
    \item[\cmark] Nascido/criado em zona rural
    \item[\cmark] Pais agricultores/pescadores
    \item[\cmark] Lacunas no CNIS antes do primeiro emprego urbano
\end{itemize}

\textbf{ARMADILHA \#8: CONTRIBUIÇÕES EM ATRASO}
\begin{itemize}
    \item[\cmark] Trabalhou como autônomo/MEI
    \item[\cmark] Lacunas no CNIS com atividade comprovável
    \item[\cmark] Faltam poucos meses para completar requisitos
\end{itemize}
\end{estrategiaCJP}

\section{Checklist Master Anti-Armadilhas}

\begin{acaoImediata}
\textbf{CHECKLIST MASTER --- PILAR 2 PARTE 2: ARMADILHAS OCULTAS (2026)}

\textbf{ETAPA 1: DETECÇÃO (Durante Entrevista)}
\begin{itemize}
    \item[$\square$] \#1 Concomitantes: Perguntei sobre dois empregos simultâneos?
    \item[$\square$] \#2 Pecúlios: Perguntei sobre valores recebidos do INSS pré-aposentadoria?
    \item[$\square$] \#3 Afastamentos: Perguntei sobre licenças, auxílios-doença, acidentes?
    \item[$\square$] \#4 Militar/CTC: Perguntei sobre serviço militar e tempo público?
    \item[$\square$] \#5 Especial: Perguntei sobre exposição a agentes nocivos?
    \item[$\square$] \#6 Sal-Maternidade: Perguntei sobre filhos e licenças?
    \item[$\square$] \#7 Rural: Perguntei sobre infância no campo?
    \item[$\square$] \#8 Atraso: Perguntei sobre trabalho autônomo não declarado?
\end{itemize}

\textbf{ETAPA 2: CONFIRMAÇÃO (Análise Documental)}
\begin{itemize}
    \item[$\square$] Cruzei CNIS com CTPS para identificar lacunas?
    \item[$\square$] Verifiquei vínculos simultâneos no mesmo período?
    \item[$\square$] Solicitei PPP de todas as empresas relevantes?
    \item[$\square$] Analisei campo EPI do PPP conforme Tema 1090?
    \item[$\square$] Busquei documentos de início de prova material (rural)?
    \item[$\square$] Calculei custo-benefício de contribuições em atraso?
\end{itemize}

\textbf{ETAPA 3: CORREÇÃO}
\begin{itemize}
    \item[$\square$] Armadilhas identificadas listadas
    \item[$\square$] Estratégia de correção definida para cada uma
    \item[$\square$] Documentação reunida
    \item[$\square$] Requerimentos administrativos protocolados
    \item[$\square$] Via judicial preparada (se necessário)
\end{itemize}
\end{acaoImediata}

\section{Conclusão do Módulo 3}

Você dominou o \textbf{Diagnóstico Implícito} do Pilar 2.

\textbf{O que você aprendeu:}
\begin{itemize}
    \item[\cmark] As 8 Armadilhas Ocultas e como detectá-las
    \item[\cmark] O impacto do Tema 1090 STJ sobre EPI
    \item[\cmark] Como impugnar declaração de EPI eficaz
    \item[\cmark] A matriz de sinais de alerta
    \item[\cmark] Estratégias de correção para cada armadilha
\end{itemize}

\section{Próximo Passo}

No \textbf{Módulo 4}, você aprenderá o \textbf{Pilar 3: Acertos e Documentação} --- como transformar as descobertas dos Módulos 2 e 3 em provas documentais sólidas.

%% Continua na Parte 4A (Módulo 4)

%% ============================================================================
%% INFOGRÁFICO DO MÓDULO 3
%% ============================================================================
\clearpage
\backtotoc

\section*{\faImage\ Infográfico de Consolidação}

\begin{figure}[H]
    \centering
    \begin{tcolorbox}[colback=white, colframe=cjpAzulEscuro, title={\textbf{\faBookOpen\ Infográfico: Módulo 3 --- Armadilhas Ocultas}}, fonttitle=\bfseries\color{white}, sharp corners=downhill, boxrule=2pt]
        \centering
        \includegraphics[width=0.95\textwidth, keepaspectratio]{modulo3}
    \end{tcolorbox}
    \caption{Resumo Visual do Módulo 3: Armadilhas Ocultas}
    \label{fig:modulo3}
\end{figure}
